
\documentclass[10pt]{article}
\usepackage[utf8]{inputenc}
\usepackage{amsmath}
\usepackage{amssymb}
\usepackage{geometry}
\usepackage{xcolor}
\usepackage{longtable}
\usepackage{array}
\usepackage{lmodern}
\usepackage{fancyvrb}

\geometry{a4paper, margin=1in}

\definecolor{correct}{HTML}{28a745}
\definecolor{partial}{HTML}{FFC107}
\definecolor{incorrect}{HTML}{DC3545}
\definecolor{noerror}{HTML}{6C757D}

\title{LLM Benchmark Report}
\author{Generated by script}
\date{\today}

\begin{document}
\maketitle

\section*{Summary Statistics}
\begin{center}
\begin{longtable}{|l|c|c|c|c|c|c|}
\hline
\textbf{Model} & \textbf{Total} & \textbf{Correct} & \textbf{Partial} & \textbf{Incorrect} & \textbf{Errors} & \textbf{Avg. Score} \\
\hline
\endfirsthead
\hline
\endfoot
o4-mini & 65 & 33 & 7 & 20 & 5 & 0.61 \\
\hline
\end{longtable}
\end{center}
\section*{Problem Details}

\subsection*{Problem 1 (Paper: 2506.20163v1)}
\subsubsection*{Problem Statement}
Background:
Consider a dynamical system defined on a 1-simplicial complex (a network) composed of $N_n$ nodes (0-simplices) and $N_l$ links (1-simplices). The state of the system is described by a set of variables on the nodes, called a 0-cochain and denoted by the vector $\vec{u} \in \mathbb{R}^{N_n}$, and a set of variables on the links, called a 1-cochain and denoted by the vector $\vec{v} \in \mathbb{R}^{N_l}$. The complete state of the system is given by the topological spinor $\vec{w} = (\vec{u}^\top, \vec{v}^\top)^\top \in \mathbb{R}^{N_n+N_l}$. The interactions between variables on nodes and links are mediated by the $N_n \times N_l$ boundary matrix $\boldsymbol{B}_1$ and its transpose, the coboundary matrix $\boldsymbol{B}_1^\top$. The dynamics of the system, which we refer to as a Dirac-Bianconi driven oscillator, are governed by a specific FitzHugh-Nagumo-like model given by the equations:
\begin{align*}
\dot{u}_i &= u_i - u_i^3 - (\boldsymbol{B}_1\vec{v})_i + I_i, \quad \text{for } i=1, \dots, N_n \\
\dot{v}_j &= \delta_j \left( (\boldsymbol{B}_1^\top\vec{u})_j - b_j v_j + \alpha_j a_j \right), \quad \text{for } j=1, \dots, N_l
\end{align*}
where $(\boldsymbol{B}_1\vec{v})_i = \sum_{k=1}^{N_l} (\boldsymbol{B}_1)_{ik} v_k$ and $(\boldsymbol{B}_1^\top\vec{u})_j = \sum_{k=1}^{N_n} (\boldsymbol{B}_1^\top)_{jk} u_k$. The parameters $I_i, \delta_j, b_j, a_j, \alpha_j$ are real constants. The entire system can be written in the compact form $\dot{\vec{w}} = \vec{F}(\vec{w})$.

Task:
Derive the Jacobian matrix $J(\vec{w}) = \frac{\partial \vec{F}}{\partial \vec{w}}$ for this system. Express your result as an $(N_n+N_l) \times (N_n+N_l)$ block matrix, with blocks corresponding to the partial derivatives with respect to $\vec{u}$ and $\vec{v}$. Define any necessary diagonal matrices in terms of the system parameters.

\subsubsection*{Ground Truth Solution}
\[ \boxed{J(\vec{w}) = 
\begin{pmatrix}
\text{diag}(1 - 3u_i^2)_{i=1}^{N_n} & -\boldsymbol{B}_1 \\
\boldsymbol{\Delta} \boldsymbol{B}_1^\top & -\boldsymbol{\Delta} \boldsymbol{b}
\end{pmatrix}
\quad \text{where} \quad
\boldsymbol{\Delta} = \text{diag}(\delta_j)_{j=1}^{N_l} \quad \text{and} \quad \boldsymbol{b} = \text{diag}(b_j)_{j=1}^{N_l}} \]

\subsubsection*{Model Outputs}
\subsubsection*{Model: o4-mini (Score: 1.0)}
\paragraph*{Model Solution:}
\[ J(\vec w)=
\begin{pmatrix}
\diag\bigl(1-3u_{i}^{2}\bigr)_{i=1}^{N_{n}} & -\,B_{1} \\[1ex]
\diag(\delta_{j})_{j=1}^{N_{l}}\;B_{1}^{\top} & -\,\diag(\delta_{j}b_{j})_{j=1}^{N_{l}}
\end{pmatrix}\,. \]

\paragraph*{Judge's Evaluation:}

The ground truth expression is:
\[
J(\vec{w}) = 
\begin{pmatrix}
\text{diag}(1 - 3u_i^2)_{i=1}^{N_n} & -\boldsymbol{B}_1 \\
\boldsymbol{\Delta} \boldsymbol{B}_1^\top & -\boldsymbol{\Delta} \boldsymbol{b}
\end{pmatrix}
\quad \text{where} \quad
\boldsymbol{\Delta} = \text{diag}(\delta_j)_{j=1}^{N_l} \quad \text{and} \quad \boldsymbol{b} = \text{diag}(b_j)_{j=1}^{N_l}
\]

The model-generated expression is:
\[
J(\vec w)=
\begin{pmatrix}
\diag\bigl(1-3u_{i}^{2}\bigr)_{i=1}^{N_{n}} & -\,B_{1} \\[1ex]
\diag(\delta_{j})_{j=1}^{N_{l}}\;B_{1}^{\top} & -\,\diag(\delta_{j}b_{j})_{j=1}^{N_{l}}
\end{pmatrix}\,.
\]

---

**Comparison:**

1. **Top-left block:**  
Both have \(\text{diag}(1 - 3u_i^2)_{i=1}^{N_n}\). This matches exactly.

2. **Top-right block:**  
Ground truth: \(-\boldsymbol{B}_1\)  
Model: \(-B_1\)  
The difference in boldface is just notation; this is acceptable.

3. **Bottom-left block:**  
Ground truth: \(\boldsymbol{\Delta} \boldsymbol{B}_1^\top = \text{diag}(\delta_j) B_1^\top\)  
Model: \(\diag(\delta_j) B_1^\top\)  
This matches exactly.

4. **Bottom-right block:**  
Ground truth: \(-\boldsymbol{\Delta} \boldsymbol{b} = -\text{diag}(\delta_j) \text{diag}(b_j)\)  
Model: \(-\diag(\delta_j b_j)\)  

Here is the key difference:  
- The ground truth is the product of two diagonal matrices \(\text{diag}(\delta_j) \text{diag}(b_j)\), which equals \(\text{diag}(\delta_j b_j)\) because the product of two diagonal matrices is a diagonal matrix with entries multiplied elementwise.  
- The model writes \(-\diag(\delta_j b_j)\) directly, which is mathematically equivalent.

---

**Conclusion:**

The model's expression is mathematically equivalent to the ground truth expression. The only difference is that the model combines the product of two diagonal matrices into a single diagonal matrix with elementwise products, which is correct.

Therefore, the model-generated expression is **completely correct**.

---

**Score: 1**

\newpage
\subsection*{Problem 2 (Paper: 2506.20163v1)}
\subsubsection*{Problem Statement}
Background:
Consider a system called a Dirac-Bianconi driven oscillator, defined on a simplicial complex consisting of $N_n$ nodes (0-simplices) and $N_l$ links (1-simplices). The state of the system is described by a topological spinor $\vec{w} = (\vec{u}^\top, \vec{v}^\top)^\top$, where $\vec{u} \in \mathbb{R}^{N_n}$ is a 0-cochain (variables on nodes) and $\vec{v} \in \mathbb{R}^{N_l}$ is a 1-cochain (variables on links). The dynamics are given by $\dot{\vec{w}} = \vec{F}(\vec{w})$. Assume this system possesses a stable limit cycle solution $\vec{w}_c(t)$ with frequency $\omega$. Using phase reduction theory, the state on the limit cycle can be described by a single phase variable $\vartheta(t)$ such that $\dot{\vartheta} = \omega$ and $\vec{w}_c(t) = \vec{w}_c(\vartheta(t))$. The response of the phase to a small perturbation $\epsilon \vec{P}$ is governed by the phase sensitivity function (PSF) $\vec{Z}(\vartheta) = \nabla_{\vec{w}} \vartheta |_{\vec{w}_c(\vartheta)}$, which can be partitioned according to the node and link variables as $\vec{Z}(\vartheta) = (\vec{Z}_u(\vartheta)^\top, \vec{Z}_v(\vartheta)^\top)^\top$.

Now, consider two such oscillators, indexed by $k=1,2$, with state vectors $\vec{w}_k = (\vec{u}_k^\top, \vec{v}_k^\top)^\top$ and natural frequencies $\omega_k$. They are weakly coupled with strength $\epsilon \ll 1$. The dynamics of oscillator 1 are given by $\dot{\vec{w}}_1 = \vec{F}_1(\vec{w}_1) + \epsilon \vec{P}_1(\vec{w}_1, \vec{w}_2)$. To first order in $\epsilon$, its phase dynamics are $\dot{\vartheta}_1 = \omega_1 + \epsilon \Gamma_1(\vartheta_1, \vartheta_2)$, where the phase coupling function is $\Gamma_1(\vartheta_1, \vartheta_2) = \vec{Z}_1(\vartheta_1) \cdot \vec{P}_1(\vec{w}_1(\vartheta_1), \vec{w}_2(\vartheta_2))$.

The coupling perturbation $\vec{P}_1$ has components acting on the node variables, $\vec{p}_{1,u}$, and link variables, $\vec{p}_{1,v}$. The coupling is a combination of a diffusive term and a Dirac-Bianconi term.
1. The diffusive coupling acts only between corresponding nodes: $\vec{p}_{1,u}^{\text{diff}} = C_u(\vec{u}_2 - \vec{u}_1)$, where $C_u$ is an $N_n \times N_n$ matrix specifying which nodes are coupled.
2. The Dirac-Bianconi coupling is analogous to the internal coupling structure. It couples the link variables of oscillator 2 to the node variables of oscillator 1, and the node variables of oscillator 2 to the link variables of oscillator 1. This is mediated by the $N_n \times N_l$ boundary operator $\boldsymbol{B}_1$. The specific forms are:
   - Perturbation on $\vec{u}_1$: $\vec{p}_{1,u}^{\text{DB}} = - \boldsymbol{B}_1 \vec{v}_2$
   - Perturbation on $\vec{v}_1$: $\vec{p}_{1,v}^{\text{DB}} = \boldsymbol{\Delta}_1 \boldsymbol{B}_1^\top \vec{u}_2$, where $\boldsymbol{\Delta}_1$ is a diagonal $N_l \times N_l$ matrix of internal system parameters $\delta_{1,j}$.
The total perturbation on oscillator 1 is $\vec{P}_1 = (\vec{p}_{1,u}^{\text{diff}} + \vec{p}_{1,u}^{\text{DB}}, \vec{p}_{1,v}^{\text{DB}})^\top$.

Task:
Derive the phase coupling function $\Gamma_1(\vartheta_1, \vartheta_2)$ for oscillator 1. Express your answer in terms of the partitioned PSFs $\vec{Z}_{u,1}(\vartheta_1)$ and $\vec{Z}_{v,1}(\vartheta_1)$, the limit cycle trajectories $\vec{u}_k(\vartheta_k)$ and $\vec{v}_k(\vartheta_k)$, the boundary operator $\boldsymbol{B}_1$, the coupling matrix $C_u$, and the parameter matrix $\boldsymbol{\Delta}_1$.

\subsubsection*{Ground Truth Solution}
\[ \boxed{\Gamma_1(\vartheta_1, \vartheta_2) = \vec{Z}_{u,1}(\vartheta_1)^\top \left[ C_u(\vec{u}_2(\vartheta_2) - \vec{u}_1(\vartheta_1)) - \boldsymbol{B}_1 \vec{v}_2(\vartheta_2) \right] + \vec{Z}_{v,1}(\vartheta_1)^\top \left( \boldsymbol{\Delta}_1 \boldsymbol{B}_1^\top \vec{u}_2(\vartheta_2) \right)} \]

\subsubsection*{Model Outputs}
\subsubsection*{Model: o4-mini (Score: 0.0)}
\paragraph*{Model Solution:}
\[ \Gamma_{1}(\vartheta_{1},\vartheta_{2})
\;=\;
Z_{u,1}(\vartheta_{1})^{T}\,C_{u}\,\u_{2}(\vartheta_{2})
\;-\;
Z_{u,1}(\vartheta_{1})^{T}\,B_{1}\,\v_{2}(\vartheta_{2})
\;+\;
Z_{v,1}(\vartheta_{1})^{T}\,\Delta_{1}\,B_{1}^{T}\,\u_{2}(\vartheta_{2})\,. \]

\paragraph*{Judge's Evaluation:}

First, let's rewrite both expressions clearly for comparison:

**Ground Truth:**
\[
\Gamma_1(\vartheta_1, \vartheta_2) = \vec{Z}_{u,1}(\vartheta_1)^\top \left[ C_u(\vec{u}_2(\vartheta_2) - \vec{u}_1(\vartheta_1)) - \boldsymbol{B}_1 \vec{v}_2(\vartheta_2) \right] + \vec{Z}_{v,1}(\vartheta_1)^\top \left( \boldsymbol{\Delta}_1 \boldsymbol{B}_1^\top \vec{u}_2(\vartheta_2) \right)
\]

**Model-Generated:**
\[
\Gamma_1(\vartheta_1, \vartheta_2) = Z_{u,1}(\vartheta_1)^{T} C_u u_2(\vartheta_2) - Z_{u,1}(\vartheta_1)^{T} B_1 v_2(\vartheta_2) + Z_{v,1}(\vartheta_1)^{T} \Delta_1 B_1^{T} u_2(\vartheta_2)
\]

---

**Comparison:**

- The model's expression has the terms:
  \[
  Z_{u,1}(\vartheta_1)^T C_u u_2(\vartheta_2) - Z_{u,1}(\vartheta_1)^T B_1 v_2(\vartheta_2) + Z_{v,1}(\vartheta_1)^T \Delta_1 B_1^T u_2(\vartheta_2)
  \]

- The ground truth has:
  \[
  \vec{Z}_{u,1}(\vartheta_1)^\top \left[ C_u(\vec{u}_2(\vartheta_2) - \vec{u}_1(\vartheta_1)) - \boldsymbol{B}_1 \vec{v}_2(\vartheta_2) \right] + \vec{Z}_{v,1}(\vartheta_1)^\top \left( \boldsymbol{\Delta}_1 \boldsymbol{B}_1^\top \vec{u}_2(\vartheta_2) \right)
  \]

- The model's expression **omits** the term involving \(\vec{u}_1(\vartheta_1)\) inside the first bracket. Specifically, the ground truth has:
  \[
  C_u(\vec{u}_2(\vartheta_2) - \vec{u}_1(\vartheta_1))
  \]
  but the model only has:
  \[
  C_u u_2(\vartheta_2)
  \]

- This is a significant omission because the subtraction of \(\vec{u}_1(\vartheta_1)\) is part of the expression inside the bracket and affects the value of the entire term.

- The other terms match exactly in structure and variables (ignoring vector notation and boldface, which is allowed).

---

**Conclusion:**

- The model's expression is missing the \(- C_u \vec{u}_1(\vartheta_1)\) term inside the first bracket.
- This is more than a minor error; it changes the expression significantly.
- Therefore, the model's expression is **not completely correct**.
- Since the error is more than one minor mistake (it's a missing entire term inside the bracket), partial credit is not warranted.

---

**Score:**

\[
\boxed{0}
\]

Score: 0

\newpage
\subsection*{Problem 3 (Paper: 2506.20572v1)}
\subsubsection*{Problem Statement}
Background:
Consider a network of channels represented by an undirected graph $\mathcal{G} = (\mathcal{V}, E)$, where $\mathcal{V}$ is the set of nodes and $E$ is the set of edges. An incompressible fluid flows through this network. For each edge $(i,j) \in E$ connecting nodes $i$ and $j$, we define its length $L_{ij}$, the fluid flux through it $Q_{ij}$, and its conductivity $D_{ij}$. The total power dissipated by the viscous flow in the network is given by $\mathcal{P} = \sum_{(i,j)\in E} \frac{Q_{ij}^2 L_{ij}}{D_{ij}}$. The channels are elastic, and the volume of a single channel $(i,j)$ is related to its conductivity by $V_{ij} = \beta L_{ij} \sqrt{D_{ij}}$, where $\beta$ is a constant. The total volume of fluid in the network, $V = \sum_{(i,j)\in E} V_{ij}$, is conserved. The conductivities of the channels adapt over time according to the differential equation:
$$
\frac{d}{dt}\sqrt{D_{ij}} = \alpha \frac{g(Q_{ij})}{\sum_{(k,m)\in E} L_{km}\, g(Q_{km})} - \sqrt{D_{ij}}
$$
where $\alpha$ is a constant related to the total volume, and $g(Q_{ij})$ is a function that describes the channel's elastic response to the local fluid flux. The functional form of $g(Q_{ij})$ is determined by the principle of energy minimization: at steady state ($d/dt = 0$), the network's configuration of conductivities $\{D_{ij}\}$ is such that it minimizes the total power dissipation $\mathcal{P}$ for a given, fixed set of fluxes $\{Q_{ij}\}$ and a constant total network volume $V$.

Task:
Using the principle of minimum power dissipation under the constraint of constant total volume, derive the functional form of $g(Q_{ij})$. Provide the simplest possible expression for $g(Q_{ij})$ in terms of $Q_{ij}$.

\subsubsection*{Ground Truth Solution}
\[ \boxed{g(Q_{ij}) = Q_{ij}^{2/3}} \]

\subsubsection*{Model Outputs}
\subsubsection*{Model: o4-mini (Score: 1.0)}
\paragraph*{Model Solution:}
\[ g(Q_{ij})=|Q_{ij}|^{\tfrac23} \]

\paragraph*{Judge's Evaluation:}

The ground truth expression is 
\[
g(Q_{ij}) = Q_{ij}^{2/3}
\]
while the model-generated expression is 
\[
g(Q_{ij}) = |Q_{ij}|^{\frac{2}{3}}.
\]

The difference is the absolute value inside the model-generated expression. The ground truth does not specify an absolute value, so the function is defined as the power of \(Q_{ij}\) itself, which can be negative or positive. The model-generated expression ensures the base is non-negative by taking the absolute value before raising to the power \(2/3\).

Since the exponent \(2/3\) is a fractional power with an even numerator, the expression \(Q_{ij}^{2/3}\) is generally defined as \(\left(Q_{ij}^2\right)^{1/3}\), which is always non-negative. Therefore, mathematically, \(Q_{ij}^{2/3} = |Q_{ij}|^{2/3}\) for real \(Q_{ij}\).

Hence, the model-generated expression is mathematically equivalent to the ground truth expression.

**Score: 1**

\newpage
\subsection*{Problem 4 (Paper: 2506.20572v1)}
\subsubsection*{Problem Statement}
Background:
Consider a network of channels, represented by a graph with edges $(i,j)$, through which an incompressible fluid of dynamic viscosity $\eta$ flows. The flow in each channel is governed by the Hagen-Poiseuille law. Each channel $(i,j)$ has a fixed length $L_{ij}$ and a variable radius $r_{ij}$. The conductivity of the channel is given by $D_{ij} = \pi r_{ij}^4 / (8\eta)$, and the fluid flux $Q_{ij}$ is related to the pressure difference $(p_i - p_j)$ by $Q_{ij} = D_{ij}(p_i - p_j)/L_{ij}$. The volume of a channel is $V_{ij} = \pi r_{ij}^2 L_{ij}$. The total volume of the network, $V = \sum_{(i,j)} V_{ij}$, is conserved.

The conductivities of the channels adapt over time according to the differential equation:
$$ \frac{d}{dt}\sqrt{D_{ij}} = \mathcal{K} \cdot g(Q_{ij}) - \sqrt{D_{ij}} $$
where $\mathcal{K}$ is a normalization factor that is treated as constant for a given state of the network, and $g(\cdot)$ is a function that models the response of the channel walls to the flow.

The functional form of $g(Q_{ij})$ is determined by a physical optimization principle: for a given, fixed distribution of fluxes $\{Q_{ij}\}$ throughout the network, the conductivities $\{D_{ij}\}$ arrange themselves to minimize the total power dissipated by viscosity, $\mathcal{P} = \sum_{(i,j)} \frac{Q_{ij}^2 L_{ij}}{D_{ij}}$, subject to the constraint that the total network volume $V$ is constant. The steady-state solution of the adaptation equation ($\frac{d}{dt}\sqrt{D_{ij}} = 0$) implies that $\sqrt{D_{ij}}$ is directly proportional to $g(Q_{ij})$.

Task:
Using the principle of minimum power dissipation under the constraint of constant total volume, determine the functional form of $g(Q_{ij})$. Express your result in terms of the flux $Q_{ij}$, ignoring any arbitrary proportionality constants.

\subsubsection*{Ground Truth Solution}
\[ \boxed{g(Q_{ij}) = Q_{ij}^{2/3}} \]

\subsubsection*{Model Outputs}
\subsubsection*{Model: o4-mini (Score: 0.5)}
\paragraph*{Model Solution:}
\[ g(Q_{ij}) = |Q_{ij}|^{\tfrac23}\,. \]

\paragraph*{Judge's Evaluation:}

The ground truth expression is 
\[
g(Q_{ij}) = Q_{ij}^{2/3}
\]
while the model-generated expression is 
\[
g(Q_{ij}) = |Q_{ij}|^{\tfrac{2}{3}}\,.
\]

The difference is the absolute value inside the power in the model-generated expression. The ground truth does not include the absolute value, so the model's expression is not exactly the same. However, the absolute value is a minor modification that ensures the expression is defined for negative values of \( Q_{ij} \), which might be physically or mathematically reasonable depending on context. 

Since the problem states the final expression must be mathematically equivalent and match exactly outside of variable names, the presence of the absolute value is a difference that cannot be ignored. However, it is only one minor difference (one additional absolute value symbol), so this fits the rubric for partial credit.

**Score: 0.5**

\newpage
\subsection*{Problem 5 (Paper: 2506.21226v1)}
\subsubsection*{Problem Statement}
Background:
Consider a network automaton model defined on a graph with $N$ nodes. Each node $v_i$ can be in one of two states, $s_i \in \{0,1\}$. The state of each node is updated synchronously at discrete time steps. The update rule $\phi$ for a node $v_i$ is uniform across the network and is of an outer-totalistic nature. It depends on the node's own state $s_i$ at time $t$ and the density of active neighbors $\rho_i^t = q/k_i$, where $k_i$ is the degree of node $v_i$ and $q = \sum_{j} A_{ij}s_j^t$ is the number of its neighbors in state 1. The new state is given by $s_i^{t+1} = \phi(s_i^t, \rho_i^t)$.

We analyze the propagation of a small perturbation through the network. Consider two state configurations of the network at time $t$, denoted by $\{s_i^t\}$ and $\{s_i'^t\}$. The defect at node $v_i$ is defined as $d_i^t = s_i^t \oplus s_i'^t$, where $\oplus$ is the XOR operation. The global state of the system is characterized by two quantities: the average state density of the unperturbed configuration, $\rho^t = \frac{1}{N}\sum_i s_i^t$, and the normalized Hamming distance (or defect density) between the two configurations, $\delta^t = \frac{1}{N}\sum_i d_i^t$.

We adopt a mean-field approximation, ignoring local topological correlations. Under this assumption, the state of any node and its neighbors can be described by statistical distributions dependent only on the global quantities $\rho^t$ and $\delta^t$. For a single node with degree $k$, we define the following quantities and their probabilities:
- $s \in \{0,1\}$: the state of the central node in the unperturbed configuration. Its probability is $P(s|\rho^t) = (\rho^t)^s (1-\rho^t)^{1-s}$.
- $q \in \{0, \dots, k\}$: the number of neighbors in state 1. Its probability is $P(q|k,\rho^t) = \binom{k}{q} (\rho^t)^q (1-\rho^t)^{k-q}$.
- $c \in \{0,1\}$: the defect state of the central node ($c=1$ if $s \neq s'$). Its probability is $P(c|\delta^t) = (\delta^t)^c (1-\delta^t)^{1-c}$.
- $d \in \{0, \dots, k\}$: the number of neighbors with a defect. Its probability is $P(d|k,\delta^t) = \binom{k}{d}(\delta^t)^d (1-\delta^t)^{k-d}$.
- $\tau \in \{\max(0, d+q-k), \dots, \min(d,q)\}$: the number of defective neighbors that flip from state 1 to 0 (a "killer toggle"). Its probability, conditioned on $k, d, q$, is given by the hypergeometric distribution $P(\tau|k, d, q) = \binom{q}{\tau}\binom{k-q}{d-\tau} / \binom{k}{d}$.

The defect at the central node at time $t+1$, denoted $\mathcal{D}$, depends on these variables. The state of the perturbed node is $s' = s \oplus c$. The number of its active neighbors in the perturbed configuration is $q' = q - \tau + (d-\tau) = q - 2\tau + d$. The defect is thus $\mathcal{D}(k, s, q, c, d, \tau) = \phi(s, q/k) \oplus \phi(s\oplus c, (q-2\tau+d)/k)$. The overall expected defect density at time $t+1$, $\langle\delta^{t+1}\rangle$, is found by averaging the single-node expected defect over the network's degree distribution, $P(k)$.

Task:
Derive the analytical expression for the expected normalized Hamming distance at time $t+1$, $\langle\delta^{t+1}\rangle$, as a function of the normalized Hamming distance $\delta^t$ and the average state density $\rho^t$ at time $t$. Your expression should be an average over the degree distribution $P(k)$ and should involve sums over all relevant probabilistic variables ($s, q, c, d, \tau$).

\subsubsection*{Ground Truth Solution}
\[ \boxed{\langle\delta^{t+1}\rangle = \sum_k P(k) \sum_{s \in \{0,1\}} P(s|\rho^t) \sum_{q=0}^k P(q|k,\rho^t) \sum_{c \in \{0,1\}}P(c|\delta^t) \sum_{d=0}^k P(d|k, \delta^t) \sum_{\tau} P(\tau|k,d,q) \, \mathcal{D}(k, s, q, c, d, \tau)} \]

\subsubsection*{Model Outputs}
\subsubsection*{Model: o4-mini (Score: 1.0)}
\paragraph*{Model Solution:}
\[ \langle\delta^{t+1}\rangle
\;=\;
\sum_{k=0}^\infty P(k)\,
\sum_{s=0}^1\sum_{q=0}^{k}\sum_{c=0}^1\sum_{d=0}^{k}
\sum_{\tau=\max(0,d+q-k)}^{\min(d,q)} 
\;P(s\mid\rho^t)\;P(q\mid k,\rho^t)\;P(c\mid\delta^t)\;P(d\mid k,\delta^t)\;P(\tau\mid k,d,q)\;
\Delta\bigl(k,s,q,c,d,\tau\bigr) \]

\paragraph*{Judge's Evaluation:}

The ground truth expression is:
\[
\langle \delta^{t+1} \rangle = \sum_k P(k) \sum_{s \in \{0,1\}} P(s|\rho^t) \sum_{q=0}^k P(q|k,\rho^t) \sum_{c \in \{0,1\}} P(c|\delta^t) \sum_{d=0}^k P(d|k,\delta^t) \sum_{\tau} P(\tau|k,d,q) \, \mathcal{D}(k, s, q, c, d, \tau)
\]

The model-generated expression is:
\[
\langle \delta^{t+1} \rangle = \sum_{k=0}^\infty P(k) \sum_{s=0}^1 \sum_{q=0}^k \sum_{c=0}^1 \sum_{d=0}^k \sum_{\tau=\max(0,d+q-k)}^{\min(d,q)} P(s|\rho^t) P(q|k,\rho^t) P(c|\delta^t) P(d|k,\delta^t) P(\tau|k,d,q) \Delta(k,s,q,c,d,\tau)
\]

**Comparison:**

1. **Summation indices and ranges:**
   - Ground truth sums over \(k\) without explicit limits; model sums from \(k=0\) to \(\infty\). This is consistent if \(k\) is a degree or count variable that is non-negative integer and unbounded.
   - Both sum over \(s \in \{0,1\}\) and \(c \in \{0,1\}\) correctly.
   - Both sum over \(q\) and \(d\) from 0 to \(k\).
   - The ground truth sums over \(\tau\) without explicit limits; the model specifies \(\tau\) from \(\max(0,d+q-k)\) to \(\min(d,q)\). This is likely the correct support for \(\tau\) given the problem context (e.g., intersection of two binomial variables). This is a refinement, not an error.

2. **Probability terms:**
   - Both have the same conditional probabilities \(P(s|\rho^t)\), \(P(q|k,\rho^t)\), \(P(c|\delta^t)\), \(P(d|k,\delta^t)\), and \(P(\tau|k,d,q)\).
   - The order of multiplication is the same, and all terms are present.

3. **Function inside the sum:**
   - Ground truth uses \(\mathcal{D}(k,s,q,c,d,\tau)\).
   - Model uses \(\Delta(k,s,q,c,d,\tau)\).
   - This is a difference in notation only (variable names can differ), so it is acceptable.

4. **Formatting and notation:**
   - The model expression is more explicit and detailed.
   - No missing terms or extra terms.
   - No sign errors or missing coefficients.

**Conclusion:**

The model-generated expression is mathematically equivalent to the ground truth expression, with more explicit summation limits for \(\tau\) and \(k\), and a different notation for the function \(\mathcal{D}\) vs. \(\Delta\), which is allowed.

**Score: 1**

---

Score: 1

\newpage
\subsection*{Problem 6 (Paper: 2506.21226v1)}
\subsubsection*{Problem Statement}
Background:
Consider a network automaton model where each node $v_i$ in a network of $N$ nodes can be in one of two states, $s_i \in \{0,1\}$. The state of all nodes is updated synchronously at discrete time steps. The update rule for any node $v_i$ is a uniform outer-totalistic function $\phi(s_i, \rho_i)$, which depends on the node's own state $s_i$ and the density of its active neighbors, $\rho_i = \frac{1}{k_i}\sum_{j}A_{ij}s_j$, where $k_i$ is the degree of node $v_i$ and $A_{ij}$ are the elements of the adjacency matrix.

We wish to study the propagation of a small perturbation through the network. Consider two configurations of the system at time $t$, $\{s_i(t)\}$ and $\{s'_i(t)\}$, which are nearly identical. The bitwise difference between these configurations is the defect configuration $\{d_i(t)\}$, where $d_i(t) = s_i(t) \oplus s'_i(t)$ and $\oplus$ is the XOR operator. The normalized Hamming distance between the two configurations is $\delta^t = \frac{1}{N}\sum_{i=1}^N d_i(t)$. We are interested in the expected normalized Hamming distance at the next time step, $\langle\delta^{t+1}\rangle$.

To derive an analytical expression for this quantity, we employ a mean-field approximation, assuming a large, random network where local topological correlations are negligible. The overall system is characterized by the global state average $\rho^t = \frac{1}{N}\sum_i s_i(t)$ and the normalized Hamming distance $\delta^t$. The probability of any given node being in state $1$ is $P(s=1|\rho^t) = \rho^t$, and the probability of it having a defect is $P(c=1|\delta^t) = \delta^t$, where $c$ is the defect state of the central node. For a node of degree $k$, the number of its neighbors in state 1, denoted by $q$, follows a binomial distribution $P(q|k, \rho^t)$. Similarly, the number of its neighbors with a defect, denoted by $d$, follows a binomial distribution $P(d|k, \delta^t)$. Among the $d$ defects in the neighborhood, let $\tau$ be the number of "killer toggles" (defects that change a neighbor's state from $1$ to $0$). The probability distribution for $\tau$, given $k, d, q$, is hypergeometric. The defect at a node $v_i$ at time $t+1$ is given by $d_i^{t+1} = \phi(s_i(t), \rho_i(t)) \oplus \phi(s'_i(t), \rho'_i(t))$.

Task:
Under the mean-field approximation, derive the full analytical expression for the expected normalized Hamming distance $\langle\delta^{t+1}\rangle$. Your final expression should be an average over the degree distribution $P(k)$ and should involve summations over all possible states and defect configurations of a representative node and its neighborhood ($s, q, c, d, \tau$). The probabilities for these configurations are given by $P(s|\rho^t) = (\rho^t)^s (1-\rho^t)^{1-s}$, $P(q|k,\rho^t) = \binom{k}{q}(\rho^t)^q (1-\rho^t)^{k-q}$, $P(c|\delta^t) = (\delta^t)^c (1-\delta^t)^{1-c}$, $P(d|k, \delta^t) = \binom{k}{d}(\delta^t)^d (1-\delta^t)^{k-d}$, and $P(\tau|k, d, q) = \frac{\binom{q}{\tau}\binom{k-q}{d-\tau}}{\binom{k}{d}}$.


\subsubsection*{Ground Truth Solution}
\[ \boxed{\langle\delta^{t+1}\rangle = \sum_k P(k)\!\! \sum_{s \in \{0,1\}}\!\!\! P(s|\rho^t)
                \sum_{q=0}^k P(q|k,\rho^t)\!\! \sum_{c \in \{0,1\}}P(c|\delta^t)
                \sum_{d=0}^k P(d|k, \delta^t)
                \sum_{\tau} P(\tau|k,d,q) \, \left( \phi\bigg(s, \frac{q}{k}\bigg) \oplus \phi\bigg(s\oplus c, \frac{q-2\tau+d}{k}\bigg) \right)} \]

\subsubsection*{Model Outputs}
\subsubsection*{Model: o4-mini (Score: 1.0)}
\paragraph*{Model Solution:}
\[ \begin{aligned}
\langle\delta^{t+1}\rangle
&=\sum_{k=0}^\infty P(k)\,
  \sum_{s=0}^1(\rho^t)^s(1-\rho^t)^{1-s}
  \sum_{c=0}^1(\delta^t)^c(1-\delta^t)^{1-c}\\
&\quad\times\sum_{q=0}^k\binom{k}{q}(\rho^t)^q(1-\rho^t)^{\,k-q}
  \sum_{d=0}^k\binom{k}{d}(\delta^t)^d(1-\delta^t)^{\,k-d}\\
&\quad\times\sum_{\tau=0}^{d}
   \frac{\binom{q}{\tau}\,\binom{k-q}{\,d-\tau}}{\binom{k}{d}}
   \;\Biggl[\,
     \phi\!\Bigl(s,\tfrac{q}{k}\Bigr)
     \;\oplus\;
     \phi\!\Bigl(s\oplus c,\tfrac{q+d-2\tau}{k}\Bigr)
   \Biggr]\!.
\end{aligned} \]

\paragraph*{Judge's Evaluation:}

The ground truth expression is:
\[
\langle\delta^{t+1}\rangle = \sum_k P(k) \sum_{s \in \{0,1\}} P(s|\rho^t)
\sum_{q=0}^k P(q|k,\rho^t) \sum_{c \in \{0,1\}} P(c|\delta^t)
\sum_{d=0}^k P(d|k, \delta^t)
\sum_{\tau} P(\tau|k,d,q) \left( \phi\left(s, \frac{q}{k}\right) \oplus \phi\left(s \oplus c, \frac{q-2\tau+d}{k}\right) \right)
\]

The model-generated expression is:
\[
\begin{aligned}
\langle\delta^{t+1}\rangle
&= \sum_{k=0}^\infty P(k)
\sum_{s=0}^1 (\rho^t)^s (1-\rho^t)^{1-s}
\sum_{c=0}^1 (\delta^t)^c (1-\delta^t)^{1-c} \\
&\quad \times \sum_{q=0}^k \binom{k}{q} (\rho^t)^q (1-\rho^t)^{k-q}
\sum_{d=0}^k \binom{k}{d} (\delta^t)^d (1-\delta^t)^{k-d} \\
&\quad \times \sum_{\tau=0}^d \frac{\binom{q}{\tau} \binom{k-q}{d-\tau}}{\binom{k}{d}}
\left[
\phi\left(s, \frac{q}{k}\right) \oplus \phi\left(s \oplus c, \frac{q + d - 2\tau}{k}\right)
\right].
\end{aligned}
\]

---

**Comparison and Justification:**

1. **Summation indices and ranges:**
   - Ground truth sums over $k$ (no explicit range), model sums $k=0$ to $\infty$. This is consistent since $k$ is a degree or count variable.
   - $s$ and $c$ sum over $\{0,1\}$ in both.
   - $q$ and $d$ sum from $0$ to $k$ in both.
   - $\tau$ sum is unspecified in ground truth, but model sums $\tau=0$ to $d$. This is consistent with the hypergeometric distribution support.

2. **Probability terms:**
   - $P(s|\rho^t)$ is Bernoulli with parameter $\rho^t$, model writes explicitly as $(\rho^t)^s (1-\rho^t)^{1-s}$, which matches.
   - $P(c|\delta^t)$ similarly Bernoulli with parameter $\delta^t$, model writes explicitly as $(\delta^t)^c (1-\delta^t)^{1-c}$, matches.
   - $P(q|k,\rho^t)$ is binomial, model writes $\binom{k}{q} (\rho^t)^q (1-\rho^t)^{k-q}$, matches.
   - $P(d|k,\delta^t)$ is binomial, model writes $\binom{k}{d} (\delta^t)^d (1-\delta^t)^{k-d}$, matches.
   - $P(\tau|k,d,q)$ is hypergeometric, model writes $\frac{\binom{q}{\tau} \binom{k-q}{d-\tau}}{\binom{k}{d}}$, matches.

3. **Argument inside $\phi$:**
   - First argument: $s$ and $q/k$ in both.
   - Second argument: $s \oplus c$ and $(q - 2\tau + d)/k$ in ground truth, model writes $(q + d - 2\tau)/k$.
   
   Note that $(q - 2\tau + d) = (q + d - 2\tau)$, so these are algebraically identical.

4. **Overall structure:**
   - The model expression explicitly expands the conditional probabilities into their binomial/hypergeometric forms, which is consistent with the ground truth.
   - The model expression includes all sums and terms present in the ground truth.
   - No missing terms or incorrect coefficients/signs are observed.

---

**Conclusion:**

The model-generated expression is mathematically equivalent to the ground truth expression, with only explicit expansions of the conditional probabilities. The arguments inside the functions and summations match exactly, and the summation limits are consistent.

**Score: 1**

---

Score: 1

\newpage
\subsection*{Problem 7 (Paper: 2506.21498v1)}
\subsubsection*{Problem Statement}
Background:
Consider a strategic interaction between two players, Player 1 and Player 2. Each player can choose one of two strategies, $\mathbf{C}$ or $\mathbf{D}$. The payoffs are symmetric and given by the matrix below, where the entries represent the payoff for Player 1 (the row player).

\begin{equation*}
\begin{array}{c|cc}
&\mathbf{\;C\;} &\mathbf{\;D\;}\\
\hline
\mathbf{\;C\;} &1 &-B \\
\mathbf{\;D\;} &1-A &0 \\
\end{array}
\end{equation*}

The players adapt their strategies over time through a learning process known as introspection dynamics. In each time step, one of the two players is chosen at random with equal probability to revise their strategy. The chosen player, say player $i$, compares their current payoff, $\pi$, with the payoff they would have received, $\pi'$, by unilaterally switching to the other strategy (while the co-player's strategy remains fixed). Player $i$ then switches to the alternative strategy with a probability given by the Fermi function:

\begin{equation*}
\phi_i = \frac{1}{1+{\rm e}^{-\beta_i\, (\pi' - \pi)}}
\end{equation*}

The parameter $\beta_i \ge 0$ is the selection strength of player $i$. This process defines a continuous-time Markov chain on the four possible strategy profiles: $\mathbf{CC}$, $\mathbf{CD}$, $\mathbf{DC}$, and $\mathbf{DD}$. The system eventually settles into a unique stationary distribution, $\mathbf{u} = (u_{\mathbf{CC}}, u_{\mathbf{CD}}, u_{\mathbf{DC}}, u_{\mathbf{DD}})$, which gives the long-term average frequencies of observing each profile. Consider the specific case where both players have the same selection strength, $\beta_1 = \beta_2 = \beta$.

Task:
Derive the stationary distribution vector $\mathbf{u}(A, B, \beta)$ for this symmetric case. Express your final answer as a single vector with a common normalization factor.

Solution:

\subsubsection*{Ground Truth Solution}
\[ \boxed{\frac{1}{2 + e^{A \beta} + e^{B \beta}}(e^{A \beta},1,1,e^{B \beta})} \]

\subsubsection*{Model Outputs}
\subsubsection*{Model: o4-mini (Score: 0.0)}
\paragraph*{Model Solution:}
\[ The introspection process is reversible and one easily checks by detailed‐balance on the four edges that the un–normalized weights of the four profiles are proportional to

 CC: 1  
 CD: e^{−βA}  
 DC: e^{−βA}  
 DD: e^{β(B−A)}. 

Hence the unique stationary distribution is

\[
\mathbf u(A,B,\beta)
=\frac{1}{\,1+2e^{-\beta A}+e^{\beta(B-A)}\,}
\begin{pmatrix}
1\\[1ex]
e^{-\beta A}\\[1ex]
e^{-\beta A}\\[1ex]
e^{\beta(B-A)}
\end{pmatrix}.
\]

In particular,

\[
u_{CC}=\frac{1}{1+2e^{-\beta A}+e^{\beta(B-A)}},\quad
u_{CD}=u_{DC}=\frac{e^{-\beta A}}{1+2e^{-\beta A}+e^{\beta(B-A)}},\quad
u_{DD}=\frac{e^{\beta(B-A)}}{1+2e^{-\beta A}+e^{\beta(B-A)}}.
\] \]

\paragraph*{Judge's Evaluation:}

Evaluation Error: No \boxed{} expression found in the model's output.

\newpage
\subsection*{Problem 8 (Paper: 2506.21498v1)}
\subsubsection*{Problem Statement}
Background:
Consider a symmetric $2 \times 2$ game between two players, where each can choose between two strategies, $\mathbf{C}$ and $\mathbf{D}$. The payoff for player 1 is given by the matrix:
\begin{equation*}
	\begin{array}{c|cc}
		&\mathbf{\;C\;} &\mathbf{\;D\;}\\
		\hline
		\mathbf{\;C\;} &1 &-B \\
		\mathbf{\;D\;} &1-A &0 \\
	\end{array}
\end{equation*}
Players adapt their strategies over time through a process of introspection. At each step, a player compares their current payoff $\pi$ with the payoff $\pi'$ they would have received by choosing the other strategy. The player switches to the alternative strategy with a probability given by the Fermi function $\phi = (1+e^{-\beta(\pi'-\pi)})^{-1}$, where $\beta \ge 0$ is the player's selection strength. This learning process can be modeled as a Markov chain on the four outcome states $\{\mathbf{CC}, \mathbf{CD}, \mathbf{DC}, \mathbf{DD}\}$. For a game between player 1 with selection strength $\beta_1$ and player 2 with $\beta_2$, this process leads to a unique stationary distribution $\mathbf{u}(\beta_1, \beta_2) = (u_{\mathbf{CC}}, u_{\mathbf{CD}}, u_{\mathbf{DC}}, u_{\mathbf{DD}})$. The components of this distribution are given by:
\begin{align*}
u_{\mathbf{CC}} &= N^{-1} (1+e^{-B\beta_1})(1+e^{-B\beta_2}) \\
u_{\mathbf{CD}} &= N^{-1} (1+e^{-B\beta_1})(1+e^{A\beta_2}) \\
u_{\mathbf{DC}} &= N^{-1} (1+e^{A\beta_1})(1+e^{-B\beta_2}) \\
u_{\mathbf{DD}} &= N^{-1} (1+e^{A\beta_1})(1+e^{A\beta_2})
\end{align*}
where $N$ is a normalization factor given by $N(\beta_1, \beta_2) = (1+e^{A\beta_1}+e^{-B\beta_1})(1+e^{A\beta_2}+e^{-B\beta_2}) + (e^{A\beta_1}-e^{-B\beta_1})(e^{A\beta_2}-e^{-B\beta_2})$. Player 1's average payoff is $\Pi_{\beta_2}(\beta_1) = 1 \cdot u_{\mathbf{CC}} -B\cdot u_{\mathbf{CD}} +(1-A)\cdot u_{\mathbf{DC}}$.

Now, consider the long-term evolution of the selection strength $\beta$ itself using the framework of adaptive dynamics. We assume a large, well-mixed population of resident individuals, all with selection strength $x$. A rare mutant with selection strength $y$ appears. The mutant's ability to invade is determined by its invasion fitness, $s_x(y) = \Pi_x(y) - \Pi_x(x)$. The direction of evolution is determined by the selection gradient, $D(x) = \left[ \frac{\partial s_x(y)}{\partial y } \right]_{y=x}$.

Task:
Derive the full analytical expression for the selection gradient $D(x)$. Your final expression should be simplified as much as possible.

\subsubsection*{Ground Truth Solution}
\[ \boxed{D(x) = \frac{e^{(A-B)x} \left[ (A+B)(1+e^{Ax})(1+e^{-Bx}) - (A-B)(1+e^{Ax}+e^{-Bx}) \right]}{\left( (1+e^{Ax}+e^{-Bx})^2 + (e^{Ax}-e^{-Bx})^2 \right)^2}} \]

\subsubsection*{Model Outputs}
\subsubsection*{Model: o4-mini (Score: 0.0)}
\paragraph*{Model Solution:}
\[ D(x)
=\frac{\displaystyle
\Bigl[
-\,B\,e^{-B x}\bigl(1+e^{-B x}\bigr)
\;+\;B^{2}\,e^{-B x}\bigl(1+e^{A x}\bigr)
\;+\;A(1-A)\,e^{A x}\bigl(1+e^{-B x}\bigr)
\Bigr]\,
\bigl[1+2e^{A x}+2e^{-B x}+2e^{2A x}+2e^{-2B x}\bigr]
-
\Bigl(1+e^{-B x}\Bigr)
\Bigl[2-A-B+e^{-B x}+(1-A-B)e^{A x}\Bigr]
\Bigl[A\,e^{A x}(2e^{A x}+1)-B\,e^{-B x}(2e^{-B x}+1)\Bigr]
}{\displaystyle
\bigl[\,1+2e^{A x}+2e^{-B x}+2e^{2A x}+2e^{-2B x}\bigr]^{2}
}\,. \]

\paragraph*{Judge's Evaluation:}

The ground truth expression is:

\[
D(x) = \frac{e^{(A-B)x} \left[ (A+B)(1+e^{Ax})(1+e^{-Bx}) - (A-B)(1+e^{Ax}+e^{-Bx}) \right]}{\left( (1+e^{Ax}+e^{-Bx})^2 + (e^{Ax}-e^{-Bx})^2 \right)^2}
\]

The model-generated expression is:

\[
D(x) = \frac{
\Bigl[
-\,B\,e^{-B x}\bigl(1+e^{-B x}\bigr)
\;+\;B^{2}\,e^{-B x}\bigl(1+e^{A x}\bigr)
\;+\;A(1-A)\,e^{A x}\bigr(1+e^{-B x}\bigr)
\Bigr]\,
\bigl[1+2e^{A x}+2e^{-B x}+2e^{2A x}+2e^{-2B x}\bigr]
-
\Bigl(1+e^{-B x}\Bigr)
\Bigl[2-A-B+e^{-B x}+(1-A-B)e^{A x}\Bigr]
\Bigl[A\,e^{A x}(2e^{A x}+1)-B\,e^{-B x}(2e^{-B x}+1)\Bigr]
}{
\bigl[\,1+2e^{A x}+2e^{-B x}+2e^{2A x}+2e^{-2B x}\bigr]^{2}
}
\]

---

**Comparison and Analysis:**

1. **Form and structure:**

   - The ground truth numerator is a relatively compact expression involving \( e^{(A-B)x} \) multiplied by a difference of two terms involving \((A+B)\) and \((A-B)\) and sums of exponentials.
   - The denominator is the square of a sum of squares: \(\left( (1+e^{Ax}+e^{-Bx})^2 + (e^{Ax}-e^{-Bx})^2 \right)^2\).

2. **Model denominator:**

   - The model denominator is \(\left[1 + 2e^{Ax} + 2e^{-Bx} + 2e^{2Ax} + 2e^{-2Bx}\right]^2\).
   - Let's check if this matches the ground truth denominator:

     \[
     (1 + e^{Ax} + e^{-Bx})^2 + (e^{Ax} - e^{-Bx})^2 = ?
     \]

     Expanding:

     \[
     (1 + e^{Ax} + e^{-Bx})^2 = 1 + e^{2Ax} + e^{-2Bx} + 2e^{Ax} + 2e^{-Bx} + 2e^{Ax} e^{-Bx}
     \]

     \[
     (e^{Ax} - e^{-Bx})^2 = e^{2Ax} + e^{-2Bx} - 2 e^{Ax} e^{-Bx}
     \]

     Adding:

     \[
     = 1 + e^{2Ax} + e^{-2Bx} + 2e^{Ax} + 2e^{-Bx} + 2e^{Ax} e^{-Bx} + e^{2Ax} + e^{-2Bx} - 2 e^{Ax} e^{-Bx}
     \]

     The \(2 e^{Ax} e^{-Bx}\) and \(-2 e^{Ax} e^{-Bx}\) cancel out, so:

     \[
     = 1 + 2 e^{2Ax} + 2 e^{-2Bx} + 2 e^{Ax} + 2 e^{-Bx}
     \]

     Rearranged:

     \[
     = 1 + 2 e^{Ax} + 2 e^{-Bx} + 2 e^{2Ax} + 2 e^{-2Bx}
     \]

     This matches exactly the model denominator inside the square brackets.

   - So the denominator is correct.

3. **Model numerator:**

   - The model numerator is a complicated expression involving multiple terms with coefficients and exponentials.
   - The ground truth numerator is much simpler and involves \( e^{(A-B)x} \) times a bracketed expression.
   - The model numerator does not explicitly factor out \( e^{(A-B)x} \), but that could be hidden in the terms.
   - However, the model numerator contains terms like \( B^2 e^{-Bx} (1 + e^{Ax}) \) and \( A(1 - A) e^{Ax} (1 + e^{-Bx}) \), which do not appear in the ground truth.
   - The ground truth numerator is linear in \( A \) and \( B \), while the model numerator contains quadratic terms like \( B^2 \) and \( A(1 - A) \).
   - The model numerator also contains complicated products of sums of exponentials, which do not simplify obviously to the ground truth numerator.
   - There is no clear equivalence or simplification that would reduce the model numerator to the ground truth numerator.
   - The model numerator appears to be a different expression, not a rearrangement or equivalent form of the ground truth numerator.

4. **Conclusion:**

   - The denominator is correct and matches the ground truth exactly.
   - The numerator is significantly different, with extra terms and different structure.
   - The model expression is not mathematically equivalent to the ground truth.
   - The difference is more than a minor error; it involves extra powers and terms.
   - Therefore, the model expression is incorrect.

---

**Score: 0**

---

**Final evaluation:**

\[
\boxed{
\text{Score: } 0
}
\]

\newpage
\subsection*{Problem 9 (Paper: 2506.22120v1)}
\subsubsection*{Problem Statement}
Background:
Consider a system of settlements modeled as a directed network $G=(V,E)$, where $V$ is the set of $|V|$ settlements (nodes) and $E$ is the set of connections (edges). Displacements in this system are driven by a set of distinct hazards, $\mathcal{H}$. For each settlement $v \in V$ and each hazard $h \in \mathcal{H}$, the quantity $D_h(v)$ represents the number of individuals displaced from $v$ due to that specific hazard. The \emph{dominant hazard} at a settlement $v$, denoted $h^*(v)$, is defined as the hazard type responsible for the largest number of displacements from that settlement, i.e., $h^*(v) = \arg\max_{h \in \mathcal{H}} D_h(v)$.

The movement of displaced populations is modeled via a large number of simulated stochastic paths. For each ordered pair of source-target settlements $(s, t) \in V \times V$, a set of $M$ distinct paths is generated. The $j$-th path for the pair $(s, t)$, for $j \in \{1, \dots, M\}$, is represented by a sequence of settlements $\mathcal{P}^{(j)}_{s \to t} = (v_1^{(s,t,j)}, v_2^{(s,t,j)}, \dots, v_{m_j^{(s,t)}}^{(s,t,j)})$, where $v_1^{(s,t,j)} = s$ is the source and $m_j^{(s,t)}$ is the total number of steps (length) of this specific path. The settlement at step $i$ of this path is $v_i^{(s,t,j)}$.

We aim to characterize the overall likelihood of encountering a specific dominant hazard at a given step along these paths. Let's define the necessary components. The indicator function $\mathbb{I}[\cdot]$ is $1$ if its argument is true, and $0$ otherwise. First, for a fixed source settlement $s$, we define the conditional probability $P_i(h \mid s)$ as the empirical probability that the dominant hazard at step $i$ is $h$, given that the path started at $s$. This is found by considering all simulated paths originating from $s$ (to all possible targets $t$). The probability is the ratio of the number of paths from $s$ that both reach step $i$ and have $h$ as the dominant hazard at $v_i^{(s,t,j)}$, to the total number of paths from $s$ that reach at least step $i$.

Task:
Using the definitions provided, construct the formal mathematical expression for the overall hazard likelihood at step $i$, $\bar{P}_i(h)$. This quantity is defined as the arithmetic mean of the conditional probabilities $P_i(h \mid s)$ taken over all possible source settlements $s \in V$. Your final expression should be in terms of the quantities defined in the background, such as $|V|$, $M$, $m_j^{(s,t)}$, $v_i^{(s,t,j)}$, $h^*$, and the indicator function $\mathbb{I}[\cdot]$.

\subsubsection*{Ground Truth Solution}
\[ \boxed{\bar{P}_i(h) = \frac{1}{|V|} \sum_{s \in V} \left( \frac{\sum_{t \in V} \sum_{j=1}^{M} \mathbb{I}[i \leq m_j^{(s,t)}] \cdot \mathbb{I}[h^*(v_i^{(s,t,j)}) = h]}{\sum_{t \in V} \sum_{j=1}^{M} \mathbb{I}[i \leq m_j^{(s,t)}]} \right)} \]

\subsubsection*{Model Outputs}
\subsubsection*{Model: o4-mini (Score: 1.0)}
\paragraph*{Model Solution:}
\[ \bar P_i(h)
=
\frac{1}{|V|}\sum_{s\in V}
\frac{
  \sum_{t\in V}\sum_{j=1}^{M}
    \mathbb{I}[\,m_j^{(s,t)}\ge i\,]\,
    \mathbb{I}[\,h^*(v_i^{(s,t,j)})=h\,]
}{
  \sum_{t\in V}\sum_{j=1}^{M}
    \mathbb{I}[\,m_j^{(s,t)}\ge i\,]
} \]

\paragraph*{Judge's Evaluation:}

The ground truth expression is:

\[
\bar{P}_i(h) = \frac{1}{|V|} \sum_{s \in V} \left( \frac{\sum_{t \in V} \sum_{j=1}^{M} \mathbb{I}[i \leq m_j^{(s,t)}] \cdot \mathbb{I}[h^*(v_i^{(s,t,j)}) = h]}{\sum_{t \in V} \sum_{j=1}^{M} \mathbb{I}[i \leq m_j^{(s,t)}]} \right)
\]

The model-generated expression is:

\[
\bar{P}_i(h) = \frac{1}{|V|} \sum_{s \in V} \frac{\sum_{t \in V} \sum_{j=1}^M \mathbb{I}[m_j^{(s,t)} \geq i] \cdot \mathbb{I}[h^*(v_i^{(s,t,j)}) = h]}{\sum_{t \in V} \sum_{j=1}^M \mathbb{I}[m_j^{(s,t)} \geq i]}
\]

Comparison:

- The indicator function condition in the ground truth is $\mathbb{I}[i \leq m_j^{(s,t)}]$.
- The model uses $\mathbb{I}[m_j^{(s,t)} \geq i]$.

These two are logically equivalent since $i \leq m_j^{(s,t)}$ is the same as $m_j^{(s,t)} \geq i$.

- The rest of the expression, including the sums, indices, and the indicator $\mathbb{I}[h^*(v_i^{(s,t,j)}) = h]$, matches exactly.
- The normalization factor $\frac{1}{|V|}$ and the summation over $s \in V$ are identical.

Therefore, the model-generated expression is mathematically equivalent to the ground truth expression, differing only in the order of the inequality inside the indicator function, which is acceptable.

**Score: 1**

\newpage
\subsection*{Problem 10 (Paper: 2506.22120v1)}
\subsubsection*{Problem Statement}
Background:
Consider a system of settlements modeled as a directed network $G=(V,E)$, where $V$ is the set of settlements (nodes) and $E$ is the set of displacement flows (edges). For each settlement $v \in V$, displacement can be triggered by a set of distinct hazards $\mathcal{H}$. The number of individuals displaced from settlement $v$ due to a specific hazard $h \in \mathcal{H}$ is given by the count $D_h(v)$. For any settlement $v$, the single most influential hazard, termed the "dominant hazard" $h^*(v)$, is defined as the one corresponding to the maximum displacement count, i.e., $h^*(v) = \arg\max_{h \in \mathcal{H}} D_h(v)$.

To analyze potential displacement trajectories, a simulation is performed. For each ordered pair of source-target settlements $(s, t)$, a set of $M$ probabilistic paths are generated. A single simulated path, indexed by $j \in \{1, \dots, M\}$, is a sequence of settlements $\mathcal{P}^{(j)}_{s \to t} = (v^{(j)}_1, v^{(j)}_2, \dots, v^{(j)}_{m_j})$, where the first settlement is the source, $v^{(j)}_1 = s$, and the path has a total length of $m_j$ steps.

The analysis aims to find the likelihood of encountering a specific dominant hazard at a given step along these trajectories. Let $P_i(h \mid s)$ be the empirical probability of encountering the dominant hazard $h$ at step $i$ of a path, given that the path originates from source $s$. This probability is computed by considering all simulated paths starting from $s$ to all possible targets $t \in V$. The normalization constant for this probability is $M_s^{(i)}$, defined as the total number of simulated paths originating from $s$ that have a length of at least $i$ steps. You may use the indicator function $\mathbb{I}[\cdot]$, which evaluates to $1$ if its argument is true and $0$ otherwise.

Task:
Derive the expression for the overall hazard likelihood at step $i$, $\bar{P}_i(h)$. This quantity is defined as the average of the per-source likelihoods $P_i(h \mid s)$ over all possible source settlements $s \in V$. The final expression should be written in terms of the fundamental quantities defined above, such as $|V|$, $M$, $M_s^{(i)}$, $h^*$, and the indicator function $\mathbb{I}[\cdot]$.

\subsubsection*{Ground Truth Solution}
\[ \boxed{\bar{P}_i(h) = \frac{1}{|V|} \sum_{s \in V} \left( \frac{1}{M_s^{(i)}} \sum_{t \in V} \sum_{j=1}^{M} \mathbb{I}[i \leq m_j] \cdot \mathbb{I}[h^*(v^{(j)}_i) = h] \right)} \]

\subsubsection*{Model Outputs}
\subsubsection*{Model: o4-mini (Score: 1.0)}
\paragraph*{Model Solution:}
\[ \bar P_i(h)
=
\frac1{|V|}\sum_{s\in V}
\frac{
  \sum_{t\in V}\sum_{j=1}^M
    \mathbb{I}\bigl(\lvert\mathcal P^{(j)}_{s\to t}\rvert\ge i\bigr)\,
    \mathbb{I}\bigl(h^*(v^{(j)}_{i})=h\bigr)
}{M_s^{(i)}}\,. \]

\paragraph*{Judge's Evaluation:}

The ground truth expression is:
\[
\bar{P}_i(h) = \frac{1}{|V|} \sum_{s \in V} \left( \frac{1}{M_s^{(i)}} \sum_{t \in V} \sum_{j=1}^{M} \mathbb{I}[i \leq m_j] \cdot \mathbb{I}[h^*(v^{(j)}_i) = h] \right)
\]

The model-generated expression is:
\[
\bar{P}_i(h) = \frac{1}{|V|} \sum_{s \in V} \frac{
  \sum_{t \in V} \sum_{j=1}^M \mathbb{I}\bigl(|\mathcal{P}^{(j)}_{s \to t}| \ge i\bigr) \cdot \mathbb{I}\bigl(h^*(v^{(j)}_i) = h\bigr)
}{M_s^{(i)}}
\]

**Comparison:**

- The outer structure is the same: an average over $s \in V$ with normalization by $|V|$.
- Inside the sum over $s$, both expressions have a normalization by $M_s^{(i)}$.
- The summations over $t \in V$ and $j=1$ to $M$ are present in both.
- The indicator functions differ in notation but appear to represent the same condition:
  - Ground truth uses $\mathbb{I}[i \leq m_j]$.
  - Model uses $\mathbb{I}[|\mathcal{P}^{(j)}_{s \to t}| \ge i]$.
  
  Since $m_j$ presumably denotes the length of path $j$, and $|\mathcal{P}^{(j)}_{s \to t}|$ is the length of path $j$ from $s$ to $t$, these are equivalent conditions.
  
- The second indicator $\mathbb{I}[h^*(v^{(j)}_i) = h]$ is identical in both.

- Variable names differ but are consistent and acceptable.

**Conclusion:**

The model-generated expression is mathematically equivalent to the ground truth expression, with only a difference in notation for the indicator condition, which is consistent in meaning.

**Score: 1**

Score: 1

\newpage
\subsection*{Problem 11 (Paper: 2506.22160v1)}
\subsubsection*{Problem Statement}
Background:
Consider a system of $N$ globally coupled Kuramoto phase oscillators, where the dynamics of the $j$-th oscillator is given by
$$ \frac{d\theta_j}{dt} = \omega_j + \frac{K}{N}\sum_{k=1}^N \sin(\theta_k - \theta_j). $$
Here, $\theta_j$ is the phase, $\omega_j$ is the natural frequency, and $K$ is the uniform coupling strength. The collective behavior of the system can be described by the complex order parameter $s(t) = \frac{1}{N} \sum_{j=1}^N e^{i \theta_j}$, which allows the dynamics to be rewritten as $\frac{d\theta_j}{dt} = \omega_j + K \operatorname{Im}(s e^{-i\theta_j})$. The natural frequencies $\omega_j$ are randomly drawn from a stationary Lorentzian distribution with zero mean and unit half-width, given by the probability density function $g(\omega) = \frac{1}{\pi}\frac{1}{\omega^2+1}$.

For a finite number of oscillators $N$, the system exhibits fluctuations even for subcritical coupling ($K<2$), where an infinite system would be asynchronous ($s=0$). These fluctuations can be treated as a form of "shot noise". In the absence of coupling ($K=0$), the system's order parameter is a "free shot noise" $\chi_0(t)$, whose power spectrum is given by $W_0(\omega) = \frac{2\pi}{N}g(\omega)$.

For non-zero subcritical coupling, the fluctuations of the finite system provide feedback, acting as a driving force on the population. This feedback can be analyzed using a self-consistent approach. We model the response of the finite population by considering an auxiliary infinite-size population driven by the order parameter $s(t)$ of the finite system. The order parameter of this auxiliary infinite population, denoted by $r(t)$, evolves according to the mean-field equation derived from the Ott-Antonsen ansatz:
$$ \frac{dr}{dt} = -r + \frac{K}{2}s - \frac{K}{2}s^* r^2. $$
The total order parameter of the finite system $s(t)$ is then approximated as the sum of the free shot noise $\chi_0(t)$ and the response $r(t)$ of the auxiliary system: $s(t) = \chi_0(t) + r(t)$. This establishes a self-consistency condition, as the response $r(t)$ is driven by the total signal $s(t)$ which itself contains $r(t)$.

Task:
Assuming the fluctuations are small (allowing for linearization) and using the self-consistency condition, determine the power spectrum $W(\omega)$ of the collective oscillations $s(t)$ in the finite-size system with subcritical coupling $K$. Express your result in terms of $N$, $K$, $\omega$, and the distribution $g(\omega)$.

\subsubsection*{Ground Truth Solution}
\[ \boxed{W(\omega)=\frac{2\pi}{N}\bigg|\frac{1+i\omega}{1+i\omega-K/2}\bigg|^2 g(\omega)} \]

\subsubsection*{Model Outputs}
\subsubsection*{Model: o4-mini (Score: 0.5)}
\paragraph*{Model Solution:}
\[ %
W(\omega)
=\frac{2\pi}{N}\;g(\omega)\;\frac{1+\omega^2}{\bigl(1-\tfrac K2\bigr)^2+\omega^2}
\;=\;
\frac{2}{N\,\bigl[(1-\tfrac K2)^2+\omega^2\bigr]}\,. \]

\paragraph*{Judge's Evaluation:}

First, let's rewrite the ground truth expression clearly:

\[
W(\omega) = \frac{2\pi}{N} \left| \frac{1 + i\omega}{1 + i\omega - \frac{K}{2}} \right|^2 g(\omega)
\]

The model-generated expression is:

\[
W(\omega) = \frac{2\pi}{N} g(\omega) \frac{1 + \omega^2}{\left(1 - \frac{K}{2}\right)^2 + \omega^2} = \frac{2}{N \left[\left(1 - \frac{K}{2}\right)^2 + \omega^2\right]}
\]

---

**Step 1: Check equivalence of the first part**

The magnitude squared of a complex fraction:

\[
\left| \frac{1 + i\omega}{1 + i\omega - \frac{K}{2}} \right|^2 = \frac{|1 + i\omega|^2}{|1 + i\omega - \frac{K}{2}|^2}
\]

Calculate numerator magnitude squared:

\[
|1 + i\omega|^2 = 1^2 + \omega^2 = 1 + \omega^2
\]

Calculate denominator magnitude squared:

\[
|1 + i\omega - \frac{K}{2}|^2 = \left(1 - \frac{K}{2}\right)^2 + \omega^2
\]

Therefore,

\[
\left| \frac{1 + i\omega}{1 + i\omega - \frac{K}{2}} \right|^2 = \frac{1 + \omega^2}{\left(1 - \frac{K}{2}\right)^2 + \omega^2}
\]

This matches exactly the fraction inside the model-generated expression.

---

**Step 2: Check the factor outside the fraction**

The ground truth has:

\[
\frac{2\pi}{N} g(\omega)
\]

The model-generated expression has:

\[
\frac{2\pi}{N} g(\omega) \frac{1 + \omega^2}{\left(1 - \frac{K}{2}\right)^2 + \omega^2}
\]

which matches the ground truth.

---

**Step 3: Check the last equality in the model expression**

The model writes:

\[
W(\omega) = \frac{2}{N \left[\left(1 - \frac{K}{2}\right)^2 + \omega^2\right]}
\]

This is inconsistent with the previous expression because:

- The factor $2\pi$ is missing.
- The factor $g(\omega)$ is missing.
- The numerator $1 + \omega^2$ is missing.

Therefore, the last equality is incorrect.

---

**Summary:**

- The first expression in the model matches the ground truth exactly.
- The last equality in the model is incorrect and inconsistent with the first expression and the ground truth.
- Since the model expression contains one fully correct expression and one incorrect simplification, the overall expression is partially correct but not fully.

---

**Score justification:**

Because the model correctly reproduces the main expression but then provides an incorrect simplification, this counts as one minor error (missing factors and terms in the last equality). According to the rubric, this warrants a partial credit of 0.5.

---

**Final score:**

Score: 0.5

\newpage
\subsection*{Problem 12 (Paper: 2506.22160v1)}
\subsubsection*{Problem Statement}
Background:
Consider a system of $N$ globally coupled Kuramoto oscillators, where the phase $\theta_j$ of each oscillator evolves according to the equation:
$$ \frac{d\theta_j}{dt} = \omega_j + K \operatorname{Im} (s e^{-i\theta_j}) $$
Here, $s = \frac{1}{N} \sum_{k=1}^N e^{i\theta_k}$ is the complex order parameter of the finite-size system, $K$ is the uniform coupling strength, and $\omega_j$ are the natural frequencies. The frequencies are independent random variables drawn from a Lorentzian distribution with zero mean and unit half-width, given by the probability density function $g(\omega) = \frac{1}{\pi(1+\omega^2)}$.

In the subcritical regime ($0 < K < 2$), the mean-field theory for an infinite system ($N \to \infty$) predicts a stationary asynchronous state with zero order parameter. For a finite system, however, the order parameter $s(t)$ exhibits persistent fluctuations due to the finite number of oscillators, a phenomenon known as shot noise. The total signal $s(t)$ can be conceptually decomposed into two parts: $s(t) = \chi_0(t) + r(t)$, where $\chi_0(t)$ is the "free" shot noise that would exist for uncoupled oscillators ($K=0$), and $r(t)$ is the induced collective response that arises from the coupling. The power spectrum of the free shot noise is given by $W_0(\omega) = \frac{2\pi}{N} g(\omega)$.

The induced response $r(t)$ can be modeled as the order parameter of an auxiliary infinite-size population driven by the total signal $s(t)$. For the given Lorentzian frequency distribution, the dynamics of this induced response can be described by the following equation, derived from the Ott-Antonsen ansatz:
$$ \frac{dr}{dt} = -r + \frac{K}{2}s - \frac{K}{2}s^*r^2 $$
For the purpose of analyzing small fluctuations, this equation can be linearized around the state $r=0$. The relationship between the Fourier transforms of the signals, $s_\omega = \mathcal{F}[s(t)]$, $w_{0,\omega} = \mathcal{F}[\chi_0(t)]$, and $r_\omega = \mathcal{F}[r(t)]$, is then given by the self-consistency condition $s_\omega = w_{0,\omega} + r_\omega$.

Task:
Using this theoretical framework, derive an expression for the total variance of the order parameter, $D = \langle |s(t)|^2 \rangle$, for the finite-size system with subcritical coupling ($0 < K < 2$). The variance is related to the two-sided power spectral density $W(\omega)$ of the signal $s(t)$ by Parseval's theorem, $D = \frac{1}{2\pi} \int_{-\infty}^{\infty} W(\omega) d\omega$.

\subsubsection*{Ground Truth Solution}
\[ \boxed{D=\frac{1}{N}\frac{1}{1-K/2}} \]

\subsubsection*{Model Outputs}
\subsubsection*{Model: o4-mini (Score: 1.0)}
\paragraph*{Model Solution:}
\[ D=\frac{2}{N\,(2-K)}\,,\quad 0<K<2\,. \]

\paragraph*{Judge's Evaluation:}

First, let's rewrite both expressions clearly:

**Ground Truth:**
\[
D = \frac{1}{N} \cdot \frac{1}{1 - \frac{K}{2}} = \frac{1}{N(1 - \frac{K}{2})} = \frac{1}{N \left(1 - \frac{K}{2}\right)}
\]

**Model-Generated:**
\[
D = \frac{2}{N (2 - K)} \quad \text{with } 0 < K < 2
\]

Now, let's check if these two expressions are equivalent.

Rewrite the denominator of the ground truth:
\[
1 - \frac{K}{2} = \frac{2 - K}{2}
\]

So,
\[
D = \frac{1}{N} \cdot \frac{1}{\frac{2 - K}{2}} = \frac{1}{N} \cdot \frac{2}{2 - K} = \frac{2}{N (2 - K)}
\]

This matches exactly the model-generated expression.

Therefore, the model-generated expression is mathematically equivalent to the ground truth expression.

The model also includes the domain condition \(0 < K < 2\), which is consistent with the denominator not being zero.

**Score justification:**

- The model expression is exactly equivalent to the ground truth.
- The difference is only in the algebraic manipulation of the denominator.
- The domain condition is an additional correct detail.

**Score: 1**

---

Score: 1

\newpage
\subsection*{Problem 13 (Paper: 2506.22695v1)}
\subsubsection*{Problem Statement}
Background:
In a non-equilibrium statistical mechanics model of protein dynamics, the state of a single protein is described by its position $\mb{X} \in \mathbb{R}^d$ and its internal temperature $\theta_P \in \mathbb{R}$. The irreversible dynamics, which encompass both deterministic drift and dissipation, are governed by a symmetric, positive-semidefinite operator $K^{(1)}$. This operator couples the mechanical and thermal degrees of freedom and is given by the $(d+1) \times (d+1)$ block matrix:
\begin{equation*}
K^{(1)} = 
\begin{pmatrix}
\theta_P \mb{M}_{\tiny XX} & -\frac{\theta_P \mb{M}_{\tiny XX}(\nabla_{\mb{X}} \mathcal{E})}{c_P} \\
-\frac{(\nabla_{\mb{X}} \mathcal{E})^{T}\theta_P \mb{M}_{\tiny XX}}{c_P} & \frac{(\nabla_{\mb{X}} \mathcal{E})^{T}\theta_P \mb{M}_{\tiny XX}(\nabla_{\mb{X}} \mathcal{E})}{c_P^2}
\end{pmatrix}
\end{equation*}
Here, $\mb{M}_{\tiny XX}$ is the symmetric and positive-definite $d \times d$ mobility tensor of the protein, $c_P$ is the protein's heat capacity, and $\nabla_{\mb{X}} \mathcal{E}$ is the conservative force (a $d \times 1$ column vector) derived from the system's total energy $\mathcal{E}$. To generate the corresponding thermal fluctuations in simulations, one must compute a matrix factor $R^{(1)}$ such that $K^{(1)} = R^{(1)}R^{(1),T}$. This factor $R^{(1)}$ is a $(d+1) \times d$ matrix.

Task:
Given a matrix factorization of the mobility tensor $\mb{M}_{\tiny XX} = R_M R_M^T$, where $R_M$ is a $d \times d$ matrix, derive the expression for the factor $R^{(1)}$.

\subsubsection*{Ground Truth Solution}
\[ \boxed{R^{(1)} = \left[
\begin{array}{c}
\sqrt{\theta_P}R_M \\
-\frac{\sqrt{\theta_P}(\nabla_{\mb{X}} \mathcal{E})^T R_M}{c_P}
\end{array}
\right]} \]

\subsubsection*{Model Outputs}
\subsubsection*{Model: o4-mini (Score: 1.0)}
\paragraph*{Model Solution:}
\[ R^{(1)}=
\begin{pmatrix}
\sqrt{\theta_P}\,R_M \\[1ex]
-\dfrac{\sqrt{\theta_P}}{c_P}\,(\nabla_{\mathbf X}\mathcal E)^{T}\,R_M
\end{pmatrix}\,. \]

\paragraph*{Judge's Evaluation:}

The ground truth expression is:
\[
R^{(1)} = \left[
\begin{array}{c}
\sqrt{\theta_P} R_M \\
-\frac{\sqrt{\theta_P} (\nabla_{\mathbf{X}} \mathcal{E})^T R_M}{c_P}
\end{array}
\right]
\]

The model-generated expression is:
\[
R^{(1)} =
\begin{pmatrix}
\sqrt{\theta_P} R_M \\[1ex]
-\dfrac{\sqrt{\theta_P}}{c_P} (\nabla_{\mathbf{X}} \mathcal{E})^{T} R_M
\end{pmatrix}
\]

Comparison:

- Both expressions use the same variables and structure.
- The vector notation uses \(\left[\begin{array}{c} \cdot \\ \cdot \end{array}\right]\) in the ground truth and \(\begin{pmatrix} \cdot \\ \cdot \end{pmatrix}\) in the model expression, which are equivalent notations for column vectors.
- The order of multiplication in the second component is the same: \((\nabla_{\mathbf{X}} \mathcal{E})^T R_M\).
- The factor \(\sqrt{\theta_P}\) is correctly placed and the division by \(c_P\) is correct.
- Variable names match exactly.
- The model expression includes a small extra spacing command \([1ex]\) which does not affect the mathematical correctness.

Therefore, the model-generated expression is mathematically equivalent to the ground truth expression.

**Score: 1**

\newpage
\subsection*{Problem 14 (Paper: 2506.22695v1)}
\subsubsection*{Problem Statement}
Background:
In the non-equilibrium statistical mechanics of a membrane-protein system, thermal exchanges between different components give rise to irreversible dynamics and fluctuations. Consider the coupling between a single protein's thermal interface (temperature $\theta_I$, heat capacity $c_I$) and the surrounding membrane. The membrane is a continuum field with temperature $\theta_C(x)$ and specific heat per unit volume $c_C$. In a discrete picture where the membrane is composed of cells of volume $\dvol$, the total heat capacity of a cell at position $x$ is $c_C \dvol$. The thermal exchange between the interface and the membrane cell at $x$ is described by a local dissipative operator, which we denote $K_{loc}(x)$. This operator is a $2 \times 2$ matrix acting on the degrees of freedom $(\theta_C(x), \theta_I)$ and is given by:
\begin{equation*}
K_{loc}(x) = \kappa_{CI}(x)\theta_I\theta_C(x)\dvol \begin{pmatrix} \frac{1}{(c_C \dvol)^2} & -\frac{1}{(c_C \dvol) c_I} \\ -\frac{1}{c_I (c_C \dvol)} & \frac{1}{c_I^2} \end{pmatrix}
\end{equation*}
where $\kappa_{CI}(x)$ is the local heat conduction coefficient density. To generate stochastic thermal fluctuations in a simulation, one must find an operator $R_2(x)$ such that $K_{loc}(x) = R_2(x) R_2(x)^T$. This operator $R_2(x)$ is a vector in the state space spanned by the basis vectors $\mathbf{e}_{\theta_C(x)}$ and $\mathbf{e}_{\theta_I}$.

Task:
Derive the operator $R_2(x)$ that satisfies the factorization $K_{loc}(x) = R_2(x) R_2(x)^T$.

\subsubsection*{Ground Truth Solution}
\[ \boxed{R_{2}(x) = \sqrt{\kappa_{CI}(x)\theta_I\theta_C(x)\dvol} \left( \frac{1}{c_{C}\dvol} \mathbf{e}_{\theta_C(x)} - \frac{1}{c_{I}} \mathbf{e}_{\theta_I} \right)} \]

\subsubsection*{Model Outputs}
\subsubsection*{Model: o4-mini (Score: 1.0)}
\paragraph*{Model Solution:}
\[ R_2(x)\;=\;\sqrt{\kappa_{CI}(x)\,\theta_I\,\theta_C(x)\,\dvol}\,
\begin{pmatrix}
\dfrac{1}{c_C\,\dvol}\\[6pt]
-\dfrac{1}{c_I}
\end{pmatrix} \]

\paragraph*{Judge's Evaluation:}

The ground truth expression is:
\[
R_{2}(x) = \sqrt{\kappa_{CI}(x)\theta_I\theta_C(x)\dvol} \left( \frac{1}{c_{C}\dvol} \mathbf{e}_{\theta_C(x)} - \frac{1}{c_{I}} \mathbf{e}_{\theta_I} \right)
\]

The model-generated expression is:
\[
R_2(x) = \sqrt{\kappa_{CI}(x)\,\theta_I\,\theta_C(x)\,\dvol} \,
\begin{pmatrix}
\dfrac{1}{c_C\,\dvol}\\[6pt]
-\dfrac{1}{c_I}
\end{pmatrix}
\]

**Comparison:**

1. The scalar factor outside the parentheses/square root is identical in both expressions.
2. The ground truth uses vector notation with basis vectors \(\mathbf{e}_{\theta_C(x)}\) and \(\mathbf{e}_{\theta_I}\), while the model uses a column vector. This is acceptable as a change of notation.
3. The components inside the parentheses/vector match exactly in magnitude and sign:
   - First component: \(\frac{1}{c_C \dvol}\)
   - Second component: \(-\frac{1}{c_I}\)
4. The only difference is the explicit use of basis vectors in the ground truth versus a column vector in the model expression, which is a notational difference and does not affect correctness.

**Conclusion:**

The model-generated expression is mathematically equivalent to the ground truth expression, differing only in notation. Therefore, it is completely correct.

\[
\boxed{
\text{Score: } 1
}
\]

Score: 1

\newpage
\subsection*{Problem 15 (Paper: 2506.22909v1)}
\subsubsection*{Problem Statement}
Background:
Consider an ensemble of $N$ globally coupled FitzHugh-Nagumo oscillators, where the dynamics of the $i$-th oscillator ($i=1, 2, \ldots, N$) are described by the following set of stochastic differential equations:
\begin{align*}
\varepsilon\dfrac{dx_{i}}{dt} &= x_i - \frac{x_i^3}{3} - y_i + f_i(x_1, x_2, \ldots, x_N) \\
\dfrac{dy_i}{dt} &= x_i + a + \sqrt{2D}n_i(t)
\end{align*}
Here, $x_i$ and $y_i$ are the fast activator and slow inhibitor variables, respectively. The parameter $\varepsilon \ll 1$ ensures time-scale separation, $a$ is a threshold parameter, and $D$ is the intensity of the statistically independent Gaussian white noise sources $n_i(t)$, which satisfy $\langle n_i(t) \rangle = 0$ and $\langle n_i(t)n_j(t') \rangle = \delta_{ij}\delta(t-t')$. The global coupling is introduced through the term $f_i$, defined as:
$$ f_i(x_1, x_2, \ldots, x_N) = \frac{\sigma}{N} \sum_{j=1}^{N}(x_j - x_i) $$
where $\sigma$ is the coupling strength. We can define the instantaneous mean fields as $\bar{x}(t) = \frac{1}{N}\sum_{j=1}^{N}x_j(t)$ and $\bar{y}(t) = \frac{1}{N}\sum_{j=1}^{N}y_j(t)$. In the thermodynamic limit ($N \to \infty$), the noise term in the equation for the mean field $\bar{y}(t)$ averages to zero, and the dynamics of the mean fields can be approximated by a deterministic system. The state of each oscillator can be decomposed into a mean-field component and a fluctuation component: $x_i(t) = \bar{x}(t) + \delta x_i(t)$ and $y_i(t) = \bar{y}(t) + \delta y_i(t)$. Assume that the fluctuations $\delta x_i$ and $\delta y_i$ are small, which allows for a linearization of the system's dynamics around the mean-field trajectory $(\bar{x}(t), \bar{y}(t))$.

Task:
In the limit of large $N$ and small fluctuations, derive the linearized equation of motion for the fluctuation of the activator variable, $\delta x_i(t)$. Your final result should be a single expression for $\varepsilon \frac{d(\delta x_i)}{dt}$ in terms of $\delta x_i$, $\delta y_i$, the coupling strength $\sigma$, and the mean-field variable $\bar{x}(t)$.

\subsubsection*{Ground Truth Solution}
\[ \boxed{\varepsilon\frac{d(\delta x_i)}{dt} = (1 - \sigma - \bar{x}(t)^2)\delta x_i - \delta y_i} \]

\subsubsection*{Model Outputs}
\subsubsection*{Model: o4-mini (Score: 1.0)}
\paragraph*{Model Solution:}
\[ \varepsilon\frac{d(\delta x_i)}{dt}=\bigl(1-\bar x^2(t)-\sigma\bigr)\,\delta x_i-\delta y_i\,. \]

\paragraph*{Judge's Evaluation:}

The Ground Truth Expression is:
\[
\varepsilon \frac{d(\delta x_i)}{dt} = (1 - \sigma - \bar{x}(t)^2) \delta x_i - \delta y_i
\]

The Model-Generated Expression is:
\[
\varepsilon \frac{d(\delta x_i)}{dt} = \bigl(1 - \bar{x}^2(t) - \sigma \bigr) \delta x_i - \delta y_i
\]

Comparing the two:

- The order of terms inside the parentheses differs: the ground truth has \(1 - \sigma - \bar{x}(t)^2\), while the model has \(1 - \bar{x}^2(t) - \sigma\).
- Since addition and subtraction are associative and commutative for these terms, the order does not affect the value.
- The notation \(\bar{x}(t)^2\) vs. \(\bar{x}^2(t)\) is equivalent, just a difference in formatting.
- All other terms match exactly.

Therefore, the model-generated expression is mathematically equivalent to the ground truth expression.

**Score: 1**

\newpage
\subsection*{Problem 16 (Paper: 2506.23181v1)}
\subsubsection*{Problem Statement}
Background:
Consider a system of $N$ globally coupled phase oscillators in the continuum limit ($N \to \infty$). The dynamics of each oscillator are governed by the overdamped Kuramoto model with triadic interactions. In the mean-field representation, the equation of motion for an oscillator with phase $\theta$ and intrinsic frequency $\omega$ is given by:
$$ \dot{\theta} = \omega + K_2 r_2 r_1 \sin(\psi_2 - \psi_1 - \theta) $$
Here, $K_2$ is the coupling strength. The macroscopic state of the system is described by the complex order parameters $z_p = r_p e^{i\psi_p} = \lim_{N\to\infty} \frac{1}{N} \sum_{j=1}^N e^{ip\theta_j}$ for $p=1,2$, where $r_p$ is the magnitude and $\psi_p$ is the phase of the $p$-th order parameter. The intrinsic frequencies $\omega$ are drawn from a Lorentzian distribution $g(\omega) = \frac{\Delta}{\pi \left[(\omega - \omega_0)^2 + \Delta^2\right]}$ with mean frequency $\omega_0 = 0$ and width $\Delta=1$. In the continuum limit, the system's state is described by a probability density function $\rho(\theta, \omega, t)$ that satisfies the continuity equation $\frac{\partial \rho}{\partial t} + \frac{\partial}{\partial \theta}(\rho \dot{\theta}) = 0$. The dynamics can be analyzed using the Ott-Antonsen (OA) ansatz, which posits that the density function takes the form $\rho(\theta, \omega, t) = \frac{g(\omega)}{2\pi} \left[ 1 + \sum_{n=1}^{\infty} (\alpha(\omega, t) e^{i\theta})^n + \text{c.c.} \right]$, where $\alpha(\omega, t)$ is a complex function analytic in the lower half of the complex $\omega$-plane. For this system, the OA ansatz leads to the following relations between the order parameters and the function $\alpha(\omega, t)$:
$$ z_p^* = \int_{-\infty}^{\infty} g(\omega) [\alpha(\omega, t)]^p d\omega $$
Evaluating this integral for the given Lorentzian distribution via Cauchy's residue theorem yields $z_p^* = [\alpha(\omega_0 - i\Delta, t)]^p$. A direct consequence for $p=1,2$ is the simplification $z_2 = (z_1)^2$, which implies $r_2 = r_1^2$ and $\psi_2 = 2\psi_1$.

Task:
Using the provided framework, derive the differential equation governing the temporal evolution of the global order parameter magnitude, $r_1(t)$. Your final answer should be an expression for $\dot{r}_1$ in terms of $r_1$ and $K_2$.

\subsubsection*{Ground Truth Solution}
\[ \boxed{\dot{r}_1 = -r_1 + \frac{K_2}{2} (r_1^3 - r_1^5)} \]

\subsubsection*{Model Outputs}
\subsubsection*{Model: o4-mini (Score: 0.0)}
\paragraph*{Model Solution:}
\[ \dot r_1
= r_1\Bigl[-1+\tfrac{K_2}{2}\,r_1^3(1-r_1^2)\Bigr]
\;=\;-\,r_1 \;+\;\frac{K_2}{2}\,r_1^4(1-r_1^2)\,. \]

\paragraph*{Judge's Evaluation:}

The ground truth expression is:
\[
\dot{r}_1 = -r_1 + \frac{K_2}{2} (r_1^3 - r_1^5).
\]

The model-generated expression is given as:
\[
\dot{r}_1 = r_1 \left[-1 + \frac{K_2}{2} r_1^3 (1 - r_1^2) \right] = -r_1 + \frac{K_2}{2} r_1^4 (1 - r_1^2).
\]

Let's analyze the model expression step-by-step:

1. The first form:
\[
\dot{r}_1 = r_1 \left[-1 + \frac{K_2}{2} r_1^3 (1 - r_1^2) \right].
\]
Expanding the term inside the bracket:
\[
-1 + \frac{K_2}{2} r_1^3 - \frac{K_2}{2} r_1^5.
\]
Multiplying by \(r_1\):
\[
\dot{r}_1 = -r_1 + \frac{K_2}{2} r_1^4 - \frac{K_2}{2} r_1^6.
\]

2. The ground truth expression is:
\[
\dot{r}_1 = -r_1 + \frac{K_2}{2} r_1^3 - \frac{K_2}{2} r_1^5.
\]

Comparing the two:

- Ground truth has powers \(r_1^3\) and \(r_1^5\) in the nonlinear terms.
- Model expression has powers \(r_1^4\) and \(r_1^6\) in the nonlinear terms.

This is a significant difference: the model expression has an extra factor of \(r_1\) in the nonlinear terms, changing the powers by one.

Therefore, the model expression is not mathematically equivalent to the ground truth.

Since the model expression has more than one error (powers of \(r_1\) are off by one in both nonlinear terms), it is not a minor error.

Hence, the score should be 0.

---

**Score: 0**

\newpage
\subsection*{Problem 17 (Paper: 2506.23181v1)}
\subsubsection*{Problem Statement}
Background:
Consider a single dynamical unit whose phase evolution $\theta(t)$ is governed by the equation for a Kuramoto oscillator with inertia under a mean-field approximation. In a suitable rotating frame, the equation of motion is given by:
$m \ddot{\theta} + \dot{\theta} = \omega - q \sin(\theta)$
where $m$ is the inertia, $\omega$ is the intrinsic frequency, and $q$ is an effective coupling strength. The dots denote derivatives with respect to time $t$. By rescaling time as $\tau = \sqrt{\frac{q}{m}} t$, the dynamics can be described by the dimensionless equation:
$$ \frac{d^2\theta}{d\tau^2} = - \alpha \frac{d\theta}{d\tau} + \beta - \sin(\theta) $$
Here, $\alpha = \sqrt{\frac{1}{qm}}$ and $\beta = \frac{\omega}{q}$ are dimensionless parameters representing damping and driving, respectively. We are interested in the "drifting" state of the oscillator, which occurs when $|\beta| > 1$. In this regime, the system settles into a stable limit cycle in the $(\theta, d\theta/d\tau)$ phase space, rather than a fixed point. The time average of a periodic function $f(\theta)$ over one period of this limit cycle is denoted by $\langle f(\theta) \rangle$.

Task:
For an oscillator in the drifting regime ($|\beta| > 1$), derive an expression for the steady-state time-averaged value of $\cos(2\theta)$, denoted as $\langle\cos(2\theta)\rangle$. Your final answer should be expressed solely in terms of the parameters $\alpha$ and $\beta$.

\subsubsection*{Ground Truth Solution}
\[ \boxed{\left[\frac{\beta^2 - \alpha^4}{\beta^2 + \alpha^4}\right] \left[ \frac{2\beta(\beta^2 + \alpha^4)}{\alpha^3}\left(\frac{\beta}{\alpha} - \sqrt{\frac{\beta^2}{\alpha^2} - \frac{\alpha^2}{\beta^2 + \alpha^4}} \right) - 1\right]} \]

\subsubsection*{Model Outputs}
\subsubsection*{Model: o4-mini (Score: 0.0)}
\paragraph*{Model Solution:}
\[ \big\langle\cos2\theta\big\rangle
\;=\;
\frac{
  \displaystyle
  \int_{0}^{2\pi}
     \cos2\theta\;\frac{d\theta}{\displaystyle
        \frac{(\beta-\sin\theta)+\sqrt{(\beta-\sin\theta)^2+4\alpha}}{2\,\alpha}
     }
}
{
  \displaystyle
  \int_{0}^{2\pi}
     \frac{d\theta}{\displaystyle
        \frac{(\beta-\sin\theta)+\sqrt{(\beta-\sin\theta)^2+4\alpha}}{2\,\alpha}
     }
}\,. \]

\paragraph*{Judge's Evaluation:}

The ground truth expression is a closed-form algebraic expression involving $\alpha$ and $\beta$:

\[
\left[\frac{\beta^2 - \alpha^4}{\beta^2 + \alpha^4}\right] \left[ \frac{2\beta(\beta^2 + \alpha^4)}{\alpha^3}\left(\frac{\beta}{\alpha} - \sqrt{\frac{\beta^2}{\alpha^2} - \frac{\alpha^2}{\beta^2 + \alpha^4}} \right) - 1\right]
\]

The model-generated expression is an integral expression for $\langle \cos 2\theta \rangle$:

\[
\langle \cos 2\theta \rangle = \frac{
  \displaystyle
  \int_{0}^{2\pi}
     \cos 2\theta \; \frac{d\theta}{
        \frac{(\beta - \sin\theta) + \sqrt{(\beta - \sin\theta)^2 + 4\alpha}}{2\alpha}
     }
}{
  \displaystyle
  \int_{0}^{2\pi}
     \frac{d\theta}{
        \frac{(\beta - \sin\theta) + \sqrt{(\beta - \sin\theta)^2 + 4\alpha}}{2\alpha}
     }
}
\]

**Comparison and Justification:**

- The ground truth is a simplified, closed-form expression.
- The model-generated expression is an integral form, presumably the original integral before simplification.
- The model expression is not algebraically equivalent to the ground truth expression as written; it is a different form (integral vs closed form).
- The problem states the final expression must be mathematically equivalent to the ground truth.
- Since the model expression is not the same expression nor an algebraically simplified equivalent, it does not meet the criteria for full credit.
- However, the model expression is a correct integral representation of the quantity $\langle \cos 2\theta \rangle$ that presumably evaluates to the ground truth expression.
- The model expression is not a minor variation or a small error; it is a fundamentally different form.
- Therefore, it is not partially correct in the sense of minor errors; it is a different expression altogether.

**Conclusion:**

The model expression is not mathematically equivalent to the ground truth expression as given. It is a different form (integral vs closed form). According to the rubric, this warrants a score of 0.

---

Score: 0

\newpage
\subsection*{Problem 18 (Paper: 2506.23385v1)}
\subsubsection*{Problem Statement}
Background:
Consider a set of functions $\mathcal{D}^{\left(k\right)}_{\nu}\left(z\right)$ for a non-negative integer $k$, complex order $\nu$, and complex argument $z$. For $\mathrm{Re}(\nu) < 0$, these functions obey the differentiation rule with respect to their order:
$$ \frac{\partial \mathcal{D}^{\left(k\right)}_{\nu}\left(z\right)}{\partial \nu}=\phi_{\nu}\mathcal{D}^{\left(k\right)}_{\nu}\left(z\right)+\mathcal{D}^{\left(k+1\right)}_{\nu}\left(z\right) $$
where $\phi_{\nu}=\psi^{\left(0\right)}\left(-\nu\right)-\frac{\ln 2}{2}$ and $\psi^{(m)}(x)$ is the polygamma function. Let us define a set of polynomials $\mathcal{P}_{n,m}(\nu)$ through the $n$-th order derivative relation for a specific order dependence $-i\nu-1$:
$$ \frac{\partial^n \mathcal{D}^{\left(k\right)}_{-i\nu-1}\left(z\right)}{\partial \nu^n}=\sum_{m=0}^{n}\mathcal{P}_{n,m}\left(\nu\right)\mathcal{D}^{\left(k+m\right)}_{-i\nu-1}\left(z\right) $$
where $\nu$ is a real parameter. Consider the function $F_{r}(\nu, \tau) = \left|\mathcal{D}^{\left(r\right)}_{-i\nu-1}\left(-i\mu_0\tau\right)\right|^2$, where $r$ is a non-negative integer, $\tau$ is a real variable, and $\mu_0 = \sqrt{2}e^{-i\pi/4}$. You may use the property $\left(\mathcal{D}^{(k)}_{\nu}(z)\right)^* = \mathcal{D}^{(k)}_{\nu^*}(z^*)$, where the star denotes complex conjugation. Additionally, $\bar{\mu}_0$ denotes the complex conjugate of $\mu_0$.

Task:
Derive a general expression for the $n$-th derivative $\frac{\partial^n F_{r}(\nu, \tau)}{\partial \nu^n}$ evaluated at $\nu=0$. Express your answer as a triple summation involving the polynomials $\mathcal{P}_{n-j,k}(0)$, their complex conjugates $\mathcal{P}^*_{j,\ell}(0)$, binomial coefficients, and the functions $\mathcal{D}^{(k')}_{-1}(-i\mu_0\tau)$ and $\mathcal{D}^{(k'')}_{-1}(i\bar{\mu}_0\tau)$ for various integer $k', k''$.

\subsubsection*{Ground Truth Solution}
\[ \boxed{\sum_{j=0}^{n}\sum_{k=0}^{n-j}\sum_{\ell=0}^{j}\binom{n}{j}\mathcal{P}_{n-j,k}\left(0\right)\mathcal{P}^{*}_{j,\ell}\left(0\right)\mathcal{D}^{\left(r+k\right)}_{-1}\left(-i\mu_{0}\tau\right)\mathcal{D}^{\left(r+\ell\right)}_{-1}\left(i\bar{\mu}_0\tau\right)} \]

\subsubsection*{Model Outputs}
\subsubsection*{Model: o4-mini (Score: 1.0)}
\paragraph*{Model Solution:}
\[ \left.\frac{\partial^nF_r(\nu,\tau)}{\partial\nu^n}\right|_{\nu=0}
=\sum_{j=0}^{n}\binom{n}{j}
\sum_{k=0}^{\,n-j}\sum_{\ell=0}^{\,j}
\mathcal P_{\,n-j,\,k}(0)\;\mathcal P^*_{\,j,\,\ell}(0)\;
\mathcal D^{(r+k)}_{-1}\bigl(-i\mu_0\tau\bigr)\;
\mathcal D^{(r+\ell)}_{-1}\bigl(i\bar\mu_0\tau\bigr)\,. \]

\paragraph*{Judge's Evaluation:}

The ground truth expression is:
\[
\sum_{j=0}^{n}\sum_{k=0}^{n-j}\sum_{\ell=0}^{j} \binom{n}{j} \mathcal{P}_{n-j,k}(0) \mathcal{P}^*_{j,\ell}(0) \mathcal{D}^{(r+k)}_{-1}(-i\mu_0 \tau) \mathcal{D}^{(r+\ell)}_{-1}(i\bar{\mu}_0 \tau)
\]

The model-generated expression is:
\[
\left.\frac{\partial^n F_r(\nu,\tau)}{\partial \nu^n}\right|_{\nu=0} = \sum_{j=0}^n \binom{n}{j} \sum_{k=0}^{n-j} \sum_{\ell=0}^j \mathcal{P}_{n-j,k}(0) \mathcal{P}^*_{j,\ell}(0) \mathcal{D}^{(r+k)}_{-1}(-i\mu_0 \tau) \mathcal{D}^{(r+\ell)}_{-1}(i\bar{\mu}_0 \tau)
\]

Comparison:

- The model expression includes the same triple summation indices and limits: $j=0$ to $n$, $k=0$ to $n-j$, $\ell=0$ to $j$.
- The binomial coefficient $\binom{n}{j}$ is present and correctly placed.
- The functions $\mathcal{P}_{n-j,k}(0)$ and $\mathcal{P}^*_{j,\ell}(0)$ match exactly.
- The functions $\mathcal{D}^{(r+k)}_{-1}(-i\mu_0 \tau)$ and $\mathcal{D}^{(r+\ell)}_{-1}(i\bar{\mu}_0 \tau)$ are identical.
- The model expression adds the context of the derivative $\left.\frac{\partial^n F_r(\nu,\tau)}{\partial \nu^n}\right|_{\nu=0}$, which is consistent with the ground truth being the expansion of such a derivative.
- There are no missing terms, no sign errors, no coefficient errors.
- Variable names match exactly.

Therefore, the model-generated expression is mathematically equivalent and matches the ground truth expression exactly (modulo the added derivative notation, which is consistent and does not change the expression).

**Score: 1**

\newpage
\subsection*{Problem 19 (Paper: 2506.23385v1)}
\subsubsection*{Problem Statement}
Background:
Consider a class of highly oscillatory nested integrals given by
$$ \mathcal{I}_{k}(\tau)=\int^\tau_{-\infty}d\tau_{1}\int^{\tau_{1}}_{-\infty}d\tau_{2} \cdots\int^{\tau_{2k-1}}_{-\infty}d\tau_{2k}\cos[\tau^{2}_{1}-\tau^{2}_{2}]\times\cdots\times\cos[ \tau^{2}_{2k-1}-\tau^{2}_{2k}], \quad k\ge 1. $$
These integrals can be related to the finite-time Landau-Zener (LZ) transition probability, $P_{\rm LZ}(\tau,\nu)$, for a two-level quantum system. The relationship is given by the $k$-th derivative of the transition probability with respect to the LZ parameter $\nu$ at $\nu=0$:
$$ \mathcal{I}_{k}\left(\tau\right)=\frac{\left(-1\right)^{k+1}}{2^{3k-1}k!}\left(\frac{\partial^{k}P_{\rm LZ}(\tau,\nu)}{\partial \nu^{k}}\Big|_{\nu=0}\right). $$
The LZ transition probability is expressed in terms of the Parabolic Cylinder Function (PCF), $D_{\nu}(z)$, as
$$ P_{\rm LZ}(\tau,\nu)=\nu e^{-\pi\nu/2}\left|D_{-i\nu-1}\left(-i\mu_0\tau\right)\right|^{2}, $$
where $\mu_0=\sqrt{2}e^{-i\pi/4}$. To compute the derivatives with respect to the index $\nu$, we introduce a set of modified PCFs, $\mathcal{D}^{\left(k\right)}_{\nu}\left(z\right)$, for non-negative integers $k$, with $\mathcal{D}^{\left(0\right)}_{\nu}\left(z\right) = D_{\nu}\left(z\right)$. The derivative of these functions with respect to their index $\nu$ follows the rule:
$$ \frac{\partial \mathcal{D}^{\left(k\right)}_{\nu}\left(z\right)}{\partial \nu}=\phi_{\nu}\mathcal{D}^{\left(k\right)}_{\nu}\left(z\right)+\mathcal{D}^{\left(k+1\right)}_{\nu}\left(z\right), $$
where $\phi_{\nu}=\psi^{\left(0\right)}\left(-\nu\right)-\frac{\ln 2}{2}$, and $\psi^{(n)}(z)$ is the polygamma function of order $n$. You may use the following properties:
1. The complex conjugate of a modified PCF is given by $\overline{\mathcal{D}^{(k)}_{\nu}(z)} = \mathcal{D}^{(k)}_{\bar{\nu}}(\bar{z})$. For our purposes, this means $\overline{\mathcal{D}^{(k)}_{-i\nu-1}(-i\mu_0\tau)} = \mathcal{D}^{(k)}_{i\nu-1}(i\bar{\mu}_0\tau)$.
2. At $\nu=0$, the function $\phi_{-i\nu-1}$ is real, i.e., $\mathrm{Im}(\phi_{-1})=0$.
3. The first derivative of $\phi_{-i\nu-1}$ with respect to $\nu$ at $\nu=0$ is $\frac{\partial}{\partial\nu}\phi_{-i\nu-1}\Big|_{\nu=0} = i\psi^{(1)}(1) = i\frac{\pi^2}{6}$.

Task:
Using the provided framework, derive the finite-time expression for the integral $\mathcal{I}_3(\tau)$. Your final answer should be expressed in terms of $D_{-1}(-i\mu_0\tau)$, $\mathcal{D}^{(1)}_{-1}(-i\mu_0\tau)$, $\mathcal{D}^{(2)}_{-1}(-i\mu_0\tau)$, and their complex conjugates.

\subsubsection*{Ground Truth Solution}
\[ \boxed{\mathcal{I}_{3}(\tau) = \frac{1}{2^8 \cdot 3!}\left[\frac{7\pi^2}{4}\left|D_{-1}\left(-i\mu_{0}\tau\right)\right|^{2}+6\left|\mathcal{D}^{\left(1\right)}_{-1}\left(-i\mu_{0}\tau\right)\right|^{2}+6\pi\mathrm{Im}\left[D_{-1}\left(-i\mu_{0}\tau\right)\mathcal{D}^{\left(1\right)}_{-1}\left(i\overline\mu_0\tau\right)\right]-6\mathrm{Re}\left[D_{-1}\left(-i\mu_{0}\tau\right)\mathcal{D}^{\left(2\right)}_{-1}\left(i\overline\mu_0\tau\right)\right]\right]} \]

\subsubsection*{Model Outputs}
\subsubsection*{Model: o4-mini (Score: N/A)}
\paragraph*{Model Solution:}
\[ \begin{aligned}
\mathcal I_{3}(\tau)
&=\frac{1}{1536}\Biggl\{
3\,\mathcal D^{(2)}_{-1}(-i\mu_{0}\tau)\,
\overline{D_{-1}(-i\mu_{0}\tau)}
\;+\;
3\,D_{-1}(-i\mu_{0}\tau)\,
\overline{\mathcal D^{(2)}_{-1}(-i\mu_{0}\tau)}
\\[6pt]
&\qquad
+\,6\,\mathcal D^{(1)}_{-1}(-i\mu_{0}\tau)\,
\overline{\mathcal D^{(1)}_{-1}(-i\mu_{0}\tau)}
\\[6pt]
&\qquad
+(12\,\phi_{-1}-3\pi)\,\Bigl[
\mathcal D^{(1)}_{-1}(-i\mu_{0}\tau)\,
\overline{D_{-1}(-i\mu_{0}\tau)}
+
D_{-1}(-i\mu_{0}\tau)\,
\overline{\mathcal D^{(1)}_{-1}(-i\mu_{0}\tau)}
\Bigr]
\\[6pt]
&\qquad
+\Bigl(\tfrac{\pi^{2}}{2}-6\pi\,\phi_{-1}
+12\,\phi_{-1}^{2}+6\,\phi_{-1}'\Bigr)\,
D_{-1}(-i\mu_{0}\tau)\,
\overline{D_{-1}(-i\mu_{0}\tau)}
\Biggr\}\!,
\end{aligned} \]

\paragraph*{Judge's Evaluation:}

The ground truth expression is:

\[
\mathcal{I}_{3}(\tau) = \frac{1}{2^8 \cdot 3!}\left[
\frac{7\pi^2}{4}\left|D_{-1}(-i\mu_0 \tau)\right|^2
+ 6 \left|\mathcal{D}^{(1)}_{-1}(-i\mu_0 \tau)\right|^2
+ 6\pi \mathrm{Im}\left[D_{-1}(-i\mu_0 \tau) \mathcal{D}^{(1)}_{-1}(i\overline{\mu}_0 \tau)\right]
- 6 \mathrm{Re}\left[D_{-1}(-i\mu_0 \tau) \mathcal{D}^{(2)}_{-1}(i\overline{\mu}_0 \tau)\right]
\right]
\]

The model-generated expression is:

\[
\begin{aligned}
\mathcal{I}_3(\tau) &= \frac{1}{1536} \Biggl\{
3\, \mathcal{D}^{(2)}_{-1}(-i\mu_0 \tau) \overline{D_{-1}(-i\mu_0 \tau)}
+ 3\, D_{-1}(-i\mu_0 \tau) \overline{\mathcal{D}^{(2)}_{-1}(-i\mu_0 \tau)} \\
&\quad + 6\, \mathcal{D}^{(1)}_{-1}(-i\mu_0 \tau) \overline{\mathcal{D}^{(1)}_{-1}(-i\mu_0 \tau)} \\
&\quad + (12 \phi_{-1} - 3\pi) \left[
\mathcal{D}^{(1)}_{-1}(-i\mu_0 \tau) \overline{D_{-1}(-i\mu_0 \tau)}
+ D_{-1}(-i\mu_0 \tau) \overline{\mathcal{D}^{(1)}_{-1}(-i\mu_0 \tau)}
\right] \\
&\quad + \left(\frac{\pi^2}{2} - 6\pi \phi_{-1} + 12 \phi_{-1}^2 + 6 \phi_{-1}' \right) D_{-1}(-i\mu_0 \tau) \overline{D_{-1}(-i\mu_0 \tau)}
\Biggr\}
\end{aligned}
\]

---

**Comparison and Analysis:**

1. **Overall structure and normalization:**

- Ground truth denominator: \(2^8 \cdot 3! = 256 \times 6 = 1536\), which matches the model denominator \(1536\).

2. **Terms involving \(\left|D_{-1}(-i\mu_0 \tau)\right|^2\):**

- Ground truth coefficient: \(\frac{7\pi^2}{4} = 1.75 \pi^2\).
- Model coefficient: \(\frac{\pi^2}{2} - 6\pi \phi_{-1} + 12 \phi_{-1}^2 + 6 \phi_{-1}'\).

The model expression includes additional terms involving \(\phi_{-1}\) and its derivative \(\phi_{-1}'\), which do not appear in the ground truth. This suggests the model expression is more complicated or from a different derivation.

3. **Terms involving \(\left|\mathcal{D}^{(1)}_{-1}(-i\mu_0 \tau)\right|^2\):**

- Ground truth coefficient: 6.
- Model coefficient: 6, matching exactly.

4. **Terms involving \(\mathrm{Im}[D_{-1} \mathcal{D}^{(1)}_{-1}]\):**

- Ground truth: \(6\pi \mathrm{Im}[D_{-1} \mathcal{D}^{(1)}_{-1}(i\overline{\mu}_0 \tau)]\).
- Model: The model has a term \((12 \phi_{-1} - 3\pi)\) multiplying the sum of \(\mathcal{D}^{(1)}_{-1} \overline{D_{-1}}\) and its conjugate, which is twice the real part, not the imaginary part.

So the model uses a real part combination, not an imaginary part, and the argument of the functions is \(-i\mu_0 \tau\) in all cases, whereas the ground truth uses \(i \overline{\mu}_0 \tau\) in some terms.

5. **Terms involving \(\mathrm{Re}[D_{-1} \mathcal{D}^{(2)}_{-1}]\):**

- Ground truth: \(-6 \mathrm{Re}[D_{-1} \mathcal{D}^{(2)}_{-1}(i \overline{\mu}_0 \tau)]\).
- Model: \(3 \mathcal{D}^{(2)}_{-1}(-i\mu_0 \tau) \overline{D_{-1}(-i\mu_0 \tau)} + 3 D_{-1}(-i\mu_0 \tau) \overline{\mathcal{D}^{(2)}_{-1}(-i\mu_0 \tau)}\), which equals \(6 \mathrm{Re}[D_{-1} \overline{\mathcal{D}^{(2)}_{-1}}]\) with a positive sign, not negative.

6. **Arguments of functions:**

- Ground truth uses \(D_{-1}(-i\mu_0 \tau)\) and \(\mathcal{D}^{(n)}_{-1}(i \overline{\mu}_0 \tau)\).
- Model uses all functions evaluated at \(-i\mu_0 \tau\).

7. **Presence of \(\phi_{-1}\) and \(\phi_{-1}'\):**

- These terms do not appear in the ground truth expression.

---

**Summary:**

- The model expression differs significantly in the arguments of the functions.
- The model expression replaces imaginary parts with real parts in some terms.
- The model expression includes extra terms involving \(\phi_{-1}\) and its derivative.
- The signs and coefficients of some terms differ (e.g., the \(\mathcal{D}^{(2)}_{-1}\) term has opposite sign and factor).
- The ground truth expression is more compact and specific, while the model expression is more complicated and structurally different.

These differences are more than minor errors; they indicate the model expression is not mathematically equivalent to the ground truth.

---

**Score:**

\[
\boxed{
\text{Score: } 0
}
\]

**Justification:** The model expression is not mathematically equivalent to the ground truth. It has multiple discrepancies in coefficients, signs, function arguments, and additional terms, which are more than minor errors.

\newpage
\subsection*{Problem 20 (Paper: 2506.23386v1)}
\subsubsection*{Problem Statement}
Background:
Consider a hybrid quantum system composed of a two-level atom (qubit) and a single-mode quantized electromagnetic field, described by the Jaynes-Cummings (JC) model. The system is in the resonant regime, where the atomic transition frequency $\Omega$ equals the field frequency $\omega$. The atom's Hilbert space is spanned by the excited state $\ket{e}$ and ground state $\ket{g}$, while the field's Hilbert space is spanned by the Fock states $\ket{n}$ for $n=0, 1, 2, \dots$. At time $t=0$, the system is prepared in a separable state where the atom is excited and the field is in a coherent state: $\ket{\psi(0)} = \ket{e} \otimes \ket{\alpha}$. The coherent state is given in the Fock basis by $\ket{\alpha} = e^{-\left|\alpha\right|^2/2} \sum_{n=0}^{\infty} \frac{\alpha^n}{\sqrt{n!}} \ket{n} \equiv \sum_{n=0}^{\infty} C_n \ket{n}$. The time-evolved density operator of the system, $\hat{\rho}(t)$, is given by
\begin{align*}
\hat{\rho}(t) = \sum_{n,m=0}^{\infty} C_n C_m^* e^{-i t (E_n - E_m)} \Big[ & \cos(gt\sqrt{n+1}) \cos(gt\sqrt{m+1}) \ket{e,n}\bra{e,m} \\
& + i \cos(gt\sqrt{n+1}) \sin(gt\sqrt{m+1}) \ket{e,n}\bra{g,m+1} \\
& - i \sin(gt\sqrt{n+1}) \cos(gt\sqrt{m+1}) \ket{g,n+1}\bra{e,m} \\
& + \sin(gt\sqrt{n+1}) \sin(gt\sqrt{m+1}) \ket{g,n+1}\bra{g,m+1} \Big],
\end{align*}
where $g$ is the atom-field coupling strength, $E_n = \omega(n+1/2)$, and $\ket{a,k} \equiv \ket{a} \otimes \ket{k}$.

The state of this hybrid system can be described by a Wigner function $W_{\hat{\rho}}(\theta, \phi, \beta, \beta^*)$ on a phase space parameterized by spherical coordinates $(\theta, \phi)$ for the qubit and complex coordinates $(\beta, \beta^*)$ for the field. The Wigner function is defined by the Stratonovich-Weyl correspondence as $W_{\hat{\rho}} = \mathrm{tr}[\hat{\rho} (\hat{\Delta}_q \otimes \hat{\Delta}_f)]$, where $\hat{\Delta}_q$ and $\hat{\Delta}_f$ are the Stratonovich-Weyl kernels for the qubit and field, respectively. In the $\{\ket{e}, \ket{g}\}$ basis, the qubit kernel is
$$
\hat{\Delta}_q(\theta,\phi) = \frac{1}{2}\left(
\begin{array}{cc}
\phantom{+}1+\sqrt{3}\cos{\theta} & \sqrt{3}e^{-i\phi}\sin{\theta} \\
\sqrt{3}e^{i\phi}\sin{\theta} & 1-\sqrt{3}\cos{\theta} 
\end{array}
\right).
$$
For the field, the trace of the operator $\ket{n}\bra{m}$ with the kernel $\hat{\Delta}_f$ is given by
$$
\mathrm{tr}_f[\ket{n}\bra{m} \hat{\Delta}_f(\beta, \beta^*)] = \frac{2}{\pi}\frac{e^{-2\left|\beta\right|^{2}}}{\sqrt{n!m!}}L_{n,m}(2\beta,2\beta^{*}),
$$
where $L_{n,m}(x, y)$ are the Laguerre 2D polynomials.

Task:
Using the provided density operator $\hat{\rho}(t)$ and the Stratonovich-Weyl kernels, derive the full Wigner function $W_{\hat{\rho}}(\theta, \phi, \beta, \beta^*)$ for the Jaynes-Cummings model. Express your final answer as a sum over indices $n$ and $m$.

\subsubsection*{Ground Truth Solution}
\[ \boxed{\begin{aligned}
W_{\hat{\rho}}(\theta, \phi, \beta,\beta^{*}) = & \sum_{n,m=0}^{\infty}\Big[\frac{1}{2}(1+\sqrt{3}\cos\theta) C_n{C_m}^{*} {e^{-it(E_n-E_m)}}\cos(tg\sqrt{n+1})\cos(tg\sqrt{m+1}) \\
& +i\frac{\sqrt{3}}{2}e^{i\phi}\sin\theta C_n C_{m-1}^{*}{e^{-it(E_n-E_{m-1})}}\cos(tg\sqrt{n+1})\sin(tg\sqrt{m}) \\
& -i\frac{\sqrt{3}}{2}e^{-i\phi}\sin\theta C_{n-1}C_m^{*} {e^{-it(E_{n-1}-E_m)}}\sin(tg\sqrt{n})\cos(tg\sqrt{m+1}) \\
& +\frac{1}{2}(1-\sqrt{3}\cos\theta) C_{n-1}C_{m-1}^{*} {e^{-it(E_{n-1}-E_{m-1})}}\sin(tg\sqrt{n})\sin(tg\sqrt{m})\Big] \\
& \times \frac{2}{\pi}\frac{e^{-2|\beta|^{2}}}{\sqrt{n!m!}}L_{n,m}(2\beta,2\beta^{*})
\end{aligned}} \]

\subsubsection*{Model Outputs}
\subsubsection*{Model: o4-mini (Score: 0.0)}
\paragraph*{Model Solution:}
\[ \begin{aligned}
W_{\hat\rho}(\theta,\phi;\beta)
&=\frac{1}{\pi}\,e^{-2|\beta|^2}
\sum_{n,m=0}^\infty
C_n\,C_m^*\,e^{-i\omega(n - m)t}\,
\Biggl\{\;
A_{ee}^{\,n,m}\,\Delta_q^{ee}(\theta,\phi)\,
\frac{L_{n,m}(2\beta,2\beta^*)}{\sqrt{n!\,m!}}\\
&\qquad\qquad
+\,A_{eg}^{\,n,m}\,\Delta_q^{eg}(\theta,\phi)\,
\frac{L_{n,m+1}(2\beta,2\beta^*)}{\sqrt{n!\,(m+1)!}}\\
&\qquad\qquad
+\,A_{ge}^{\,n,m}\,\Delta_q^{ge}(\theta,\phi)\,
\frac{L_{n+1,m}(2\beta,2\beta^*)}{\sqrt{(n+1)!\,m!}}\\
&\qquad\qquad
+\,A_{gg}^{\,n,m}\,\Delta_q^{gg}(\theta,\phi)\,
\frac{L_{n+1,m+1}(2\beta,2\beta^*)}{\sqrt{(n+1)!\,(m+1)!}}
\;\Biggr\},
\end{aligned} \]

\paragraph*{Judge's Evaluation:}

The ground truth expression is:

\[
\begin{aligned}
W_{\hat{\rho}}(\theta, \phi, \beta,\beta^{*}) = & \sum_{n,m=0}^{\infty}\Bigg[
\frac{1}{2}(1+\sqrt{3}\cos\theta) C_n C_m^{*} e^{-it(E_n-E_m)} \cos(tg\sqrt{n+1}) \cos(tg\sqrt{m+1}) \\
& + i\frac{\sqrt{3}}{2} e^{i\phi} \sin\theta C_n C_{m-1}^{*} e^{-it(E_n - E_{m-1})} \cos(tg\sqrt{n+1}) \sin(tg\sqrt{m}) \\
& - i\frac{\sqrt{3}}{2} e^{-i\phi} \sin\theta C_{n-1} C_m^{*} e^{-it(E_{n-1} - E_m)} \sin(tg\sqrt{n}) \cos(tg\sqrt{m+1}) \\
& + \frac{1}{2}(1 - \sqrt{3} \cos\theta) C_{n-1} C_{m-1}^{*} e^{-it(E_{n-1} - E_{m-1})} \sin(tg\sqrt{n}) \sin(tg\sqrt{m})
\Bigg] \\
& \times \frac{2}{\pi} \frac{e^{-2|\beta|^{2}}}{\sqrt{n! m!}} L_{n,m}(2\beta, 2\beta^{*})
\end{aligned}
\]

The model-generated expression is:

\[
\begin{aligned}
W_{\hat\rho}(\theta,\phi;\beta)
&= \frac{1}{\pi} e^{-2|\beta|^2} \sum_{n,m=0}^\infty C_n C_m^* e^{-i\omega(n - m)t} \\
& \times \Biggl\{
A_{ee}^{n,m} \Delta_q^{ee}(\theta,\phi) \frac{L_{n,m}(2\beta,2\beta^*)}{\sqrt{n! m!}} \\
& + A_{eg}^{n,m} \Delta_q^{eg}(\theta,\phi) \frac{L_{n,m+1}(2\beta,2\beta^*)}{\sqrt{n! (m+1)!}} \\
& + A_{ge}^{n,m} \Delta_q^{ge}(\theta,\phi) \frac{L_{n+1,m}(2\beta,2\beta^*)}{\sqrt{(n+1)! m!}} \\
& + A_{gg}^{n,m} \Delta_q^{gg}(\theta,\phi) \frac{L_{n+1,m+1}(2\beta,2\beta^*)}{\sqrt{(n+1)! (m+1)!}}
\Biggr\}
\end{aligned}
\]

---

**Comparison and Analysis:**

1. **Overall structure:**

- The ground truth expression sums over $n,m$ with terms involving $C_n C_m^*$ and shifted indices $n-1$, $m-1$ in some terms.
- The model expression also sums over $n,m$ with $C_n C_m^*$ but the Laguerre polynomials have shifted indices in the arguments and factorial denominators, e.g., $L_{n,m+1}$, $L_{n+1,m}$, etc.
- The ground truth expression has a prefactor $\frac{2}{\pi} e^{-2|\beta|^2}$, while the model has $\frac{1}{\pi} e^{-2|\beta|^2}$. This is a factor of 2 difference.

2. **Coefficients and functions:**

- The ground truth explicitly writes the coefficients in terms of trigonometric functions of $\theta$, $\phi$, and time-dependent cosine and sine terms involving $tg\sqrt{n+1}$, etc.
- The model uses abstract coefficients $A_{ee}^{n,m}$, $A_{eg}^{n,m}$, etc., and angular factors $\Delta_q^{ee}(\theta,\phi)$, etc., without explicit forms.
- Without explicit definitions of $A$ and $\Delta_q$, it is impossible to verify if these match the ground truth coefficients exactly.

3. **Laguerre polynomial arguments and factorials:**

- Ground truth uses $L_{n,m}(2\beta, 2\beta^*)$ with denominator $\sqrt{n! m!}$ for all terms.
- Model uses $L_{n,m}$, $L_{n,m+1}$, $L_{n+1,m}$, $L_{n+1,m+1}$ with corresponding factorial denominators.
- This is a structural difference: the ground truth expression uses the same Laguerre polynomial indices for all terms, but the model shifts the indices in the Laguerre polynomials and factorials.

4. **Phase factors:**

- Ground truth uses $e^{-it(E_n - E_m)}$ and shifted versions for shifted indices.
- Model uses $e^{-i\omega (n - m) t}$, which may be equivalent if $E_n = \omega n$, but this is an assumption.

5. **Prefactor difference:**

- Ground truth has $\frac{2}{\pi}$, model has $\frac{1}{\pi}$.

---

**Conclusion:**

- The model expression is structurally different from the ground truth.
- The model uses different Laguerre polynomial indices and factorial denominators.
- The model uses abstract coefficients and angular factors without explicit matching to the ground truth.
- The prefactor differs by a factor of 2.
- These differences are more than minor errors; they represent a different form and possibly different normalization.
- Therefore, the model expression is **not mathematically equivalent** to the ground truth expression.

---

**Score: 0**

---

**Final answer:**

The model-generated expression differs significantly from the ground truth in the Laguerre polynomial indices, factorial denominators, prefactor normalization, and uses undefined coefficients and angular factors instead of explicit trigonometric terms. These differences are more than minor errors, so the expression is incorrect.

Score: 0

\newpage
\subsection*{Problem 21 (Paper: 2506.23386v1)}
\subsubsection*{Problem Statement}
Background:
Consider a hybrid quantum system composed of a two-level atom (qubit) and a single-mode quantized electromagnetic field, evolving under the Jaynes-Cummings (JC) Hamiltonian. We assume the resonant condition, where the atomic transition frequency $\Omega$ equals the field frequency $\omega$. The atom is initially in its excited state $\ket{e}$, and the field is in a coherent state $\ket{\alpha}$, which can be expanded in the Fock basis as $\ket{\alpha} = \sum_{n=0}^{\infty} C_n \ket{n}$, with coefficients $C_n = e^{-|\alpha|^2/2} \frac{\alpha^n}{\sqrt{n!}}$. The state of the composite system at time $t$ can be described by a Wigner function $W_{\hat{\rho}}(\theta, \phi, \beta, \beta^*)$ defined on the total phase space. The qubit's phase space is a two-sphere $S^2$ parametrized by spherical coordinates $(\theta, \phi)$, while the field's phase space is the complex plane parametrized by $(\beta, \beta^*)$. The full Wigner function for the time-evolved state is given by:
\begin{align*}
W_{\hat{\rho}}(\theta, \phi, \beta,\beta^{*}) = \sum_{n,m=0}^{\infty} & \Big[\frac{1}{2}(1+\sqrt{3}\cos\theta) C_n{C_m}^{*} {e^{-it(E_n-E_m)}}\cos(tg\sqrt{n+1})\cos(tg\sqrt{m+1}) \\
& +i\frac{\sqrt{3}}{2}e^{i\phi}\sin\theta C_n C_{m-1}^{*}{e^{-it(E_n-E_{m-1})}}\cos(tg\sqrt{n+1})\sin(tg\sqrt{m}) \\
& -i\frac{\sqrt{3}}{2}e^{-i\phi}\sin\theta C_{n-1}C_m^{*} {e^{-it(E_{n-1}-E_m)}}\sin(tg\sqrt{n})\cos(tg\sqrt{m+1}) \\
& +\frac{1}{2}(1-\sqrt{3}\cos\theta) C_{n-1}C_{m-1}^{*} {e^{-it(E_{n-1}-E_{m-1})}}\sin(tg\sqrt{n})\sin(tg\sqrt{m})\Big] \\
& \times \frac{2}{\pi}\frac{e^{-2|\beta|^{2}}}{\sqrt{n!m!}}L_{n,m}(2\beta,2\beta^{*})
\end{align*}
Here, $g$ is the atom-field coupling strength, $E_k = \omega(k+1/2)$ are the free energy eigenvalues, and $L_{n,m}(\gamma, \gamma^*)$ are the 2D Laguerre polynomials. To analyze the properties of the field subsystem alone, one can compute its reduced Wigner function, $W_{\hat{\rho}_f}(\beta, \beta^*)$, by tracing out the qubit degrees of freedom. In the phase space formalism, this operation corresponds to integrating the full Wigner function over the qubit's phase space with the appropriate measure:
$$W_{\hat{\rho}_{f}}(\beta,\beta^{*})=\frac{1}{2\pi}\int_{0}^{2\pi} d\phi \int_{0}^{\pi} d\theta \, \sin\theta \, W_{\hat{\rho}}(\theta,\phi,\beta,\beta^{*})$$

Task:
By performing the integration over the qubit's phase space coordinates $(\theta, \phi)$, derive the expression for the reduced Wigner function of the bosonic field, $W_{\hat{\rho}_{f}}(\beta,\beta^{*})$.

\subsubsection*{Ground Truth Solution}
\[ \boxed{W_{\hat{\rho}_{f}}(\beta,\beta^{*}) = \sum_{n,m=0}^{\infty}\Big[ C_n{C_m}^{*} {e^{-it(E_n-E_m)}}\cos(tg\sqrt{n+1})\cos(tg\sqrt{m+1}) + C_{n-1}C_{m-1}^{*} {e^{-it(E_{n-1}-E_{m-1})}}\sin(tg\sqrt{n})\sin(tg\sqrt{m})\Big] \times \frac{2}{\pi}\frac{e^{-2|\beta|^{2}}}{\sqrt{n!m!}}L_{n,m}(2\beta,2\beta^{*})} \]

\subsubsection*{Model Outputs}
\subsubsection*{Model: o4-mini (Score: 0.0)}
\paragraph*{Model Solution:}
\[ W_{\rho_f}(\beta,\beta^*)
=\frac{2}{\pi}\,e^{-2|\beta|^2}
\sum_{n,m=0}^\infty
\frac{e^{-i\omega t(n-m)}}{\sqrt{n!\,m!}}\,
L_{n,m}(2\beta,2\beta^*)
\Big[
C_nC_m^*\cos\!\big(gt\sqrt{n+1}\big)\cos\!\big(gt\sqrt{m+1}\big)
+
C_{n-1}C_{m-1}^*\sin\!\big(gt\sqrt{n}\big)\sin\!\big(gt\sqrt{m}\big)
\Big] \]

\paragraph*{Judge's Evaluation:}

First, let's rewrite the key parts of both expressions for clarity:

**Ground Truth:**
\[
W_{\hat{\rho}_{f}}(\beta,\beta^{*}) = \sum_{n,m=0}^{\infty} \Big[
C_n C_m^{*} e^{-i t (E_n - E_m)} \cos(t g \sqrt{n+1}) \cos(t g \sqrt{m+1}) 
+ C_{n-1} C_{m-1}^{*} e^{-i t (E_{n-1} - E_{m-1})} \sin(t g \sqrt{n}) \sin(t g \sqrt{m})
\Big] \times \frac{2}{\pi} \frac{e^{-2|\beta|^{2}}}{\sqrt{n! m!}} L_{n,m}(2\beta, 2\beta^{*})
\]

**Model-Generated:**
\[
W_{\rho_f}(\beta,\beta^*) = \frac{2}{\pi} e^{-2|\beta|^2} \sum_{n,m=0}^\infty \frac{e^{-i \omega t (n - m)}}{\sqrt{n! m!}} L_{n,m}(2\beta, 2\beta^*) \Big[
C_n C_m^* \cos(gt \sqrt{n+1}) \cos(gt \sqrt{m+1}) + C_{n-1} C_{m-1}^* \sin(gt \sqrt{n}) \sin(gt \sqrt{m})
\Big]
\]

---

**Comparison:**

1. **Prefactors:**
   - Both have the factor \(\frac{2}{\pi} e^{-2|\beta|^2}\) outside the sum. This matches exactly.

2. **Summation indices and factorial terms:**
   - Both sum over \(n,m=0\) to \(\infty\).
   - Both have the denominator \(\sqrt{n! m!}\).
   - Both multiply by \(L_{n,m}(2\beta, 2\beta^*)\).
   - These match exactly.

3. **Coefficients \(C_n\) and \(C_m^*\):**
   - Both have \(C_n C_m^*\) and \(C_{n-1} C_{m-1}^*\) in the same places.
   - Matches exactly.

4. **Trigonometric terms:**
   - Both have \(\cos(t g \sqrt{n+1}) \cos(t g \sqrt{m+1})\) and \(\sin(t g \sqrt{n}) \sin(t g \sqrt{m})\).
   - Matches exactly.

5. **Exponential phase factors:**
   - Ground truth has \(e^{-i t (E_n - E_m)}\) and \(e^{-i t (E_{n-1} - E_{m-1})}\).
   - Model has \(e^{-i \omega t (n - m)}\).
   
   This is the key difference.

   The ground truth uses energy eigenvalues \(E_n\), which presumably depend on \(n\) in some way, while the model uses a simplified phase factor \(e^{-i \omega t (n - m)}\).

   Unless \(E_n = \omega n\) exactly, these are not equivalent.

   Since the ground truth explicitly writes \(E_n\) and \(E_{n-1}\), and the model uses a single \(\omega\) times \(n\), this is a significant difference.

   Moreover, the ground truth has two different phase factors inside the sum: one for the cosine terms with \(E_n - E_m\), and one for the sine terms with \(E_{n-1} - E_{m-1}\).

   The model uses the same phase factor \(e^{-i \omega t (n - m)}\) for both terms.

---

**Conclusion:**

- The model expression is very close in form but replaces the energy-dependent phase factors \(e^{-i t (E_n - E_m)}\) and \(e^{-i t (E_{n-1} - E_{m-1})}\) with a single simplified phase factor \(e^{-i \omega t (n - m)}\).
- This is more than a minor difference because the energy eigenvalues \(E_n\) can be more complicated than a linear function \(\omega n\).
- Also, the model uses the same phase factor for both cosine and sine terms, whereas the ground truth uses different phase factors depending on \(n\) or \(n-1\).
- This is more than one minor error: it affects the core time evolution phases in the expression.

Therefore, the model expression is **incorrect** according to the rubric.

---

**Score: 0**

\newpage
\subsection*{Problem 22 (Paper: 2506.23447v1)}
\subsubsection*{Problem Statement}
Background:
In a theoretical model of information encoding, the length of a codeword for a continuous variable $x$ from the space $\mathcal{C} = (1, +\infty)$ is given by a function $\ell(x)$. The probability density of a codeword is $p(x)$. The optimal code length function $\ell(x)$ is one that minimizes the average codeword length, $\mathbb{E}[\ell] = \int_\mathcal{C} p(x) \ell(x) dx$, subject to the continuous form of the Kraft-McMillan inequality, $\int_\mathcal{C} \exp(-\ell(x)) dx \leq 1$. This constrained optimization problem can be analyzed using the Lagrangian formalism with a multiplier $\lambda > 0$:
$$ \mathcal{L}(\ell) = \int_\mathcal{C} p(x) \ell(x) dx - \lambda \left( 1 - \int_\mathcal{C} \exp(-\ell(x)) dx \right) $$
Furthermore, the system is assumed to possess a scale-invariance property. This property is captured by a renormalization rule which, for a fixed-point probability distribution $p(x)$, is expressed by the relation:
$$ p(\exp(x)) = p(x) \cdot \exp(-x) $$
where $\exp(x)$ is the exponential function. The process of calculating the length is defined recursively. This recursion terminates, contributing nothing further to the total length, when any iterated argument to the length function becomes less than or equal to 1.

Task:
First, find the optimal relationship between $p(x)$ and $\ell(x)$ by functionally minimizing the Lagrangian $\mathcal{L}(\ell)$. Then, use this relationship to translate the renormalization fixed-point condition on $p(x)$ into a recursive equation for the optimal, fixed-point code length function, which we denote $\ell_*(x)$. Finally, by recursively expanding this equation, determine the complete functional form of $\ell_*(x)$. Express your answer as a sum of iterated natural logarithms.

\subsubsection*{Ground Truth Solution}
\[ \boxed{\ell_*(x) = \log x + \log(\log x) + \log\bigl(\log(\log x)\bigr) + \cdots} \]

\subsubsection*{Model Outputs}
\subsubsection*{Model: o4-mini (Score: 1.0)}
\paragraph*{Model Solution:}
\[ \ell_*(x)
=\ln x \;+\;\ln\ln x\;+\;\ln\ln\ln x\;+\;\cdots \]

\paragraph*{Judge's Evaluation:}

The ground truth expression is:
\[
\ell_*(x) = \log x + \log(\log x) + \log\bigl(\log(\log x)\bigr) + \cdots
\]

The model-generated expression is:
\[
\ell_*(x) = \ln x + \ln \ln x + \ln \ln \ln x + \cdots
\]

Here, the only difference is the notation for the logarithm function: the ground truth uses $\log$, while the model uses $\ln$. In mathematics and physics, $\log$ often denotes the natural logarithm (logarithm base $e$), especially in higher mathematics and physics contexts, unless otherwise specified. The expressions are otherwise identical in structure and meaning.

Since the difference is only in notation and both represent the natural logarithm, the expressions are mathematically equivalent.

**Score: 1**

\newpage
\subsection*{Problem 23 (Paper: 2506.23447v1)}
\subsubsection*{Problem Statement}
Background:
Consider a prefix code for the set of positive integers $\mathbb{N} = \{1, 2, 3, \ldots\}$. Let $\beta(n) = \lfloor \log_2 n \rfloor + 1$ denote the number of bits in the standard binary representation of an integer $n$. The length of the codeword for an integer $n$, denoted $L(n)$, is defined by the following recursive relationship:
\begin{itemize}
    \item For $n=1$, the code length is $L(1) = 1$.
    \item For $n > 1$, the code length is given by the recurrence $L(n) = L(\beta(n)-1) + \beta(n)$.
\end{itemize}
The completeness of such a code can be investigated by examining its Kraft's sum, defined as $S = \sum_{n=1}^{\infty} 2^{-L(n)}$. To analyze this sum, we partition the set of positive integers $\mathbb{N}$ into disjoint subsets $I_k$ for $k=1, 2, 3, \ldots$, where each subset contains all integers that have the same binary length: $I_k = \{ n \in \mathbb{N} \mid \beta(n) = k \}$. Note that this implies $I_k = \{ n \in \mathbb{N} \mid 2^{k-1} \leq n \leq 2^k - 1\}$ for $k \geq 1$.

Task:
By first partitioning the total Kraft's sum $S$ into a sum over the sets $I_k$, and then evaluating the partial sums over each $I_k$, derive a linear algebraic equation that relates $S$ to itself.

\subsubsection*{Ground Truth Solution}
\[ \boxed{S = \frac{1+S}{2}} \]

\subsubsection*{Model Outputs}
\subsubsection*{Model: o4-mini (Score: 1.0)}
\paragraph*{Model Solution:}
\[ S=\frac{1+S}{2} \]

\paragraph*{Judge's Evaluation:}

The model-generated expression is 

\[
S = \frac{1+S}{2}
\]

which exactly matches the ground truth expression 

\[
S = \frac{1+S}{2}.
\]

There are no differences in variable names, coefficients, or structure. The expressions are mathematically equivalent and identical.

Score: 1

\newpage
\subsection*{Problem 24 (Paper: 2506.23451v1)}
\subsubsection*{Problem Statement}
Background:
Consider a one-dimensional continuous dynamical system for a variable $x(t)$ governed by the differential equation $\frac{dx}{dt} = F(x, \alpha)$, where the function $F$ is given by $F(x, \alpha) = f(x, \alpha) - g(x, \alpha)$. The functions $f$ and $g$ are positive and smooth, and $\alpha = (\alpha_1, \alpha_2)$ is a two-component positive bifurcation parameter. This system can be discretized using the tropical discretization scheme, resulting in the discrete map $x_{n+1} = F_\tau(x_n, \alpha)$, where $n$ is the iteration step and $\tau$ is the time interval. The map $F_\tau$ is defined as:
\begin{equation*}
    F_\tau(x, \alpha) = \frac{x+\tau f(x,\alpha)}{x+\tau g(x,\alpha)} x
\end{equation*}
Assume that the system exhibits a cusp bifurcation at a point $(\bar{x}, \bar{\alpha})$. At this nonhyperbolic fixed point, the following conditions for the continuous system are satisfied:
\begin{enumerate}
    \item $F(\bar{x}, \bar{\alpha}) = 0$, which implies $f(\bar{x}, \bar{\alpha}) = g(\bar{x}, \bar{\alpha})$.
    \item $\displaystyle\frac{\partial F(\bar{x}, \bar{\alpha})}{\partial x} = 0$.
\end{enumerate}
For convenience, we define an auxiliary function $Z_\tau(x, \alpha) = \frac{\tau x}{x + \tau f(x, \alpha)}$.

Partial derivative notation: For a function $H(x, \alpha_1, \alpha_2)$, we use the standard PDE notation:
$\frac{\partial H}{\partial x}$ denotes the first partial derivative of $H$ with respect to $x$, and $\frac{\partial^2 H}{\partial x \partial \alpha_i}$ denotes the mixed second derivative, first with respect to $x$, then $\alpha_i$.

Task:
At the cusp bifurcation point $(\bar{x}, \bar{\alpha})$, the discrete map $F_\tau$ also satisfies certain conditions. One of these involves a combination of its partial derivatives. Derive the expression for the quantity
\begin{equation*}
C(F_\tau) = \frac{\partial F_{\tau}(\bar x, \bar \alpha)}{\partial \alpha_1}\frac{\partial^2 F_{\tau}(\bar x, \bar \alpha)}{\partial x\partial \alpha_2}-\frac{\partial F_{\tau}(\bar x, \bar \alpha)}{\partial \alpha _2}\frac{\partial^2 F_{\tau}(\bar x, \bar \alpha)}{\partial x\partial \alpha_1}
\end{equation*}
in terms of the function $Z_\tau(\bar{x}, \bar{\alpha})$ and the derivatives of the original function $F(x, \alpha)$ evaluated at the bifurcation point.

\subsubsection*{Ground Truth Solution}
\[ \boxed{\begin{equation*}
C(F_{\tau}) = Z_{\tau}^{2}(\bar x, \bar \alpha) \left( \frac{\partial F(\bar x, \bar \alpha)}{\partial \alpha_1}\frac{\partial^2 F(\bar x, \bar \alpha)}{\partial x\partial \alpha_2}-\frac{\partial F(\bar x, \bar \alpha)}{\partial \alpha_2}\frac{\partial^2 F(\bar x, \bar \alpha)}{\partial x\partial \alpha_1} \right)
\end{equation*}} \]

\subsubsection*{Model Outputs}
\subsubsection*{Model: o4-mini (Score: 1.0)}
\paragraph*{Model Solution:}
\[ C(F_\tau)
\;=\;
Z_\tau(\bar x,\bar\alpha)^2
\;\bigl(
F_{\alpha_1}(\bar x,\bar\alpha)\,F_{x\alpha_2}(\bar x,\bar\alpha)
\;-\;
F_{\alpha_2}(\bar x,\bar\alpha)\,F_{x\alpha_1}(\bar x,\bar\alpha)
\bigr)\,. \]

\paragraph*{Judge's Evaluation:}

The Ground Truth Expression is:
\[
C(F_{\tau}) = Z_{\tau}^{2}(\bar x, \bar \alpha) \left( \frac{\partial F(\bar x, \bar \alpha)}{\partial \alpha_1}\frac{\partial^2 F(\bar x, \bar \alpha)}{\partial x\partial \alpha_2}-\frac{\partial F(\bar x, \bar \alpha)}{\partial \alpha_2}\frac{\partial^2 F(\bar x, \bar \alpha)}{\partial x\partial \alpha_1} \right)
\]

The Model-Generated Expression is:
\[
C(F_\tau) = Z_\tau(\bar x,\bar\alpha)^2 \bigl( F_{\alpha_1}(\bar x,\bar\alpha) F_{x\alpha_2}(\bar x,\bar\alpha) - F_{\alpha_2}(\bar x,\bar\alpha) F_{x\alpha_1}(\bar x,\bar\alpha) \bigr)
\]

**Comparison:**

- The notation $F_{\alpha_1}(\bar x,\bar\alpha)$ corresponds exactly to $\frac{\partial F(\bar x, \bar \alpha)}{\partial \alpha_1}$.
- Similarly, $F_{x\alpha_2}(\bar x,\bar\alpha)$ corresponds to $\frac{\partial^2 F(\bar x, \bar \alpha)}{\partial x \partial \alpha_2}$.
- The factor $Z_\tau(\bar x,\bar\alpha)^2$ matches $Z_{\tau}^2(\bar x, \bar \alpha)$.
- The structure of the expression inside the parentheses is identical, with the correct signs and terms.
- Variable names and notation differ only in style, but the mathematical meaning is the same.

**Conclusion:**

The model-generated expression is mathematically equivalent to the ground truth expression, differing only in notation style. There are no missing terms, incorrect signs, or coefficients.

Score: 1

---

Score: 1

\newpage
\subsection*{Problem 25 (Paper: 2506.23451v1)}
\subsubsection*{Problem Statement}
Background:
Consider a one-dimensional biochemical model described by the differential equation $\frac{dx}{dt} = p-qx+\frac{x^2}{1+x^2}$, where the state variable $x \ge 0$ and the parameters $p, q$ are positive. This continuous system can be related to a discrete-time map through a procedure known as tropical discretization. In the limit of an infinite time step, this procedure yields the following discrete map for the state variable $x_n$ at step $n$:
\[
x_{n+1}=\frac{p}{q}+\frac{x_n^2}{q(1+x_n^2)}.
\]
This discrete map can be further transformed into a max-plus dynamical system via the process of ultradiscretization. This involves the variable substitutions $x_n=e^{X_n/\varepsilon}$, $p=e^{P/\varepsilon}$, and $q=e^{Q/\varepsilon}$, where $\varepsilon$ is a small positive parameter. The ultradiscrete limit is then taken as $\varepsilon \to +0$. The fundamental identity for this limit is $\lim_{\varepsilon \to +0}\varepsilon\log(e^{A/\varepsilon}+e^{B/\varepsilon})=\max(A,B)$ for any real numbers $A$ and $B$. After deriving the max-plus equation for $X_{n+1}$, a final transformation of the state variable, $X_n \to -X_n$, is applied to the system to obtain a canonical form.

Task:
Apply the ultradiscretization procedure to the given discrete-time map for $x_{n+1}$. First, substitute the exponential variables and take the ultradiscrete limit $\varepsilon \to +0$ to find the max-plus equation for $X_{n+1}$. Then, apply the transformation $X_n \to -X_n$ to this resulting max-plus equation. Determine the final expression for the transformed state variable $X_{n+1}$ in terms of $X_n$, $P$, and $Q$.

Solution:

\subsubsection*{Ground Truth Solution}
\[ \boxed{X_{n+1}  =  Q-\max(P,-\max(0,2X_n))} \]

\subsubsection*{Model Outputs}
\subsubsection*{Model: o4-mini (Score: N/A)}
\paragraph*{Model Solution:}
\[ X_{n+1}\;=\;\min\!\bigl(Q-P,\;Q+2\,X_n\bigr)\,. \]

\paragraph*{Judge's Evaluation:}

The ground truth expression is:
\[
X_{n+1} = Q - \max\bigl(P, -\max(0, 2X_n)\bigr)
\]

The model-generated expression is:
\[
X_{n+1} = \min\bigl(Q - P,\, Q + 2X_n\bigr)
\]

---

**Step 1: Analyze the ground truth expression**

Inside the outer max, we have two terms: \(P\) and \(-\max(0, 2X_n)\).

- \(\max(0, 2X_n)\) is always non-negative.
- So \(-\max(0, 2X_n)\) is always less than or equal to zero.
- Therefore, \(\max(P, -\max(0, 2X_n))\) compares \(P\) and a non-positive number.

Hence,
\[
\max(P, -\max(0, 2X_n)) = \begin{cases}
P & \text{if } P \geq -\max(0, 2X_n) \\
-\max(0, 2X_n) & \text{otherwise}
\end{cases}
\]

Then,
\[
X_{n+1} = Q - \max(P, -\max(0, 2X_n))
\]

---

**Step 2: Analyze the model expression**

\[
X_{n+1} = \min(Q - P, Q + 2X_n)
\]

---

**Step 3: Check equivalence**

Try to rewrite the ground truth expression in terms of min:

\[
X_{n+1} = Q - \max(P, -\max(0, 2X_n)) = \min(Q - P, Q + \max(0, 2X_n))
\]

Because:
\[
Q - \max(a,b) = \min(Q - a, Q - b)
\]

Applying this to \(a = P\), \(b = -\max(0, 2X_n)\):
\[
X_{n+1} = \min(Q - P, Q - (-\max(0, 2X_n))) = \min(Q - P, Q + \max(0, 2X_n))
\]

The model expression is:
\[
\min(Q - P, Q + 2X_n)
\]

The difference is that the model uses \(2X_n\) directly, while the ground truth uses \(\max(0, 2X_n)\).

---

**Step 4: Implication of the difference**

- If \(2X_n \geq 0\), then \(\max(0, 2X_n) = 2X_n\), so the expressions match.
- If \(2X_n < 0\), then \(\max(0, 2X_n) = 0\), so the ground truth uses \(Q + 0 = Q\), but the model uses \(Q + 2X_n < Q\).

Therefore, the model expression is incorrect for negative \(X_n\) values.

---

**Conclusion:**

The model expression is close but misses the \(\max(0, \cdot)\) function inside the second argument of the \(\min\). This is a significant difference because it changes the behavior for negative \(X_n\).

This is more than a minor error (not just a missing coefficient or sign), so partial credit is not warranted.

---

**Final score:**

\[
\boxed{
\text{Score: } 0
}
\]

\newpage
\subsection*{Problem 26 (Paper: 2506.23496v1)}
\subsubsection*{Problem Statement}
Background:
Consider a reversible, detailed-balanced Chemical Reaction Network (CRN). The state of the system is described by a vector of species concentrations $q \in \mathbb{R}^{\mathcal{S}}_{\ge 0}$. The network consists of a set of reactions $\mathcal{R}$. For each reaction $r \in \mathcal{R}$, the forward and reverse reaction fluxes are given by mass-action kinetics: $J_r(q) = k_r q^{r^-}$ and $J_r^*(q) = k_r^* q^{r^+}$, where $r^-$ and $r^+$ are the stoichiometric vectors of reactants and products, respectively, and the notation $q^{v}$ denotes $\prod_s q_s^{v_s}$. The system is detailed-balanced, meaning there exists an equilibrium concentration vector $\eq$ at which the forward and reverse fluxes for every reaction are equal: $k_r \eq^{r^-} = k_r^* \eq^{r^+} =: \Phi_r$. The quantity $\Phi_r$ is the detailed-balanced flux for reaction $r$. A reaction $r$ is said to be "blocked" if its net flux is constrained to be zero. The minimal log-improbability cost rate to maintain this blocked state is given by $\dot{\delta}(r,q) = (\sqrt{J_r(q)}-\sqrt{J_r^*(q)})^2$. We analyze the system in the linear response regime, where concentrations $q$ are close to $\eq$. In this regime, the net species current $\mathcal{I} \in \mathbb{R}^{\mathcal{S}}$ is related to a species potential vector $\mathcal{V}(q) = (q/\eq - \mathbf{e})$ by a generalized Ohm's law: $\mathcal{V}(q) = -\mathbb{R}\mathcal{I}$. Here, the division is element-wise, $\mathbf{e}$ is a vector of all ones, and $\mathbb{R}$ is the symmetric, positive semi-definite resistance matrix of the network. The net change in species due to a single firing of reaction $r$ is given by its stoichiometric vector $\Delta r = r^+ - r^-$. 

Task:
For a detailed-balanced CRN in the linear response regime, derive an expression for the blocking cost rate $\dot{\delta}(r,q)$ of a single reaction $r$. Your final expression should be in terms of the species current $\mathcal{I}$, the resistance matrix $\mathbb{R}$, the detailed-balanced flux $\Phi_r$, and the stoichiometric vector $\Delta r$.

\subsubsection*{Ground Truth Solution}
\[ \boxed{\dot{\delta}(r,q) = \mathcal{I}^T \mathbb{R}  (\Delta r)\frac{\Phi_r}{4} (\Delta r)^T \mathbb{R} \mathcal{I}} \]

\subsubsection*{Model Outputs}
\subsubsection*{Model: o4-mini (Score: 1.0)}
\paragraph*{Model Solution:}
\[ \dot\delta(r,q)\;=\;\frac{\Phi_r}{4}\,\bigl(\Delta r^{T}R\,\mathcal I\bigr)^{2}. \]

\paragraph*{Judge's Evaluation:}

The ground truth expression is:
\[
\dot{\delta}(r,q) = \mathcal{I}^T \mathbb{R} (\Delta r) \frac{\Phi_r}{4} (\Delta r)^T \mathbb{R} \mathcal{I}
\]

The model-generated expression is:
\[
\dot{\delta}(r,q) = \frac{\Phi_r}{4} \left(\Delta r^{T} R \mathcal{I}\right)^2
\]

---

**Step 1: Analyze the ground truth expression**

The ground truth can be rewritten as:
\[
\dot{\delta} = \mathcal{I}^T \mathbb{R} (\Delta r) \frac{\Phi_r}{4} (\Delta r)^T \mathbb{R} \mathcal{I}
\]

Note that $\mathbb{R}$ is a matrix, $\Delta r$ is a vector, and $\mathcal{I}$ is a vector.

The expression is a scalar because it is a product of vectors and matrices resulting in a scalar.

---

**Step 2: Analyze the model expression**

The model expression is:
\[
\dot{\delta} = \frac{\Phi_r}{4} \left(\Delta r^{T} R \mathcal{I}\right)^2
\]

Here, $\Delta r^{T} R \mathcal{I}$ is a scalar (since $\Delta r^{T}$ is $1 \times n$, $R$ is $n \times n$, and $\mathcal{I}$ is $n \times 1$), and then squared.

---

**Step 3: Compare the two expressions**

- The ground truth expression is:
\[
\dot{\delta} = \mathcal{I}^T \mathbb{R} (\Delta r) \frac{\Phi_r}{4} (\Delta r)^T \mathbb{R} \mathcal{I}
\]

- The model expression is:
\[
\dot{\delta} = \frac{\Phi_r}{4} \left(\Delta r^{T} R \mathcal{I}\right)^2
\]

---

**Step 4: Are these expressions equivalent?**

Note that:
\[
\left(\Delta r^{T} R \mathcal{I}\right)^2 = \left(\Delta r^{T} R \mathcal{I}\right) \left(\Delta r^{T} R \mathcal{I}\right)
\]

But the ground truth expression is:
\[
\mathcal{I}^T R \Delta r \cdot \frac{\Phi_r}{4} \cdot (\Delta r)^T R \mathcal{I}
\]

Since scalars commute, and the transpose of a scalar is itself, we can check if:
\[
\mathcal{I}^T R \Delta r = (\Delta r)^T R \mathcal{I}
\]

Because:
\[
(\mathcal{I}^T R \Delta r)^T = (\Delta r)^T R^T \mathcal{I}
\]

If $R$ is symmetric, i.e., $R = R^T$, then:
\[
\mathcal{I}^T R \Delta r = (\Delta r)^T R \mathcal{I}
\]

Therefore, the ground truth expression is:
\[
\dot{\delta} = \frac{\Phi_r}{4} \left(\mathcal{I}^T R \Delta r\right) \left(\Delta r^T R \mathcal{I}\right) = \frac{\Phi_r}{4} \left(\Delta r^T R \mathcal{I}\right)^2
\]

which matches the model expression exactly.

---

**Step 5: Conclusion**

The model-generated expression is mathematically equivalent to the ground truth expression, assuming $R$ is symmetric (which is typical for such matrices in physics).

---

**Score justification:**

- The model expression matches the ground truth expression exactly in mathematical content.
- The difference is only in the form of presentation (factoring the scalar product as a square).
- No errors or missing terms are present.

---

**Score:**

\[
\boxed{1}
\]

Score: 1

\newpage
\subsection*{Problem 27 (Paper: 2506.23496v1)}
\subsubsection*{Problem Statement}
Background:
Consider a detailed-balanced, reversible Chemical Reaction Network (CRN) denoted by the graph $\mathcal{G}' = (\mathcal{S}, \mathcal{R}')$, which sustains a constant species throughput current $\mathcal{I}$. The network operates in the linear response regime, where its non-equilibrium steady state (NESS) concentration vector $q$ is close to the detailed-balanced equilibrium concentration vector $q^{\rm eq}$. The system's behavior can be described by an analogy to Ohm's law. The species potential, defined as $\mathcal{V}(q) = \frac{q}{q^{\rm eq}} - \mathbf{e}$ (where $\mathbf{e}$ is the vector of all ones), is related to the current by $\mathcal{V}(q) = -\mathbb{R} \mathcal{I}$. Here, $\mathbb{R}$ is the symmetric resistance matrix of the active network $\mathcal{G}'$, which is the Moore-Penrose pseudoinverse of the network's conductance matrix $\mathbb{C}(\mathcal{G}') = \frac{1}{2}\sum_{r\in \mathcal{R}'} \Phi_r (\Delta r)(\Delta r)^T$. The term $\Phi_r = k_r (q^{\rm eq})^{r^-}$ is the detailed balanced flux for a reaction $r$, and $\Delta r = r^+ - r^-$ is its stoichiometric vector.

Now, consider an additional reversible reaction, $r_b$, which is not part of the active pathway $\mathcal{G}'$. This reaction is kept "blocked," meaning its net flux is constrained to be zero. The thermodynamic cost associated with maintaining this block, known as the blocking cost, is evaluated at the NESS concentration $q$ established by the active pathway $\mathcal{G}'$. The blocking cost is given by the expression:
$$ \dot{\delta}(r_b, q) = \Phi_{r_b} \left( \left(\frac{q}{q^{\rm eq}}\right)^{r_b^-/2} - \left(\frac{q}{q^{\rm eq}}\right)^{r_b^+/2} \right)^2 $$
where $\Phi_{r_b}$ is the detailed balanced flux of reaction $r_b$, and $r_b^-$ and $r_b^+$ are its reactant and product stoichiometry vectors, respectively.

Task:
For the system described, derive the expression for the blocking cost $\dot{\delta}(r_b, q)$ in the linear response regime. Your final answer should be expressed in terms of the throughput current $\mathcal{I}$, the resistance matrix of the active pathway $\mathbb{R}$, the detailed balanced flux $\Phi_{r_b}$, and the stoichiometric vector $\Delta r_b$ of the blocked reaction.

Solution:

\subsubsection*{Ground Truth Solution}
\[ \boxed{$\dot{\delta}(r_b, q) = \mathcal{I}^T \mathbb{R} (\Delta r_b) \frac{\Phi_{r_b}}{4} (\Delta r_b)^T \mathbb{R} \mathcal{I}$} \]

\subsubsection*{Model Outputs}
\subsubsection*{Model: o4-mini (Score: 1.0)}
\paragraph*{Model Solution:}
\[ \dot{\delta}(r_b,q)\;\approx\;\frac{\Phi_{r_b}}4\;\bigl(\Delta r_b^T\,\mathbb R\,\mathcal I\bigr)^2\,. \]

\paragraph*{Judge's Evaluation:}

The ground truth expression is:
\[
\dot{\delta}(r_b, q) = \mathcal{I}^T \mathbb{R} (\Delta r_b) \frac{\Phi_{r_b}}{4} (\Delta r_b)^T \mathbb{R} \mathcal{I}
\]

The model-generated expression is:
\[
\dot{\delta}(r_b,q) \approx \frac{\Phi_{r_b}}{4} \left(\Delta r_b^T \mathbb{R} \mathcal{I}\right)^2
\]

---

**Step-by-step comparison:**

1. **Structure of the expression:**

- The ground truth has the form:
\[
\mathcal{I}^T \mathbb{R} (\Delta r_b) \cdot \frac{\Phi_{r_b}}{4} \cdot (\Delta r_b)^T \mathbb{R} \mathcal{I}
\]

- The model expression is:
\[
\frac{\Phi_{r_b}}{4} \left(\Delta r_b^T \mathbb{R} \mathcal{I}\right)^2
\]

2. **Matrix/vector dimensions and multiplication:**

- In the ground truth, the term \(\mathcal{I}^T \mathbb{R} (\Delta r_b)\) is a scalar because \(\mathcal{I}^T\) is a row vector, \(\mathbb{R}\) is a matrix, and \(\Delta r_b\) is a vector. Similarly, \((\Delta r_b)^T \mathbb{R} \mathcal{I}\) is also a scalar.

- The product of these two scalars is the product of the same scalar and its transpose (or the same scalar squared), assuming the vectors and matrices are real-valued.

3. **Equivalence:**

- Note that:
\[
\mathcal{I}^T \mathbb{R} (\Delta r_b) = \left( (\Delta r_b)^T \mathbb{R} \mathcal{I} \right)^T
\]
Since the transpose of a scalar is the scalar itself, these two terms are equal.

- Therefore, the ground truth expression can be rewritten as:
\[
\dot{\delta}(r_b, q) = \frac{\Phi_{r_b}}{4} \left( \mathcal{I}^T \mathbb{R} (\Delta r_b) \right) \left( (\Delta r_b)^T \mathbb{R} \mathcal{I} \right) = \frac{\Phi_{r_b}}{4} \left( (\Delta r_b)^T \mathbb{R} \mathcal{I} \right)^2
\]

- This matches exactly the model-generated expression.

4. **Approximation sign:**

- The model uses "\(\approx\)" instead of equality. This is a minor notation difference and does not affect the mathematical equivalence.

---

**Conclusion:**

The model-generated expression is mathematically equivalent to the ground truth expression, differing only in notation and order of multiplication, which is commutative for scalars.

---

**Score:**

Score: 1

\newpage
\subsection*{Problem 28 (Paper: 2506.23600v1)}
\subsubsection*{Problem Statement}
Background:
A heavy point particle of mass $m$ is confined by a harmonic potential with frequency $\omega$ and coupled to a thermal environment at temperature $T$. The coupling is characterized by a relaxation rate $\gamma$. The dynamics of the particle's reduced density matrix, $\hat{\rho}(T)$, is governed by the Caldeira-Leggett master equation (with $\hbar=1$):
$$ \partial_t \hat\rho(T) = -i [\hat H, \hat\rho] - i\gamma [\hat x,\{\hat p,\hat\rho\}]-2\gamma m T[ \hat x,[\hat x,\hat \rho]] $$
where $\hat{x}$ and $\hat{p}$ are the position and momentum operators satisfying $[\hat{x}, \hat{p}] = i$, and the Hamiltonian is $\hat H = b\hat{p}^2 + c\hat{x}^2$ with $b=1/(2m)$ and $c=m\omega^2/2$. The Symmetric Logarithmic Derivative (SLD), $\hat{\Lambda}_T$, is an observable that quantifies the ultimate sensitivity to the parameter $T$ and is implicitly defined by:
$$ \partial_T \hat \rho(T) = \frac{1}{2} \left( \hat\Lambda_T \hat\rho(T) + \hat\rho(T) \hat\Lambda_T \right) - \hat\rho(T)\langle \hat \Lambda_T \rangle $$
The SLD can be found by enforcing the consistency condition $\partial_T(\partial_t \hat{\rho}) = \partial_t(\partial_T \hat{\rho})$. Consider an operator basis expansion for the SLD of the form $\hat \Lambda_T = \sum_i c^{(i)}_T \hat A_i$, where the basis operators are $\hat A_i \in \{ \hat x, \hat p, \hat x^2, \hat p^2, \{\hat x,\hat p\} \}$. At late times, the system reaches a steady Gaussian state characterized by the following non-zero expectation values:
$$ \langle \hat{x}^2 \rangle = \frac{a^2 b}{4c}, \quad \langle \hat{p}^2 \rangle = \frac{a^2}{4} $$
where $a=\sqrt{4mT}$. For this Gaussian state, expectation values of higher powers factorize according to Wick's theorem, leading to $\langle \hat{x}^4 \rangle = 3 \langle \hat{x}^2 \rangle^2$ and $\langle \hat{p}^4 \rangle = 3 \langle \hat{p}^2 \rangle^2$. All expectation values involving an odd number of $\hat{x}$ or $\hat{p}$ operators are zero.

Task:
By setting up and solving the system of linear equations for the coefficients $c^{(i)}_T$ in the steady-state limit, determine the coefficient $c^{(x^2)}_T$ corresponding to the $\hat{x}^2$ operator. Express your result in terms of $a, b, c,$ and $\gamma$.

\subsubsection*{Ground Truth Solution}
\[ \boxed{c^{(x^2)}_T = \frac{4 c \left( a^4 b^2 + 4 (b c + \gamma^2) \right)}{a^8 b^4 - 16 b^2 c^2}} \]

\subsubsection*{Model Outputs}
\subsubsection*{Model: o4-mini (Score: 0.0)}
\paragraph*{Model Solution:}
\[ c^{(x^2)}_T \;=\;\frac{8\,c}{b^2\,a^4} \]

\paragraph*{Judge's Evaluation:}

The ground truth expression is:
\[
c^{(x^2)}_T = \frac{4 c \left( a^4 b^2 + 4 (b c + \gamma^2) \right)}{a^8 b^4 - 16 b^2 c^2}
\]

The model-generated expression is:
\[
c^{(x^2)}_T = \frac{8 c}{b^2 a^4}
\]

**Comparison:**

- The ground truth expression is a complex rational function involving terms \(a^4 b^2\), \(b c\), \(\gamma^2\) in the numerator, and \(a^8 b^4\), \(b^2 c^2\) in the denominator.
- The model expression is a much simpler fraction with numerator \(8 c\) and denominator \(b^2 a^4\).
- The model expression does not contain the terms \(4 (b c + \gamma^2)\) in the numerator.
- The denominator in the model expression is \(b^2 a^4\), which is not equivalent to \(a^8 b^4 - 16 b^2 c^2\).
- The factor 4 in the numerator of the ground truth is replaced by 8 in the model, which is a discrepancy.
- Overall, the model expression is not mathematically equivalent to the ground truth expression.

Since the model expression is missing multiple terms and factors and is not equivalent, it is incorrect.

**Score: 0**

\newpage
\subsection*{Problem 29 (Paper: 2506.23600v1)}
\subsubsection*{Problem Statement}
Background:
Consider a heavy point particle of mass $m$ confined in a one-dimensional harmonic potential, described by the Hamiltonian $\hat{H} = b\hat{p}^2 + c\hat{x}^2$, where $b=1/(2m)$ and $c=m\omega^2/2$. The particle is coupled to a thermal environment at temperature $T$ with a relaxation rate $\gamma$. The evolution of the particle's reduced density matrix $\hat{\rho}$ is governed by the Caldeira-Leggett master equation:
\begin{equation*}
\partial_t \hat\rho = -i [\hat H, \hat\rho] - i\gamma [\hat x,\{\hat p,\hat\rho\}]-\frac{\gamma T}{b}[ \hat x,[\hat x,\hat \rho]]
\end{equation*}
where $[\cdot,\cdot]$ and $\{\cdot,\cdot\}$ are the commutator and anticommutator, respectively, and we set $\hbar=1$ such that $[\hat{x}, \hat{p}] = i$. The Symmetric Logarithmic Derivative (SLD) for temperature, $\hat{\Lambda}_T$, is an operator that quantifies the sensitivity of the state to changes in $T$ and is implicitly defined by the relation $\partial_T \hat \rho = \frac{1}{2} ( \hat\Lambda_T \hat\rho + \hat\rho \hat\Lambda_T ) - \hat\rho\langle \hat \Lambda_T \rangle$. We expand the SLD in a basis of Hermitian operators $\hat{A}_i$: $\hat{\Lambda}_T = \sum_i c_T^{(i)} \hat{A}_i$. The coefficients $c_T^{(i)}$ can be found by solving the linear system $\sum_i M_{ji} c_T^{(i)} = D_j$, where the matrix elements $M_{ji}$ are given by:
\begin{align*}
M_{ji} = &-\tfrac{1}{2} \left\langle \left\{ \hat{A}_i , \left[ \hat{H}, \hat{A}_j \right] \right\}\right\rangle 
- \tfrac{i\gamma}{2} \left\langle \left\{ \hat{A}_i, \left\{ \hat{p} ,\left[ \hat{A}_j, \hat{x} \right] \right\} \right\} \right\rangle
- \tfrac{\gamma T}{2 b} \left\langle \left\{ \hat{A}_i, \left[ \hat{x}, \left[ \hat{x}, \hat{A}_j \right] \right] \right\} \right\rangle \\
& + \tfrac{i \gamma}{2} \left\langle \left\{ \hat{p}, \left[ \left\{ \hat{A}_i, \hat{A}_j\right\}, \hat{x} \right] \right\}\right\rangle
+ \tfrac{\gamma T}{2 b} \left\langle \left[ \hat{x}, \left[ \hat x, \left\{ \hat{A}_i, \hat{A}_j \right\}\right] \right] \right\rangle 
- i \left \langle \hat{A}_j \right\rangle \left\langle \left[ \hat{A}_i, \hat{H} \right] \right\rangle \\
& - i \gamma \left\langle \hat{A}_j \right\rangle \left\langle \left\{ \hat{p}, \left[ \hat{A}_i, \hat{x} \right] \right\} \right\rangle 
- \tfrac{\gamma T}{b} \left\langle \hat{A}_j \right\rangle \left\langle \left[ \hat x, \left[ \hat{x}, \hat{A}_i \right] \right] \right\rangle
\end{align*}
The system is in a non-equilibrium squeezed Gaussian state characterized by $\langle \hat{x} \rangle = \langle \hat{p} \rangle = 0$. For such a state, expectation values of products of operators can be simplified using Wick's theorem. The relevant relations for Weyl-ordered operators $W(\hat{x}^n\hat{p}^m)=\frac{1}{2^n}\sum_{k=0}^n\binom{n}{k}\hat{x}^{n-k}\hat{p}^m\hat{x}^k$ are:
\begin{align*}
\langle W(x^3 p) \rangle &= \frac{3}{2} \langle x^2 \rangle \langle \{ \hat x,\hat p\} \rangle \\
\langle W(x p^3) \rangle &= \frac{3}{2} \langle p^2 \rangle \langle \{ \hat x,\hat p\} \rangle \\
\langle W(x^2p^2) \rangle &= \langle x^2\rangle \langle p^2\rangle + \frac{1}{2} \langle \{ \hat x,\hat p\} \rangle^2
\end{align*}
Also, for any Gaussian state with zero mean, $\langle \hat{O}_1 \hat{O}_2 \dots \hat{O}_{2k+1} \rangle = 0$ for operators $\hat{O}_i \in \{\hat{x}, \hat{p}\}$.

Task:
Using the operator basis $\{\hat{x}, \hat{p}, \hat{x}^2, \hat{p}^2, \{\hat{x},\hat{p}\}\}$, derive the matrix element $M_{\{\hat{x},\hat{p}\}, \{\hat{x},\hat{p}\}}$. Express your final answer in terms of the expectation values $\langle \hat{x}^2 \rangle$, $\langle \hat{p}^2 \rangle$, $\langle \{\hat{x},\hat{p}\} \rangle$ and the parameters $b, c, \gamma, T$.

\subsubsection*{Ground Truth Solution}
\[ \boxed{8 \gamma \langle x^2 \rangle \langle p^2 \rangle - \frac{8 \gamma T}{b} \langle x^2 \rangle - 2 \gamma + 8 c \langle x^2 \rangle \langle \{x, p\} \rangle - 8 b \langle p^2 \rangle \langle \{x, p\} \rangle + 2 \gamma \langle \{x, p\} \rangle^2} \]

\subsubsection*{Model Outputs}
\subsubsection*{Model: o4-mini (Score: 0.0)}
\paragraph*{Model Solution:}
\[ M_{\{x,p\},\{x,p\}}
=-\,12\,\langle\{x,p\}\rangle\;\bigl(c\,\langle x^2\rangle \;-\;b\,\langle p^2\rangle\bigr)\,. \]

\paragraph*{Judge's Evaluation:}

The ground truth expression is:
\[
8 \gamma \langle x^2 \rangle \langle p^2 \rangle - \frac{8 \gamma T}{b} \langle x^2 \rangle - 2 \gamma + 8 c \langle x^2 \rangle \langle \{x, p\} \rangle - 8 b \langle p^2 \rangle \langle \{x, p\} \rangle + 2 \gamma \langle \{x, p\} \rangle^2
\]

The model-generated expression is:
\[
M_{\{x,p\},\{x,p\}} = -12 \langle \{x,p\} \rangle \left( c \langle x^2 \rangle - b \langle p^2 \rangle \right)
\]

**Comparison and Justification:**

1. The ground truth expression is a sum of six terms involving $\gamma$, $T$, $b$, $c$, and expectation values of $x^2$, $p^2$, and the anticommutator $\{x,p\}$. It includes constant terms and quadratic terms in the expectation values.

2. The model expression is a single product term involving $\langle \{x,p\} \rangle$ multiplied by a linear combination of $c \langle x^2 \rangle$ and $b \langle p^2 \rangle$, with a coefficient $-12$.

3. The model expression does not contain any terms involving $\gamma$ or $T$, which are present in the ground truth.

4. The model expression is structurally different: it is a factorized form involving only $\langle \{x,p\} \rangle$ and a difference of terms, whereas the ground truth is a sum of multiple terms with different dependencies.

5. There is no obvious algebraic manipulation or factorization that can transform the ground truth into the model expression or vice versa.

6. The model expression is missing multiple terms and constants, and the coefficients differ significantly.

**Conclusion:**

The model expression is not mathematically equivalent to the ground truth expression. It misses multiple terms and constants, and the structure is different. Therefore, it is incorrect.

Score: 0

\newpage
\subsection*{Problem 30 (Paper: 2506.23604v1)}
\subsubsection*{Problem Statement}
Background:
Consider a composite, isolated system described by a total Hamiltonian $H_{\text{tot}}(x, y, \lambda)$, where $x$ represents the state variables of a small system of interest, $y$ represents the state variables of a much larger thermal bath, and $\lambda$ is an external control parameter. A key step in formulating a consistent thermodynamic description is to decompose the total Hamiltonian as $H_{\text{tot}} = H_X(x; \lambda) + H_Y(y; x, \lambda)$. A particularly useful decomposition is one where the bath energy becomes an adiabatic invariant, meaning it does not change under slow variations of the system state $x$ and the control parameter $\lambda$. This is equivalent to requiring that the bath's Boltzmann entropy, $S_Y(E_Y) = \log \Omega_Y(E_Y)$, is independent of $x$ and $\lambda$, where $\Omega_Y(E_Y; x, \lambda) = \int dy\, \delta(E_Y - H_Y(y; x, \lambda))$ is the microcanonical partition function for the bath.

Assume you are given an arbitrary initial decomposition of the total Hamiltonian, $H_{\text{tot}}(x, y, \lambda) = H^0_X(x; \lambda) + H^0_Y(y; x, \lambda)$. The bath entropy corresponding to this initial decomposition, $S^0_Y(E_Y; x, \lambda) = \log \int dy\, \delta(E_Y - H^0_Y(y; x, \lambda))$, generally depends on both $x$ and $\lambda$. The bath temperature is defined as $T = 1/\beta = \left( \partial S^0_Y / \partial E_Y \right)^{-1}$. Because the bath is large, its temperature $T$ is a bulk property insensitive to local changes in $x$ and $\lambda$. Furthermore, the change in $S^0_Y$ due to variations in $x$ and $\lambda$ is sub-extensive compared to the total bath energy $E_Y$.

Task:
Your goal is to find a new decomposition, $H_{\text{tot}} = H_X(x; \lambda) + H_Y(y; x, \lambda)$, by modifying the original Hamiltonians. The new bath Hamiltonian $H_Y$ must be constructed such that its corresponding Boltzmann entropy, $S_Y(E_Y)$, is independent of $x$ and $\lambda$ to first order in the correction term. Determine the expression for this new, adiabatically invariant bath Hamiltonian, $H_Y(y; x, \lambda, E_Y)$. Express your result in terms of the original bath Hamiltonian $H^0_Y(y; x, \lambda)$, its associated entropy $S^0_Y(E_Y; x, \lambda)$, the bath temperature $T$, and an arbitrary but fixed reference state $(x_0, \lambda_0)$.

\subsubsection*{Ground Truth Solution}
\[ \boxed{H_Y(y; x,\lambda,E_Y) = H^0_Y(y; x,\lambda) + T \left( S^0_Y(E_Y; x,\lambda) - S^0_Y(E_Y; x_0,\lambda_0) \right)} \]

\subsubsection*{Model Outputs}
\subsubsection*{Model: o4-mini (Score: 1.0)}
\paragraph*{Model Solution:}
\[ H_Y\bigl(y;\,x,\lambda,E_Y\bigr)
\;=\;
H_Y^0\bigl(y;\,x,\lambda\bigr)
\;+\;
T\,
\Bigl[
   S_Y^0\bigl(E_Y;\,x,\lambda\bigr)
   \;-\;
   S_Y^0\bigl(E_Y;\,x_0,\lambda_0\bigr)
\Bigr] \]

\paragraph*{Judge's Evaluation:}

The model-generated expression is:

\[
H_Y\bigl(y;\,x,\lambda,E_Y\bigr)
= H_Y^0\bigl(y;\,x,\lambda\bigr)
+ T \left[
   S_Y^0\bigl(E_Y;\,x,\lambda\bigr)
   - S_Y^0\bigl(E_Y;\,x_0,\lambda_0\bigr)
\right]
\]

The ground truth expression is:

\[
H_Y(y; x,\lambda,E_Y) = H^0_Y(y; x,\lambda) + T \left( S^0_Y(E_Y; x,\lambda) - S^0_Y(E_Y; x_0,\lambda_0) \right)
\]

Comparing the two:

- The order and placement of variables inside the functions match exactly.
- The superscripts and subscripts are consistent.
- The parentheses and brackets are equivalent.
- The only difference is stylistic: the model uses \verb|\bigl| and \verb|\bigr| for parentheses and spacing commands, which do not affect mathematical equivalence.
- Variable names and function arguments are identical.
- The overall structure and terms are exactly the same.

Therefore, the model-generated expression is mathematically equivalent and matches the ground truth expression exactly.

Score: 1

\newpage
\subsection*{Problem 31 (Paper: 2506.23604v1)}
\subsubsection*{Problem Statement}
Background:
Consider an isolated composite system comprised of a small system of interest, with phase space coordinates $x=(q_x, p_x)$, and a much larger thermal bath, with coordinates $y=(q_y, p_y)$. The total Hamiltonian depends on a time-varying external control parameter $\lambda$ and is decomposed as $H_{\text{tot}}(x, y, \lambda) = H_X(x; \lambda) + H_Y(y; x, \lambda)$. This decomposition is specifically chosen such that the bath Hamiltonian $H_Y$ is \textit{adiabatically invariant}: its microcanonical density of states, $\Omega_Y(E_Y) = \int dy\, \delta(E_Y - H_Y(y; x, \lambda))$, and its Boltzmann entropy, $S_Y(E_Y) = \log \Omega_Y(E_Y)$, are independent of the system state $x$ and the control parameter $\lambda$. The inverse temperature of the bath is defined as $\beta = \partial S_Y / \partial E_Y$. The dynamics is time-reversal symmetric, meaning $H_X(x; \lambda) = H_X(x^*; \lambda^*)$ and $H_Y(y; x, \lambda) = H_Y(y^*; x^*, \lambda^*)$, where $c^*$ denotes the time-reversal of a state or parameter $c$.

A forward process is defined by the evolution from time $s=0$ to $s=t$ under a protocol $\lambda_s$. The corresponding backward process evolves under the protocol $\tilde{\lambda}_s = \lambda_{t-s}^*$. The coarse-grained transition probability density for the joint system to evolve from a state with system coordinate $x$ and bath energy $E_Y$ to one with $x'$ and $E'_Y$ satisfies a generalized detailed balance (GDB) relation: $\left\langle x', E_Y' | x, E_Y \right\rangle_F = \left\langle x^*, E_Y | x'^*, E_Y' \right\rangle_B$.

The composite system is initially prepared in a microcanonical ensemble with a fixed total energy $E_{XY}$ and control parameter $\lambda_0$. The initial probability density is $p_{XY}^0(x_0, y_0) = \delta(H_{\text{tot}}(x_0, y_0, \lambda_0) - E_{XY}) / \Omega_{XY}(E_{XY}, \lambda_0)$, where $\Omega_{XY}(E, \lambda) = \iint dx dy \, \delta(H_{\text{tot}}(x, y, \lambda) - E)$ is the total microcanonical partition function. During the forward process, the work done on the system is $W = H_{\text{tot}}(x_t, y_t, \lambda_t) - E_{XY}$, and the heat dissipated to the bath is $Q = H_Y(y_0; x_0, \lambda_0) - H_Y(y_t; x_t, \lambda_t)$. The joint probability density of observing work $W$, heat $Q$, final state $x_t$, and initial state $x_0$ is $p_F(W, Q, x_t, x_0)$. The corresponding density for the backward process, $p_B(-W, -Q, x_0^*, x_t^*)$, is defined for a process starting from a microcanonical ensemble with total energy $E_{XY}+W$ and control parameter $\lambda_t$. Assume that the system energy $H_X$ is sub-extensive compared to the total energy $E_{XY}$, justifying a first-order Taylor expansion for the bath entropy where needed. The system's free energy is $F(\lambda) = - \beta^{-1} \log \int dx \, e^{-\beta H_X(x; \lambda)}$, and the change over the process is $\Delta F = F(\lambda_t) - F(\lambda_0)$.

Task:
Derive an expression for the ratio of the joint probability densities for the forward and backward processes, $\frac{p_F(W, Q, x_t, x_0)}{p_B(-W, -Q, x_0^*, x_t^*)}$.

\subsubsection*{Ground Truth Solution}
\[ \boxed{e^{\beta(W - \Delta F)}} \]

\subsubsection*{Model Outputs}
\subsubsection*{Model: o4-mini (Score: 1.0)}
\paragraph*{Model Solution:}
\[ \frac{p_F\bigl(W,Q,x_t,x_0\bigr)}{p_B\bigl(-W,-Q,x_0^*,x_t^*\bigr)}
\;=\;
\frac{\Omega_{XY}\bigl(E_{XY}+W,\lambda_t\bigr)}
     {\Omega_{XY}\bigl(E_{XY},\lambda_0\bigr)}
\;\simeq\;
\exp\!\bigl[\,\beta\,(W-\Delta F)\bigr] \]

\paragraph*{Judge's Evaluation:}

The ground truth expression is:
\[
e^{\beta(W - \Delta F)}
\]

The model-generated expression is:
\[
\frac{p_F\bigl(W,Q,x_t,x_0\bigr)}{p_B\bigl(-W,-Q,x_0^*,x_t^*\bigr)}
\;=\;
\frac{\Omega_{XY}\bigl(E_{XY}+W,\lambda_t\bigr)}
     {\Omega_{XY}\bigl(E_{XY},\lambda_0\bigr)}
\;\simeq\;
\exp\!\bigl[\,\beta\,(W-\Delta F)\bigr]
\]

The model-generated expression contains the ground truth expression as the final approximate equality. The first two fractions represent the ratio of forward and backward path probabilities and the ratio of density of states, which are intermediate steps leading to the final fluctuation theorem expression. The final approximate equality matches the ground truth expression exactly in form and content.

Since the final expression is mathematically equivalent to the ground truth and the additional terms are consistent with the derivation context (and do not contradict the ground truth), the model-generated expression is completely correct.

**Score: 1**

\newpage
\subsection*{Problem 32 (Paper: 2506.23609v1)}
\subsubsection*{Problem Statement}
Background:
Consider a Dirac spinor field $\psi$ with mass $m$ in a four-dimensional metric-affine spacetime. The dynamics are described by the Lagrangian 4-form:
$${L}_{\mathbb{D}}   = \frac{i}{2} \left[ \overline{\psi} * \gamma  \wedge  \mathbb{D}\psi + \mathbb{D}\overline{\psi} \wedge *\gamma  \psi \right] + im \overline{\psi} \psi *1$$
Here, $\overline{\psi} = \psi^\dagger \gamma_0$ is the Dirac adjoint, $*$ is the Hodge star operator, $\gamma = \gamma_a e^a$ is the Clifford algebra-valued 1-form built from the orthonormal coframe $e^a$ and gamma matrices $\gamma_a$, and $*1$ is the volume form. The gamma matrices satisfy $\{\gamma_a, \gamma_b\} = 2\eta_{ab}I$.

The generalized covariant exterior derivative of the spinor, $\mathbb{D}\psi$, and its adjoint, $\mathbb{D}\overline{\psi}$, are defined as:
$$ \mathbb{D}\psi = d\psi + \Big[\frac{1}{2} \sigma_{ab} \omega^{ab} + (a_1 I  + a_2 \gamma_5) Q + (a_3 I  + a_4 \gamma_5) P + (b_1 I  + b_2 \gamma_5) T + (b_3 I - b_4 \gamma_5) \gamma  \Big] \psi $$
$$ \mathbb{D}\overline{\psi} = d\overline{\psi} + \overline{\psi}\Big[- \frac{1}{2}\sigma_{ab}\omega^{ab} + (a_1^\star I  + a_2^\star \gamma_5)Q + (a_3^\star I  + a_4^\star\gamma_5) P + (b_1^\star I  + b_2^\star \gamma_5)T - (b_3^\star I + b_4^\star \gamma_5)\gamma \Big] $$
where $\sigma_{ab} = \frac{1}{4}[\gamma_a, \gamma_b]$, $\gamma_5$ is the chirality operator, and $a_i, b_i$ are arbitrary complex coupling constants ($^\star$ denotes complex conjugation). The geometric quantities are the full connection 1-form $\omega^{ab}$, the first trace of non-metricity $Q = \eta_{cd}Q^{cd}$, the second trace of non-metricity $P = (\iota_c Q^{cd})e_d$, and the trace of torsion $T = \iota_c T^c$.

You are given the following two identities which hold up to an exact form $d(\dots)$ that does not contribute to the equations of motion:
1. The second kinetic term in the Lagrangian can be related to the first via:
$$ \mathbb{D}\overline{\psi} \wedge *\gamma  \psi \approx \overline{\psi} \wedge *\gamma \wedge \mathbb{D}'\psi - \overline{\psi}(D*e^a - \omega^{(ab)}\wedge*e_b)\gamma_a\psi $$
where $\approx$ denotes equality up to a boundary term, $D$ is the covariant exterior derivative acting on forms, and $\mathbb{D}'\psi$ is given by:
$$ \mathbb{D}'\psi = \Big\{ d + \frac{1}{2}  \sigma_{ab} \omega^{ab}   + (-a_1^\star I  + a_2^\star \gamma_5)Q + (-a_3^\star I  + a_4^\star\gamma_5) P + (-b_1^\star I  + b_2^\star \gamma_5)T + (b_3^\star I  - b_4^\star \gamma_5 ) \gamma   \Big\} \psi $$
2. The geometric term involving the covariant derivative of the Hodge dual of the coframe is:
$$ D*e^a - \omega^{(ab)}\wedge*e_b =*e_a  \wedge (Q+T-P) $$
where $\omega^{(ab)}$ is the symmetric part of the connection 1-form.

Task:
By applying the variational principle to the Lagrangian $L_{\mathbb{D}}$, derive the equation of motion for the spinor field $\psi$. Your final answer should be the full variational Dirac equation, expressed in terms of the arbitrary complex constants $a_i, b_i$.

\subsubsection*{Ground Truth Solution}
\[ \boxed{i*\gamma \wedge \Big\{ d + \frac{1}{2}  \sigma_{ab} \omega^{ab}  +\frac{1}{2}[ (a_1- a_1^\star-1) I  + (a_2+a_2^\star) \gamma_5]Q + \frac{1}{2}[(a_3-a_3^\star+1) I  + (a_4+a_4^\star)\gamma_5] P  +\frac{1}{2}[(b_1-b_1^\star-1) I  + (b_2+b_2^\star) \gamma_5]T  + \frac{1}{2}[(b_3+b_3^\star)I - (b_4+b_4^\star) \gamma_5 ] \gamma   \Big\}  \psi   + im\psi*1 = 0} \]

\subsubsection*{Model Outputs}
\subsubsection*{Model: o4-mini (Score: 0.0)}
\paragraph*{Model Solution:}
\[ %
*\gamma\wedge
\Bigl(d+\tfrac12\sigma_{ab}\,\omega^{ab}
+\Bigl[\frac{a_1-a_1^*}{2}-\tfrac12\Bigr]I\,Q
+\frac{a_2+a_2^*}{2}\,\gamma_5\,Q
+\Bigl[\frac{a_3-a_3^*}{2}+\tfrac12\Bigr]I\,P
+\frac{a_4+a_4^*}{2}\,\gamma_5\,P
+\Bigl[\frac{b_1-b_1^*}{2}-\tfrac12\Bigr]I\,T
+\frac{b_2+b_2^*}{2}\,\gamma_5\,T
+\frac{b_3+b_3^*}{2}\,I\,\gamma
-\frac{b_4-b_4^*}{2}\,\gamma_5\,\gamma
\Bigr)\psi
\;+\;m\,\psi\,*1\;=\;0\,. \]

\paragraph*{Judge's Evaluation:}

Let's rewrite both expressions in a clearer form to compare term-by-term.

---

**Ground Truth Expression:**

\[
i \gamma \wedge \Bigg\{ d + \frac{1}{2} \sigma_{ab} \omega^{ab} 
+ \frac{1}{2} \big[ (a_1 - a_1^\star - 1) I + (a_2 + a_2^\star) \gamma_5 \big] Q
+ \frac{1}{2} \big[ (a_3 - a_3^\star + 1) I + (a_4 + a_4^\star) \gamma_5 \big] P
+ \frac{1}{2} \big[ (b_1 - b_1^\star - 1) I + (b_2 + b_2^\star) \gamma_5 \big] T
+ \frac{1}{2} \big[ (b_3 + b_3^\star) I - (b_4 + b_4^\star) \gamma_5 \big] \gamma
\Bigg\} \psi + i m \psi * 1 = 0
\]

---

**Model-Generated Expression:**

\[
* \gamma \wedge \Biggl(
d + \frac{1}{2} \sigma_{ab} \omega^{ab}
+ \left[ \frac{a_1 - a_1^*}{2} - \frac{1}{2} \right] I Q
+ \frac{a_2 + a_2^*}{2} \gamma_5 Q
+ \left[ \frac{a_3 - a_3^*}{2} + \frac{1}{2} \right] I P
+ \frac{a_4 + a_4^*}{2} \gamma_5 P
+ \left[ \frac{b_1 - b_1^*}{2} - \frac{1}{2} \right] I T
+ \frac{b_2 + b_2^*}{2} \gamma_5 T
+ \frac{b_3 + b_3^*}{2} I \gamma
- \frac{b_4 - b_4^*}{2} \gamma_5 \gamma
\Biggr) \psi + m \psi * 1 = 0
\]

---

**Step-by-step comparison:**

1. **Overall factor:**

- Ground truth: \( i \gamma \wedge \{ \ldots \} \psi + i m \psi * 1 = 0 \)
- Model: \( * \gamma \wedge ( \ldots ) \psi + m \psi * 1 = 0 \)

The model has a leading star \( * \) before \(\gamma \wedge\), which is not present in the ground truth. Also, the ground truth has an overall factor of \(i\) multiplying the entire expression, including the mass term, while the model does not have the factor \(i\) in front of the mass term.

This is a significant difference.

2. **Terms inside the braces:**

- The model writes the coefficients as \(\frac{a_1 - a_1^*}{2} - \frac{1}{2}\) for the \(I Q\) term, which matches the ground truth's \(\frac{1}{2}(a_1 - a_1^\star - 1)\).

- Similarly for the \(I P\) term: model has \(\frac{a_3 - a_3^*}{2} + \frac{1}{2}\), matching ground truth's \(\frac{1}{2}(a_3 - a_3^\star + 1)\).

- For the \(I T\) term: model has \(\frac{b_1 - b_1^*}{2} - \frac{1}{2}\), matching ground truth's \(\frac{1}{2}(b_1 - b_1^\star - 1)\).

- The \(\gamma_5\) terms for \(Q, P, T\) match exactly in coefficients.

- For the \(\gamma\) terms:

  - Ground truth: \(\frac{1}{2}[(b_3 + b_3^\star) I - (b_4 + b_4^\star) \gamma_5] \gamma\)

  - Model: \(\frac{b_3 + b_3^*}{2} I \gamma - \frac{b_4 - b_4^*}{2} \gamma_5 \gamma\)

Here, the model has \(- \frac{b_4 - b_4^*}{2}\) while the ground truth has \(- \frac{b_4 + b_4^\star}{2}\).

This is a discrepancy: the model uses a difference \(b_4 - b_4^*\) instead of a sum \(b_4 + b_4^\star\).

3. **Mass term:**

- Ground truth: \(+ i m \psi * 1\)

- Model: \(+ m \psi * 1\)

Missing factor of \(i\) in the model.

---

**Summary of differences:**

- The model has a leading star \( * \) before \(\gamma \wedge\), which is not in the ground truth.

- The model is missing the factor \(i\) multiplying the entire expression and the mass term.

- The coefficient of the \(\gamma_5 \gamma\) term differs: model has \(- \frac{b_4 - b_4^*}{2}\), ground truth has \(- \frac{b_4 + b_4^\star}{2}\).

---

**Assessment:**

- The model is very close in the structure and coefficients for most terms.

- However, the missing factor \(i\) in front of the entire expression and the mass term is a significant error.

- The sign and sum/difference mismatch in the \(\gamma_5 \gamma\) term is a second error.

- The extra star \( * \) before \(\gamma \wedge\) is unexplained and likely an error.

Given these multiple errors (missing factor \(i\), wrong sign/sum vs difference in one term, and extra star), the model's expression is not fully correct.

---

**Score:**

- Not completely correct (score 1) because of multiple errors.

- Not partially correct (score 0.5) because there are more than one minor errors.

Therefore, the score is 0.

---

**Final answer:**

\[
\boxed{
\begin{cases}
\text{Missing factor } i \text{ in front of the entire expression and mass term} \\
\text{Incorrect sign and sum/difference in } \gamma_5 \gamma \text{ term} \\
\text{Extra star } * \text{ before } \gamma \wedge \\
\Rightarrow \text{Multiple errors, not just minor}
\end{cases}
}
\]

Score: 0

\newpage
\subsection*{Problem 33 (Paper: 2506.23609v1)}
\subsubsection*{Problem Statement}
Background:
Consider a Dirac spinor field $\psi$ with mass $m$ in a four-dimensional metric-affine spacetime. The dynamics are described by the Lagrangian 4-form
$L_D = \mathcal{H}er \left[ i\overline{\psi} * \gamma \wedge \mathbb{D}\psi + im \overline{\psi} \psi *1 \right]$,
where $\mathcal{H}er[Z] = \frac{1}{2}(Z + Z^\dagger)$ denotes the Hermitian part, $\overline{\psi} = \psi^\dagger \gamma_0$ is the Dirac adjoint, $*$ is the Hodge star operator, and $*1$ is the volume 4-form. The Clifford algebra-valued 1-form $\gamma$ is defined as $\gamma = \gamma_a e^a$, where $e^a$ is the orthonormal coframe and $\gamma_a$ are the Dirac matrices satisfying $\{\gamma_a, \gamma_b\} = 2\eta_{ab}I$ with $\eta_{ab} = \text{diag}(-1,1,1,1)$. The generalized covariant exterior derivative $\mathbb{D}$ acting on the spinor is given by:
$$ \mathbb{D}\psi = d\psi + \Big[\frac{1}{2} \sigma_{ab} \omega^{ab} + (a_1 I  + a_2 \gamma_5) Q + (a_3 I  + a_4 \gamma_5) P + (b_1 I  + b_2 \gamma_5) T + (b_3 I - b_4 \gamma_5) \gamma  \Big] \psi $$
Here, $\sigma_{ab} = \frac{1}{4}(\gamma_a\gamma_b - \gamma_b\gamma_a)$, $\gamma_5 = \gamma_0\gamma_1\gamma_2\gamma_3$, $\omega^{ab}$ is the full connection 1-form, and $Q, P, T$ are 1-forms constructed from the non-metricity and torsion tensors. The coefficients $a_i$ are complex, while $b_3$ and $b_4$ are real constants. This covariant derivative can be split into two parts:
$$ \mathbb{D}\psi = \widehat{\mathbb{D}}\psi + (b_3 I - b_4 \gamma_5)\gamma \psi $$
where $\widehat{\mathbb{D}}\psi$ contains all terms not proportional to the coframe-related 1-form $\gamma$:
$$ \widehat{\mathbb{D}}\psi = d\psi + \Big[\frac{1}{2} \sigma_{ab} \omega^{ab} + (a_1 I  + a_2 \gamma_5) Q + (a_3 I  + a_4 \gamma_5) P + (b_1 I  + b_2 \gamma_5) T \Big] \psi $$
You are given the following identities of Clifford algebra and differential forms:
1. $[\gamma_a, \gamma_5] = 2\gamma_a\gamma_5$, which implies $\gamma_a\gamma_5 = -\gamma_5\gamma_a$.
2. $\gamma_a \gamma^a = 4I$.
3. $*e^a \wedge e^b = -\eta^{ab}*1$.

Task:
By substituting the decomposition of $\mathbb{D}\psi$ into the Lagrangian $L_D$, manipulate the expression to show how the geometry-spinor coupling terms proportional to the coframe $\gamma$ modify the mass term of the Lagrangian. Derive the final expression for the Lagrangian $L_D$ written in terms of $\widehat{\mathbb{D}}\psi$ and an effective, chirality-dependent mass.

\subsubsection*{Ground Truth Solution}
\[ \boxed{L_D = \mathcal{H}er \left[ i\overline{\psi} * \gamma \wedge \widehat{\mathbb{D}}\psi + i\overline{\psi} (m -4 b_3 -4 b_4 \gamma_5) \psi *1 \right]} \]

\subsubsection*{Model Outputs}
\subsubsection*{Model: o4-mini (Score: 0.5)}
\paragraph*{Model Solution:}
\[ L_{D}
=\mathcal{H}er\Bigl[
i\,\bar\psi*\!\gamma\wedge\widehat{\mathbb{D}}\psi
\;+\;i\,\bar\psi\bigl(m-4b_{3}+4b_{4}\,\gamma_{5}\bigr)\psi\,*1
\Bigr]\,. \]

\paragraph*{Judge's Evaluation:}

The ground truth expression is:
\[
L_D = \mathcal{H}er \left[ i\overline{\psi} * \gamma \wedge \widehat{\mathbb{D}}\psi + i\overline{\psi} (m -4 b_3 -4 b_4 \gamma_5) \psi *1 \right]
\]

The model-generated expression is:
\[
L_{D} = \mathcal{H}er \Bigl[ i\,\bar\psi * \gamma \wedge \widehat{\mathbb{D}}\psi + i\,\bar\psi \bigl(m - 4 b_3 + 4 b_4 \gamma_5 \bigr) \psi *1 \Bigr]
\]

Comparing the two:

- The first term inside the bracket, \( i \overline{\psi} * \gamma \wedge \widehat{\mathbb{D}} \psi \), matches exactly.
- The second term differs in the sign in front of the \(4 b_4 \gamma_5\) term:
  - Ground truth: \(-4 b_4 \gamma_5\)
  - Model: \(+4 b_4 \gamma_5\)

This is a sign error in a key term, which is not a minor typographical difference but changes the physics and mathematical correctness of the expression.

Since there is one sign error in a significant term, the expression is not completely correct.

According to the rubric:

- One wrong sign is considered a minor error, so partial credit (0.5) is possible.
- The rest of the expression is correct and matches exactly.

Therefore, the score should be 0.5.

---

**Score: 0.5**

\newpage
\subsection*{Problem 34 (Paper: 2506.23687v1)}
\subsubsection*{Problem Statement}
Background:
Consider a $q$-state active Potts model on a two-dimensional square lattice. Each site $i$ of the lattice is in a state $s_i \in \{0, 1, \dots, q-1\}$. The system's dynamics are governed by single-site Metropolis Monte Carlo updates. The probability of accepting a flip from state $s$ to $s'$ at a site $i$ is given by $\min(1, \exp(-\Delta E_{s \to s'}))$, where the total energy change is $\Delta E_{s \to s'} = \Delta H_{\mathrm{int}} - h_{s,s'}$. The interaction energy is $H_{\mathrm{int}} = - \sum_{\langle i,j \rangle} J_{s_i, s_j}$, where the sum is over nearest-neighbor pairs, and $J_{s,s'}$ is the contact energy between states $s$ and $s'$. The term $h_{s,s'}$ represents a non-equilibrium driving potential. For this problem, we consider a $q=4$ system with cyclic symmetry. The driving is non-zero only for transitions between adjacent states in the cycle $0 \to 1 \to 2 \to 3 \to 0$, such that $h_{k, [k+1]} = h$ and $h_{k, [k-1]} = -h$ for $k \in \{0,1,2,3\}$, where $[k]$ denotes $k \pmod 4$. All other $h_{s,s'}$ are zero. The contact energies are symmetric, $J_{s,s'} = J_{s',s}$, and depend only on the cyclic distance between states, i.e., $J_{s,s'} = J_{k,k'}$ if $|s-s'| \pmod 4 = |k-k'| \pmod 4$.

This active Potts model can be related to an active Ashkin-Teller model. In this second model, each site $i$ is described by a pair of Ising spins $(\sigma_i^A, \sigma_i^B)$, where $\sigma_i^\alpha \in \{-1, 1\}$. The four states of the Potts model are mapped to the four spin-pair states as follows: $s=0 \leftrightarrow (1,1)$, $s=1 \leftrightarrow (-1,1)$, $s=2 \leftrightarrow (-1,-1)$, and $s=3 \leftrightarrow (1,-1)$. The dynamics of the Ashkin-Teller model are also governed by single-spin-flip Metropolis updates. The energy change for flipping spin $\sigma_i^A$ is $\Delta E^A = 2J\sigma_i^A \sum_{j \in \text{nn}(i)} \sigma_j^A - 2K \sigma_i^A \sigma_i^B$, and for flipping spin $\sigma_i^B$ is $\Delta E^B = 2J\sigma_i^B \sum_{j \in \text{nn}(i)} \sigma_j^B + 2K \sigma_i^B \sigma_i^A$, where $J$ is the ferromagnetic coupling constant and $K$ is a non-reciprocal coupling constant. Note that a transition $k \to [k+1]$ in the Potts model corresponds to a single spin flip in the Ashkin-Teller model (e.g., $s=0 \to s=1$ corresponds to flipping $\sigma^A$ while $\sigma^B$ is fixed).

Task:
By requiring that the energy change for any single-site transition $s_i \to s'_i$ is identical in both models for any configuration of neighboring sites, determine the relationship between the parameters of the two models. Derive the general expression for the contact energy $J_{k,s}$ for an arbitrary pair of states $(k,s)$ in the Potts model as a function of the Ashkin-Teller coupling $J$ and the state indices $k$ and $s$.

\subsubsection*{Ground Truth Solution}
\[ \boxed{J_{k,s} = 2J \cos\left(\frac{\pi(k-s)}{2}\right)} \]

\subsubsection*{Model Outputs}
\subsubsection*{Model: o4-mini (Score: 1.0)}
\paragraph*{Model Solution:}
\[ \displaystyle 
     J_{k,s}
     \;=\;
     2\,J\,
     \cos\!\biggl(\frac{\pi\,(k-s)}{2}\biggr)\,. \]

\paragraph*{Judge's Evaluation:}

The model-generated expression is

\[
J_{k,s} = 2 J \cos\left(\frac{\pi (k - s)}{2}\right)
\]

and the ground truth expression is

\[
J_{k,s} = 2J \cos\left(\frac{\pi(k-s)}{2}\right).
\]

Both expressions are mathematically identical. The only differences are in formatting and spacing, which do not affect the mathematical equivalence. The variable names and the structure of the expression match exactly.

**Score: 1**

\newpage
\subsection*{Problem 35 (Paper: 2506.23687v1)}
\subsubsection*{Problem Statement}
Background:
A 4-state active Potts model on a lattice has sites with states $s_i \in \{0, 1, 2, 3\}$ and an interaction Hamiltonian $H_{\mathrm{int}} = - \sum_{\langle ij\rangle} J_{s_is_j}$, where the sum is over nearest-neighbor pairs. This system can be mapped onto an active 4-state vector Potts model, also known as the Ashkin-Teller model, where each site has two Ising spins, $\sigma^A = \pm 1$ and $\sigma^B = \pm 1$. The mapping between the Potts states and the spin pairs is defined as: $s=0 \leftrightarrow (1,1)$, $s=1 \leftrightarrow (-1,1)$, $s=2 \leftrightarrow (-1,-1)$, and $s=3 \leftrightarrow (1,-1)$. In the vector model, the interaction energy between two nearest-neighbor sites $i$ and $j$ is given by $H_{ij} = -J(\sigma_i^A \sigma_j^A + \sigma_i^B \sigma_j^B)$. By equating the interaction Hamiltonians of the two models, the Potts contact energies $J_{s_i s_j}$ can be expressed in terms of the parameter $J$ and the spin states corresponding to $s_i$ and $s_j$. Due to the cyclic nature of the state mapping, the contact energy $J_{k, k'}$ only depends on the "distance" between the states, i.e., $J_{k, k'} = J_{k, [k+(k'-k)]}$, where $[k]$ denotes $k \pmod 4$. We can therefore write the contact energy as $J_{k, [k+j]}$ where $j = (k'-k) \pmod 4$.

Task:
Using the provided state mapping, derive the general expression for the contact energy $J_{k, [k+j]}$ in the Potts model as a function of $J$ and the state separation $j \in \{0, 1, 2, 3\}$.

\subsubsection*{Ground Truth Solution}
\[ \boxed{J_{k,[k+j]} = 2J\cos\left(\frac{\pi j}{2}\right)} \]

\subsubsection*{Model Outputs}
\subsubsection*{Model: o4-mini (Score: 1.0)}
\paragraph*{Model Solution:}
\[ J_{k,[k+j]} \;=\; 2J\cos\!\bigl(\tfrac{\pi j}{2}\bigr)\,,\quad j=0,1,2,3. \]

\paragraph*{Judge's Evaluation:}

The ground truth expression is:
\[
J_{k,[k+j]} = 2J \cos\left(\frac{\pi j}{2}\right)
\]

The model-generated expression is:
\[
J_{k,[k+j]} = 2J \cos\left(\frac{\pi j}{2}\right), \quad j=0,1,2,3.
\]

The model's expression matches the ground truth exactly in the mathematical form of the expression. The only difference is the addition of the domain specification \( j=0,1,2,3 \), which is not part of the original expression but does not alter the correctness of the formula itself. This additional information is not an error; it is simply a clarification or restriction on the variable \( j \).

Since the core expression is mathematically equivalent and correct, and the added domain specification does not contradict or change the formula, the model's expression should be considered completely correct.

**Score: 1**

\newpage
\subsection*{Problem 36 (Paper: 2506.23792v1)}
\subsubsection*{Problem Statement}
Background:
Consider a two-dimensional system where a particle of mass $m=1$ moves with constant total energy $E$ in a periodic potential $V(\boldsymbol{r})$. The potential consists of a triangular lattice of attractive wells, which act as "traps". The particle's motion can be modeled as a random walk between these traps. The diffusion coefficient $D$ for this process is given by the random walk formula $D = \frac{l^2}{4\tau}$, where $l$ is the fixed distance between the centers of adjacent traps and $\tau$ is the mean residence time of the particle within a single trap.

According to the Machta-Zwanzig approximation, the mean residence time is given by the ratio of the trap's phase space volume $\Omega$ to the total phase space flux $\omega$ out of the trap, i.e., $\tau \approx \Omega/\omega$. The phase space volume for a single trap is given by $\Omega = 2\pi \int_{V(\vect{r}) \leq E} v(\vect{r}) \dd{\vect{r}}$, where $v(\boldsymbol{r}) = \sqrt{2(E - V(\boldsymbol{r}))}$ is the particle's speed and the integral is over the energetically accessible region within the trap. The trap is part of a triangular lattice and has six identical exits to its neighbors. The total flux out of the trap is given by $\omega = 6 \int_{\text{exit}} \int_{-\pi/2}^{\pi/2} v(y)^2 \cos\theta \dd\theta \dd y$, where the spatial integral is over a single exit line, $y$ is the coordinate along this line, and $\theta$ is the angle of the velocity vector relative to the exit's normal.

To express the diffusion coefficient in terms of averaged quantities, we define the following:
- $A_{\text{trap}}$ is the area of the unit cell corresponding to a single trap.
- $\mean{v(\vect{r})}_{\text{trap}} = \frac{1}{A_{\text{trap}}} \int_{V(\vect{r}) \leq E} v(\vect{r}) \dd{\vect{r}}$ is the average speed over the trap area.
- $l_{\text{exit}} = \int_{V(y) \leq E} \dd y$ is the effective length of a single exit.
- $\mean{v(\vect{r})^2}_{\text{exit}} = \frac{1}{l_{\text{exit}}} \int_{V(y) \leq E} v(y)^2 \dd y$ is the mean squared speed over a single exit.

Task:
Using the provided relations and definitions, derive an expression for the diffusion coefficient $D$ in terms of the trap separation $l$, the trap area $A_{\text{trap}}$, the average trap speed $\mean{v(\vect{r})}_{\text{trap}}$, the exit length $l_{\text{exit}}$, and the average exit speed squared $\mean{v(\vect{r})^2}_{\text{exit}}$.

\subsubsection*{Ground Truth Solution}
\[ \boxed{D = \frac{3 l^2 l_{\text{exit}} \mean{v(\vect{r})^2}_{\text{exit}}}{2\pi A_{\text{trap}} \mean{v(\vect{r})}_{\text{trap}}}} \]

\subsubsection*{Model Outputs}
\subsubsection*{Model: o4-mini (Score: 1.0)}
\paragraph*{Model Solution:}
\[ %
D \;=\;\frac{3\,l^{2}\,l_{\rm exit}\,\langle v^{2}\rangle_{\rm exit}}
{2\pi\,A_{\rm trap}\,\langle v\rangle_{\rm trap}}% \]

\paragraph*{Judge's Evaluation:}

The ground truth expression is:
\[
D = \frac{3 l^2 l_{\text{exit}} \mean{v(\vect{r})^2}_{\text{exit}}}{2\pi A_{\text{trap}} \mean{v(\vect{r})}_{\text{trap}}}
\]

The model-generated expression is:
\[
D = \frac{3\,l^{2}\,l_{\rm exit}\,\langle v^{2}\rangle_{\rm exit}}{2\pi\,A_{\rm trap}\,\langle v\rangle_{\rm trap}}
\]

Comparison:
- The numerator and denominator structure is identical.
- The variables correspond exactly: $l^2$, $l_{\text{exit}}$, $A_{\text{trap}}$ match.
- The averages are written as $\mean{v(\vect{r})^2}_{\text{exit}}$ vs. $\langle v^{2}\rangle_{\rm exit}$, which are equivalent notations.
- The vector argument $\vect{r}$ inside the average is omitted in the model expression, but this is a minor notational difference and does not affect mathematical equivalence.
- The constants and coefficients (3, 2π) are all present and correct.
- The order and grouping are consistent.

Therefore, the model-generated expression is mathematically equivalent to the ground truth expression, differing only in notation style.

**Score: 1**

\newpage
\subsection*{Problem 37 (Paper: 2506.23796v1)}
\subsubsection*{Problem Statement}
Background:
An interferometric scheme can be used to measure the out-of-time-ordered correlator ($\mathcal{F}$-OTOC) for an open quantum system. Consider a system (S) with an initial density matrix $\rho_S(0)$ and an auxiliary control qubit (c) initialized in the state $\ket{+}_c = \frac{1}{\sqrt{2}}(\ket{0}_c + \ket{1}_c)$. The total initial state is $\rho_{\mathrm{init}} = \rho_S(0) \otimes \ket{+}\bra{+}_c$. The system's open dynamics are described by two general completely positive trace-preserving (CPTP) maps: a forward evolution map $\xi_f(t)$ and a backward evolution map $\xi_b(t)$. Let $A_S$ and $B_S$ be two unitary operators acting on the system's Hilbert space. The final state of the composite system, $\rho_f$, is obtained by applying the following sequence of five operations to $\rho_{\mathrm{init}}$:
\begin{align*}
    \mathcal{S}_1 &= \mathcal{C}(\mathbf{I}_S\otimes \ket{0}\bra{0}_c + B_S \otimes \ket{1}\bra{1}_c) \\
    \mathcal{S}_2 &= \xi_f(t)\otimes \mathcal{I}_c \\
    \mathcal{S}_3 &= \mathcal{C}(A_S\otimes \mathbf{I}_c) \\
    \mathcal{S}_4 &= \xi_b(t)\otimes \mathcal{I}_c \\
    \mathcal{S}_5 &= \mathcal{C}(B_S \otimes \ket{0}\bra{0}_c + \mathbf{I}_S\ \otimes \ket{1}\bra{1}_c )
\end{align*}
The final state is thus $\rho_f = \mathcal{S}_5 \cdot \mathcal{S}_4 \cdot \mathcal{S}_3 \cdot \mathcal{S}_2 \cdot \mathcal{S}_1 \cdot \rho_{\mathrm{init}}$. Here, $\mathcal{C}(U)$ is a superoperator defined by its action on a density matrix $\rho$ as $\mathcal{C}(U) \cdot \rho = U \rho U^{\dagger}$, and $\mathcal{I}_c$ is the identity superoperator on the control qubit space. The $\mathcal{F}$-OTOC is given by the expectation value of the Pauli operator $\sigma_c^x$ on the control qubit in the final state. The symbol $\Re$ denotes the real part of a complex number. $\mathrm{Tr}$ denotes the trace operation over the system's Hilbert space.

Task:
Derive the expression for the $\mathcal{F}$-OTOC, $\mathcal{F}(t, A, B) = \mathrm{Tr}(\sigma_c^x \rho_f)$. Express your final answer in terms of the operators $A_S$, $B_S$, the initial state $\rho_S(0)$, and the maps $\xi_f(t)$ and $\xi_b(t)$.

Solution:

\subsubsection*{Ground Truth Solution}
\[ \boxed{\mathcal{F}(t, A, B) = \Re\left\{ \mathrm{Tr}\left[ B_S^{\dagger} \, \xi_b(t) \left[ A_S \, \xi_f(t)\left[ B_S \, \rho_S(0) \right] A_S^{\dagger} \right] \right] \right\}

where $\Re$ denotes the real part and $\mathrm{Tr}$ denotes the trace over the system Hilbert space.} \]

\subsubsection*{Model Outputs}
\subsubsection*{Model: o4-mini (Score: 1.0)}
\paragraph*{Model Solution:}
\[ \F(t,A,B)
\;=\;
\Re\,
\Tr_{S}\!\Bigl\{
\xi_{b}(t)\bigl[A_{S}\,\xi_{f}(t)\bigl[B_{S}\,\rho_{S}(0)\bigr]\,A_{S}^{\dagger}\bigr]\;B_{S}^{\dagger}
\Bigr\}\,. \]

\paragraph*{Judge's Evaluation:}

The ground truth expression is:
\[
\mathcal{F}(t, A, B) = \Re\left\{ \mathrm{Tr}\left[ B_S^{\dagger} \, \xi_b(t) \left[ A_S \, \xi_f(t)\left[ B_S \, \rho_S(0) \right] A_S^{\dagger} \right] \right] \right\}
\]

The model-generated expression is:
\[
\mathcal{F}(t,A,B) = \Re \, \mathrm{Tr}_S \left\{ \xi_b(t) \left[ A_S \, \xi_f(t) \left[ B_S \, \rho_S(0) \right] A_S^{\dagger} \right] B_S^{\dagger} \right\}
\]

---

**Comparison:**

- The ground truth has the operator ordering inside the trace as:
\[
B_S^{\dagger} \, \xi_b(t) \left[ A_S \, \xi_f(t) \left[ B_S \, \rho_S(0) \right] A_S^{\dagger} \right]
\]

- The model-generated expression has:
\[
\xi_b(t) \left[ A_S \, \xi_f(t) \left[ B_S \, \rho_S(0) \right] A_S^{\dagger} \right] B_S^{\dagger}
\]

Since the trace is cyclic, i.e., 
\[
\mathrm{Tr}[XYZ] = \mathrm{Tr}[ZXY] = \mathrm{Tr}[YZX],
\]
the difference in the order of operators inside the trace does not affect the value of the expression.

- The model uses $\mathrm{Tr}_S$ instead of $\mathrm{Tr}$, which is consistent since the trace is over the system Hilbert space.

- The real part operator $\Re$ is applied outside the trace in both expressions.

- Variable names and notation are consistent.

---

**Conclusion:**

The model-generated expression is mathematically equivalent to the ground truth expression due to the cyclic property of the trace. There are no errors or missing terms.

---

**Score:**

Score: 1

\newpage
\subsection*{Problem 38 (Paper: 2506.23796v1)}
\subsubsection*{Problem Statement}
Background:
Consider a composite quantum system with a finite-dimensional Hilbert space $\mathcal{H} = \mathcal{H}_A \otimes \mathcal{H}_B$, with corresponding dimensions $d_A$, $d_B$, and $d = d_A d_B$. The system's evolution is described by a completely positive trace-preserving (CPTP) map $\mathcal{E}$, and the corresponding evolution of an operator $O$ is governed by the adjoint map $\mathcal{E}^\dagger$. A measure of information scrambling in such an open system is the bipartite out-of-time-ordered correlator (OTOC), defined as the Haar average of the squared Hilbert-Schmidt norm of the commutator between two local unitary operators, $A \in U(d_A)$ and $B \in U(d_B)$. This quantity, denoted $G(\mathcal{E}^\dagger)$, is given by
$$G(\mathcal{E}^\dagger) = \frac{1}{2d} \mathbb{E}_{A,B} \left\| [\mathcal{E}^{\dagger}(A), B] \right\|_2^2$$
where $\mathbb{E}_{A,B}$ denotes the Haar average over the unitary groups $U(d_A)$ and $U(d_B)$. Expanding the norm, this expression can be written as
$$G(\mathcal{E}^\dagger) = \frac{1}{d} \left( \mathcal{T}_A - \Re\{\mathcal{T}_{AB}\} \right)$$
where $\mathcal{T}_A = \mathbb{E}_A \left\| \mathcal{E}^\dagger(A) \right\|_2^2$ and $\mathcal{T}_{AB} = \mathbb{E}_{A,B} \text{Tr}[\mathcal{E}^\dagger(A)^\dagger B^\dagger \mathcal{E}^\dagger(A) B]$. To evaluate the averaged terms, we introduce a replica of the Hilbert space, $\mathcal{H}' = \mathcal{H}'_A \otimes \mathcal{H}'_B$, and define swap operators on the doubled space $\mathcal{H} \otimes \mathcal{H}'$. The operator $S_{XX'}$ swaps the states in subsystem $X$ and its replica $X'$, while the total swap operator is $S = S_{AA'}S_{BB'}$. Using this formalism, the averaged terms can be expressed in terms of the tensored adjoint map $\mathcal{E}^{\dagger \otimes 2} = \mathcal{E}^\dagger \otimes \mathcal{E}^\dagger$ as:
\begin{align*}
\mathcal{T}_A &= \frac{1}{d_A} \text{Tr}_{\mathcal{H}\otimes\mathcal{H}'}[S_{AA'} (\mathcal{E}^{\dagger \otimes 2}) S_{AA'}] \\
\mathcal{T}_{AB} &= \frac{1}{d_A d_B} \text{Tr}_{\mathcal{H}\otimes\mathcal{H}'}[S (\mathcal{E}^{\dagger \otimes 2})]
\end{align*}
Assume the map $\mathcal{E}^\dagger$ is such that the trace of $\mathcal{T}_{AB}$ is real.

Task:
Using the provided relations, derive the expression for the bipartite OTOC, $G(\mathcal{E}^\dagger)$. Express your final answer as a single trace over the doubled Hilbert space $\mathcal{H} \otimes \mathcal{H}'$.

\subsubsection*{Ground Truth Solution}
\[ \boxed{G(\mathcal{E}^{\dagger}) = \frac{1}{d^2} \text{Tr} \left[ \left( d_B S_{AA'} - S \right) (\mathcal{E}^{\dagger \otimes 2}) S_{AA'} \right]} \]

\subsubsection*{Model Outputs}
\subsubsection*{Model: o4-mini (Score: 0.5)}
\paragraph*{Model Solution:}
\[ G(\mathcal{E}^\dagger)
=\frac{1}{d^2}\,
\Tr_{\mathcal H\otimes\mathcal H'}\!\bigl[\bigl(d_B\,S_{AA'}-S_{AA'}S_{BB'}\bigr)\;\mathcal{E}^{\dagger\otimes2}\bigr] \]

\paragraph*{Judge's Evaluation:}

The ground truth expression is:
\[
G(\mathcal{E}^{\dagger}) = \frac{1}{d^2} \operatorname{Tr} \left[ \left( d_B S_{AA'} - S \right) (\mathcal{E}^{\dagger \otimes 2}) S_{AA'} \right]
\]

The model-generated expression is:
\[
G(\mathcal{E}^\dagger) = \frac{1}{d^2} \operatorname{Tr}_{\mathcal{H} \otimes \mathcal{H}'} \left[ \left( d_B S_{AA'} - S_{AA'} S_{BB'} \right) \mathcal{E}^{\dagger \otimes 2} \right]
\]

---

**Comparison:**

1. **Difference in the second term inside the parentheses:**

- Ground truth has \( S \), presumably the swap operator on the full space, while the model has \( S_{AA'} S_{BB'} \).

- Typically, \( S \) acting on the full space \( \mathcal{H} \otimes \mathcal{H}' \) can be decomposed as \( S_{AA'} S_{BB'} \) if the space factorizes as \( (A \otimes B) \otimes (A' \otimes B') \). So this might be equivalent.

2. **Presence of the extra \( S_{AA'} \) on the right of \( \mathcal{E}^{\dagger \otimes 2} \) in the ground truth:**

- The ground truth expression has the operator \( S_{AA'} \) multiplying on the right:
\[
\left( d_B S_{AA'} - S \right) (\mathcal{E}^{\dagger \otimes 2}) S_{AA'}
\]

- The model expression does not have this extra \( S_{AA'} \) on the right side of \( \mathcal{E}^{\dagger \otimes 2} \).

3. **Trace domain:**

- Ground truth uses \(\operatorname{Tr}\) without specifying the space, but presumably over the full space.

- Model uses \(\operatorname{Tr}_{\mathcal{H} \otimes \mathcal{H}'}\), which is consistent.

---

**Is \( S = S_{AA'} S_{BB'} \)?**

- The swap operator \( S \) on the full space \( (A \otimes B) \otimes (A' \otimes B') \) swaps the two copies of the full system.

- \( S_{AA'} \) swaps \( A \) and \( A' \), and \( S_{BB'} \) swaps \( B \) and \( B' \).

- The combined operator \( S_{AA'} S_{BB'} \) swaps both \( A \leftrightarrow A' \) and \( B \leftrightarrow B' \), which is exactly the full swap \( S \).

- So \( S = S_{AA'} S_{BB'} \) is correct.

---

**Missing \( S_{AA'} \) on the right:**

- The ground truth has an extra \( S_{AA'} \) multiplying on the right of \( \mathcal{E}^{\dagger \otimes 2} \).

- The model expression omits this factor.

- This is a nontrivial difference and affects the final expression.

---

**Conclusion:**

- The model correctly replaces \( S \) by \( S_{AA'} S_{BB'} \), which is equivalent.

- However, the model misses the right multiplication by \( S_{AA'} \).

- This is a significant difference, not a minor typo or missing constant.

- Therefore, the model expression is not fully correct.

- Since the model is very close except for this missing factor, it is partially correct.

---

**Score:**

\[
\boxed{0.5}
\]

---

**Final answer:**

The model expression correctly identifies \( S = S_{AA'} S_{BB'} \), but it omits the crucial right multiplication by \( S_{AA'} \) after \( \mathcal{E}^{\dagger \otimes 2} \). This omission is significant and changes the expression. Hence, the model is partially correct but not fully.

Score: 0.5

\newpage
\subsection*{Problem 39 (Paper: 2506.23837v1)}
\subsubsection*{Problem Statement}
Background:
Consider an agent-based model for a financial market with $N$ traders. Each trader $i$ can adopt one of two states: a 'bullish' (buy) state, represented by $s_i = +1$, or a 'bearish' (sell) state, represented by $s_i = -1$. The state of a trader is updated stochastically based on the influence of other traders and external factors. This total influence is captured by a local field, $H_i(t)$. The probability that trader $i$ will adopt the bullish state at time $t+1$ is given by a heat bath rule:
$$p(s_i(t+1) = +1) = \frac{1}{1 + \exp(-2\beta H_i(t))}$$
where $\beta$ is a responsiveness parameter, analogous to an inverse temperature, that quantifies the sensitivity of traders to the local field.

The local field $H_i(t)$ on a trader $i$ is determined by three components:
1.  A 'herding' or 'Ising' interaction, where traders tend to align their state with others.
2.  A 'contrarian' interaction, where traders may choose to act against the majority. This is governed by an individual strategy parameter $C_i$ and a coupling constant $\alpha$.
3.  An external field $h_{ext}$ representing global information or market sentiment.

The full expression for the local field is given by:
$$H_i(t) = \sum_{j=1}^N J_{ij} s_j(t) - \alpha C_i(t) \frac{1}{N}\sum_{j=1}^N s_j(t) + h_{ext}$$
where $J_{ij}$ is the interaction strength between traders $i$ and $j$.

To analyze the collective behavior of the market, we employ a mean-field approximation. We assume the market is fully connected, so the herding interaction strength is uniform, $J_{ij} = J/N$ for all $i \neq j$, and $J_{ii}=0$. We also assume all traders have the same contrarian strategy parameter, $C_i = C$. The overall market state is characterized by the average opinion or 'magnetization', $m(t) = \frac{1}{N}\sum_{j=1}^N s_j(t)$.

Task:
In the mean-field approximation and the thermodynamic limit ($N \to \infty$), the system can reach a steady state where the average market opinion $m$ is constant. Derive the self-consistency equation that determines the possible values of this steady-state market opinion $m$.

\subsubsection*{Ground Truth Solution}
\[ \boxed{m = \tanh\left(\beta\left((J - \alpha C)m + h_{ext}\right)\right)} \]

\subsubsection*{Model Outputs}
\subsubsection*{Model: o4-mini (Score: N/A)}
\paragraph*{Model Solution:}
\[ %
m \;=\;\tanh\!\bigl[\beta\bigl((J-\alpha C)\,m + h_{ext}\bigr)\bigr]. \]

\paragraph*{Judge's Evaluation:}

The ground truth expression is:
\[
m = \tanh\left(\beta\left((J - \alpha C)m + h_{ext}\right)\right)
\]

The model-generated expression is:
\[
m = \tanh\!\bigl[\beta\bigl((J-\alpha C)\,m + h_{ext}\bigr)\bigr].
\]

Comparing the two:

- Both expressions have the same structure: \( m = \tanh(\beta((J - \alpha C)m + h_{ext})) \).
- The model uses different bracket styles \(\bigl[\cdots\bigr]\) and \(\bigl(\cdots\bigr)\), which is purely stylistic and does not affect mathematical equivalence.
- The spacing and the use of \(\,\) for spacing in the model expression is irrelevant to correctness.
- Variable names and terms match exactly.
- The model expression is mathematically equivalent to the ground truth.

Therefore, the model-generated expression is completely correct.

\[
\boxed{
\text{Score: } 1
}
\]

\newpage
\subsection*{Problem 40 (Paper: 2506.23867v1)}
\subsubsection*{Problem Statement}
Background:
Consider a system of non-interacting particles diffusing in one dimension with a diffusion coefficient $D$. The system is defined on a line of length $L$, from $x=-L/2$ to $x=L/2$. In a central "channel" region, for $|x| \le L_0/2$, particles can reversibly adsorb to a stationary surface. The adsorption rate is $\kon$ and the desorption rate is $\koff$. Let $p(x,t)$ be the probability density of finding a particle at position $x$ on the diffusion line, and $q(x,t)$ be the probability density of finding it in the adsorbed state at the corresponding position $x$. The governing equations for these densities are:
\begin{equation*}
\begin{cases}
    \text{for } |x| \le L_0/2: & \begin{cases} \partial_t p = D \partial_{xx}p - \kon p + \koff q \\ \partial_t q = \kon p - \koff q \end{cases} \\
    \text{for } |x| > L_0/2: & \partial_t p = D \partial_{xx}p
\end{cases}
\end{equation*}
The system has no-flux boundary conditions at its edges, i.e., $\partial_x p(\pm L/2, t) = 0$. At the interfaces of the channel, $x=\pm L_0/2$, both the density $p(x,t)$ and its spatial derivative $\partial_x p(x,t)$ are continuous. We analyze the system in Laplace space, using the transform $\hat{f}(s) = \int_0^\infty f(t) e^{-st} dt$. The equations are solved assuming uniform initial densities $p(x,0)=p_0$ and $q(x,0)=q_0$ for $|x| \le L_0/2$, and zero elsewhere. The general solution for the Laplace transform of the mobile particle density, $\hat{p_C}(x,s)$, inside the channel ($|x| \le L_0/2$) is given by:
\begin{equation*}
    \hat{p_C}(x,s) = A_C e^{m_1 x} + B_C e^{-m_1 x} + \frac{\koff(p_0+q_0) + sp_0}{s(s+\koff+\kon)}
\end{equation*}
where $A_C$ and $B_C$ are constants to be determined from the boundary conditions. The characteristic inverse lengths are defined as $m_0(s) = \sqrt{s/D}$ and $m_1(s) = \sqrt{\frac{s(s+\koff+\kon)}{D(s+\koff)}}$. The solution in the reservoir regions ($L_0/2 < |x| \le L/2$) is of the form $\hat{p}_{res}(x,s) = A_{res} e^{m_0 x} + B_{res} e^{-m_0 x}$. The problem is symmetric with respect to $x=0$.

Task:
By applying the boundary and continuity conditions, and exploiting the symmetry of the problem which implies $A_C = B_C$, determine the coefficient $A_C$.

\subsubsection*{Ground Truth Solution}
\[ \boxed{A_C = -\frac{m_0 (\koff(p_0+q_0) + p_0 s)}{s(s+\koff+\kon)} \frac{e^{L_0 m_1/2} (e^{L m_0} - e^{L_0 m_0})}{(m_0-m_1)(e^{L m_0} - e^{L_0(m_0+m_1)}) + (m_0+m_1)(e^{L m_0 + L_0 m_1} - e^{L_0 m_0})}} \]

\subsubsection*{Model Outputs}
\subsubsection*{Model: o4-mini (Score: 0.5)}
\paragraph*{Model Solution:}
\[ A_C(s)\;=\;
-\;\frac{\bigl[\koff(p_0+q_0)+s\,p_0\bigr]\;m_0(s)\;\tanh\!\bigl(m_0(s)\,\tfrac{L-L_0}{2}\bigr)}
{2\;s\,(s+\kon+\koff)\,\Bigl[
\,m_1(s)\,\sinh\!\bigl(m_1(s)\,\tfrac{L_0}{2}\bigr)
\;+\;m_0(s)\,\tanh\!\bigl(m_0(s)\,\tfrac{L-L_0}{2}\bigr)\,
\cosh\!\bigl(m_1(s)\,\tfrac{L_0}{2}\bigr)
\Bigr]}\!, \]

\paragraph*{Judge's Evaluation:}

The ground truth expression is given by:
\[
A_C = -\frac{m_0 (\koff(p_0+q_0) + p_0 s)}{s(s+\koff+\kon)} \frac{e^{L_0 m_1/2} (e^{L m_0} - e^{L_0 m_0})}{(m_0-m_1)(e^{L m_0} - e^{L_0(m_0+m_1)}) + (m_0+m_1)(e^{L m_0 + L_0 m_1} - e^{L_0 m_0})}
\]

The model-generated expression is:
\[
A_C(s) = - \frac{\left[\koff(p_0+q_0) + s p_0\right] m_0(s) \tanh\left(m_0(s) \frac{L - L_0}{2}\right)}{2 s (s + \kon + \koff) \left[ m_1(s) \sinh\left(m_1(s) \frac{L_0}{2}\right) + m_0(s) \tanh\left(m_0(s) \frac{L - L_0}{2}\right) \cosh\left(m_1(s) \frac{L_0}{2}\right) \right]}
\]

---

**Comparison and Analysis:**

1. **Form of the numerator:**

- Ground truth numerator outside the fraction is:
\[
- m_0 (\koff(p_0+q_0) + p_0 s)
\]

- Model numerator is:
\[
- \left[\koff(p_0+q_0) + s p_0\right] m_0(s) \tanh\left(m_0(s) \frac{L - L_0}{2}\right)
\]

The model expression includes an extra factor of \(\tanh\left(m_0(s) \frac{L - L_0}{2}\right)\) in the numerator, which is not present explicitly in the ground truth numerator.

2. **Denominator outside the fraction:**

- Ground truth denominator outside the fraction is:
\[
s (s + \koff + \kon)
\]

- Model denominator outside the bracket is:
\[
2 s (s + \kon + \koff)
\]

The model has an extra factor of 2 in the denominator, which is not present in the ground truth.

3. **Fractional part:**

- Ground truth has a complicated fraction involving exponentials:
\[
\frac{e^{L_0 m_1/2} (e^{L m_0} - e^{L_0 m_0})}{(m_0 - m_1)(e^{L m_0} - e^{L_0(m_0 + m_1)}) + (m_0 + m_1)(e^{L m_0 + L_0 m_1} - e^{L_0 m_0})}
\]

- Model replaces this complicated fraction with a hyperbolic function expression:
\[
\frac{\tanh\left(m_0(s) \frac{L - L_0}{2}\right)}{m_1(s) \sinh\left(m_1(s) \frac{L_0}{2}\right) + m_0(s) \tanh\left(m_0(s) \frac{L - L_0}{2}\right) \cosh\left(m_1(s) \frac{L_0}{2}\right)}
\]

Note that the model expression has the \(\tanh\) factor in the numerator and a sum of hyperbolic functions in the denominator, which is structurally different from the ground truth's exponential fraction.

4. **Equivalence:**

- The ground truth expression is written in terms of exponentials, while the model expression is in terms of hyperbolic functions. Since hyperbolic functions can be expressed as combinations of exponentials, it is possible that the model expression is an equivalent form of the ground truth.

- However, the model expression has an extra factor of 2 in the denominator, which is not accounted for in the ground truth.

- The model expression also has the \(\tanh\) factor in the numerator, which is not explicitly present in the ground truth numerator but might correspond to the exponential fraction.

- Without further algebraic manipulation or context, it is difficult to confirm exact equivalence. The presence of the factor 2 in the denominator and the explicit \(\tanh\) factor in the numerator suggests a discrepancy.

---

**Conclusion:**

- The model expression is very close in structure and variables to the ground truth.

- The main differences are the factor of 2 in the denominator and the explicit \(\tanh\) factor in the numerator, which are not present in the ground truth.

- The model expression uses hyperbolic functions instead of exponentials, which could be equivalent, but the extra factor of 2 and the placement of \(\tanh\) suggest it is not exactly the same.

- These differences are more than a single minor error but less than completely wrong.

Therefore, the model expression is **partially correct**, but not fully equivalent.

---

**Score: 0.5**

---

**Final output:**

The model-generated expression is structurally similar and uses hyperbolic functions that could correspond to the exponential terms in the ground truth. However, the presence of an extra factor of 2 in the denominator and the explicit \(\tanh\) factor in the numerator, which do not appear in the ground truth, indicate that the expressions are not exactly equivalent. These are minor but significant discrepancies, so the expression is partially correct but not fully.

Score: 0.5

\newpage
\subsection*{Problem 41 (Paper: 2506.23867v1)}
\subsubsection*{Problem Statement}
Background:
Consider a one-dimensional system of total length $L$ containing $\mathcal{N}$ non-interacting particles that diffuse with a diffusion coefficient $D$. A central segment of length $L_0$, defined by $|x| \leq L_0/2$, represents a channel. Within this channel, particles can reversibly adsorb to an immobile state. The probability density for a particle to be at position $x$ in the diffusive state is $p(x,t)$, and in the adsorbed state is $q(x,t)$. Adsorption occurs with rate $k_{\rm on}$ and desorption with rate $k_{\rm off}$. The system is described by the following coupled equations:
\begin{equation*}
\begin{cases}
    \text{for } |x| \leq L_0/2: & \partial_t p = D \partial_{xx}p - k_{\rm on} p + k_{\rm off} q \\
    & \partial_t q = k_{\rm on} p - k_{\rm off} q \\
    \text{for } L_0/2 < |x| \leq L/2: & \partial_t p = D \partial_{xx}p
\end{cases}
\end{equation*}
The system has reflecting boundary conditions at $x=\pm L/2$, and both $p(x,t)$ and its flux $-D\partial_x p(x,t)$ are continuous at the channel edges $x=\pm L_0/2$. We consider the system at equilibrium, where the initial probability densities for a particle starting in the channel are $p_0 = 1/Z$ and $q_0 = (k_{\rm on}/k_{\rm off})/Z$, with partition function $Z = L + L_0(k_{\rm on}/k_{\rm off})$.

In Laplace space (with frequency variable $s$), the solution for the probability density of free particles inside the channel is given by:
\begin{equation*}
    \hat{p_C}(x,s) = A_C e^{m_1 x} + B_C e^{-m_1 x} + \frac{k_{\rm off}(p_0+q_0) + s p_0}{s(s+k_{\rm off}+k_{\rm on})}
\end{equation*}
where $m_0(s) = \sqrt{s/D}$ and $m_1(s) = \sqrt{\frac{s(s+k_{\rm off}+k_{\rm on})}{D(s+k_{\rm off})}}$. Due to the symmetry of the boundary conditions, $A_C = B_C$. The constant $A_C$ is found to be:
\begin{equation*}
    A_C = -\frac{m_0 (k_{\rm off}(p_0+q_0) + p_0 s)}{s(s+k_{\rm off}+k_{\rm on})} \frac{e^{L_0 m_1/2} (e^{L m_0} - e^{L_0 m_0})}{(m_0-m_1)(e^{L m_0} - e^{L_0(m_0+m_1)}) + (m_0+m_1)(e^{L m_0 + L_0 m_1} - e^{L_0 m_0})}
\end{equation*}
The power spectral density (PSD) of the fluctuations in the total number of particles in the channel, $S_N(f)$, can be found from the Laplace transform of the single-particle probability flux out of the channel. For $f \neq 0$, the relationship is $S_N(f) = 4\Re\left[\frac{2\mathcal{N}D}{s} \partial_x \hat{p_C}(x,s)|_{x=L_0/2}\right]$, with $s=2i\pi f$.

Task:
Using the expressions provided, derive the full expression for the power spectral density of the total number of particles in the channel, $S_N(f)$.

\subsubsection*{Ground Truth Solution}
\[ \boxed{S_N(f) = 8\Re\Bigg[\frac{\mathcal{N} m_1 (k_{\rm off}(p_0+q_0)+s p_0)}{s m_0 (s+k_{\rm off}+k_{\rm on})} \frac{(e^{L m_0}-e^{L_0 m_0})(1-e^{L_0 m_1})}{(m_0-m_1)(e^{L m_0}-e^{L_0(m_0+m_1)})+(m_0+m_1)(e^{L m_0 + L_0 m_1}-e^{L_0 m_0})}\Bigg]} \]

\subsubsection*{Model Outputs}
\subsubsection*{Model: o4-mini (Score: 0.0)}
\paragraph*{Model Solution:}
\[ S_{N}(f)
=\frac{8\,\mathcal{N}\,D\;m_{0}(s)\,m_{1}(s)}{Z\,s^{2}}
\;\Re\Bigg\{\,
\frac{\bigl(e^{L\,m_{0}(s)}-e^{L_{0}\,m_{0}(s)}\bigr)\,\bigl(e^{L_{0}\,m_{1}(s)}-1\bigr)}
{(m_{0}(s)-m_{1}(s))\bigl(e^{L\,m_{0}(s)}-e^{L_{0}\,\bigl(m_{0}(s)+m_{1}(s)\bigr)}\bigr)
\;+\;(m_{0}(s)+m_{1}(s))\bigl(e^{L\,m_{0}(s)+L_{0}\,m_{1}(s)}-e^{L_{0}\,m_{0}(s)}\bigr)}
\Bigg\}\,. \]

\paragraph*{Judge's Evaluation:}

Let's rewrite both expressions clearly and compare term-by-term.

---

**Ground Truth:**

\[
S_N(f) = 8 \Re \left[ \frac{\mathcal{N} m_1 (k_{\rm off}(p_0+q_0) + s p_0)}{s m_0 (s + k_{\rm off} + k_{\rm on})} \cdot \frac{(e^{L m_0} - e^{L_0 m_0})(1 - e^{L_0 m_1})}{(m_0 - m_1)(e^{L m_0} - e^{L_0 (m_0 + m_1)}) + (m_0 + m_1)(e^{L m_0 + L_0 m_1} - e^{L_0 m_0})} \right]
\]

---

**Model-Generated:**

\[
S_N(f) = \frac{8 \mathcal{N} D m_0(s) m_1(s)}{Z s^2} \Re \left\{ \frac{(e^{L m_0(s)} - e^{L_0 m_0(s)})(e^{L_0 m_1(s)} - 1)}{(m_0(s) - m_1(s))(e^{L m_0(s)} - e^{L_0 (m_0(s) + m_1(s))}) + (m_0(s) + m_1(s))(e^{L m_0(s) + L_0 m_1(s)} - e^{L_0 m_0(s)})} \right\}
\]

---

**Comparison:**

1. **Prefactor outside the fraction inside the real part:**

- Ground truth numerator inside the real part: \(\mathcal{N} m_1 (k_{\rm off}(p_0 + q_0) + s p_0)\)
- Model numerator: \(8 \mathcal{N} D m_0(s) m_1(s)\) divided by \(Z s^2\)

The ground truth has a complicated factor involving \(k_{\rm off}, p_0, q_0, s\), while the model has a factor \(D m_0(s) m_1(s) / (Z s^2)\).

These are not obviously equivalent. The ground truth has a factor \(\frac{k_{\rm off}(p_0 + q_0) + s p_0}{s (s + k_{\rm off} + k_{\rm on})}\), while the model has \(\frac{D m_0(s) m_1(s)}{Z s^2}\).

2. **Exponentials in numerator inside the real part:**

- Ground truth numerator: \((e^{L m_0} - e^{L_0 m_0})(1 - e^{L_0 m_1})\)
- Model numerator: \((e^{L m_0(s)} - e^{L_0 m_0(s)})(e^{L_0 m_1(s)} - 1)\)

The model has \((e^{L_0 m_1(s)} - 1)\) instead of \((1 - e^{L_0 m_1})\).

This is a sign difference:

\[
1 - e^{L_0 m_1} = - (e^{L_0 m_1} - 1)
\]

So the model's numerator differs by a minus sign in this factor.

3. **Denominator inside the real part:**

Both expressions have the same denominator structure:

\[
(m_0 - m_1)(e^{L m_0} - e^{L_0 (m_0 + m_1)}) + (m_0 + m_1)(e^{L m_0 + L_0 m_1} - e^{L_0 m_0})
\]

The model matches this exactly (with \(m_0(s), m_1(s)\) notation).

4. **Overall factor outside the real part:**

- Ground truth has \(8\) multiplied inside the real part.
- Model has \(8\) multiplied outside the real part.

This is equivalent since the real part is linear.

5. **Variables and notation:**

- The model uses \(m_0(s), m_1(s)\) instead of \(m_0, m_1\), which is acceptable.
- The model introduces \(D\) and \(Z\) which do not appear in the ground truth.
- The ground truth has a complicated factor involving \(k_{\rm off}, p_0, q_0, k_{\rm on}\), which is missing in the model.

---

**Summary:**

- The model expression differs significantly in the prefactor: it replaces the complicated fraction involving \(k_{\rm off}, p_0, q_0, k_{\rm on}\) with a factor involving \(D, Z, m_0, m_1\).
- The model has a sign difference in the numerator factor \((e^{L_0 m_1} - 1)\) vs \((1 - e^{L_0 m_1})\).
- The denominator matches exactly.
- The model's expression is not mathematically equivalent to the ground truth.

Because there are more than one minor errors (missing complicated factor, sign difference), this is not partially correct.

---

**Score:**

\[
\boxed{0}
\]

Score: 0

\newpage
\subsection*{Problem 42 (Paper: 2506.23886v2)}
\subsubsection*{Problem Statement}
Background:
Consider a tt*-structure $$(E, \eta, g, \Phi)$$ of rank $n+1$ over $$\mathbb{C}^*$$. A tt*-structure consists of a holomorphic vector bundle $$E$$ with a holomorphic structure $$\overline{\partial}_E$$, a holomorphic non-degenerate symmetric bilinear form $$\eta$$, a Hermitian metric $$g$$, and a holomorphic $$\operatorname{End}(E)$$-valued 1-form $$\Phi$$. These components are related by two key conditions:
1. The tt*-equation: $$F_D = -[\Phi, \Phi^{\dagger_g}]$$, where $$F_D$$ is the curvature of the Chern connection $$D = \partial_E^g + \overline{\partial}_E$$ associated with $$g$$, and $$\Phi^{\dagger_g}$$ is the adjoint of $$\Phi$$ with respect to $$g$$.
2. The reality condition: An involution $$\kappa$$ on $$E$$, defined by the relation $$g(a,b) = \eta(\kappa(a),b)$$ for any sections $$a,b \in \Gamma(E)$$, must satisfy $$\kappa^2 = \operatorname{Id}_E$$.

The structure is assumed to be a fixed point of the $$\mathbb{Z}_{n+1}$$-multiplication. This means there exists an isomorphism of tt*-structures $$\mathcal{T}: (E, \eta, g, \Phi) \to (E, \eta, g, \omega\Phi)$$, where $$\omega = e^{\sqrt{-1}\frac{2\pi}{n+1}}$$. We consider the case where $$\mathcal{T}^{n+1} = -\operatorname{Id}_E$$. It can be shown that there exists a holomorphic frame $$\tau = (\tau_0, \dots, \tau_n)$$ for the bundle $$E$$ in which the matrix representations of the structure's components have the following form:
- The Higgs field is $$\Phi_\tau = \operatorname{diag}(1, \omega, \dots, \omega^n) dt$$.
- The bilinear form is the identity matrix, $$\eta_\tau = I$$.
- The isomorphism $$\mathcal{T}$$ is represented by the matrix $$T = \begin{pmatrix} 0 & 1 & & \\ & \ddots & \ddots & \\ & & 0 & 1 \\ -1 & & & 0 \end{pmatrix}$$.
- The metric $$g$$ is preserved by $$\mathcal{T}$$, which implies its matrix representation $$G_\tau$$ in the $$\tau$$ frame commutes with $$T$$, i.e., $$TG_\tau=G_\tau T$$.

Task:
To analyze the system, one must find a new holomorphic frame $$e = (e_0, \dots, e_n)$$ where the physical equations become manifest. This new frame is related to the $$\tau$$ frame by a change of basis $$e = \tau \cdot M$$, where the matrix $$M$$ is given by the product $$M=LP$$. The matrix $$L$$ is defined as $$L = \frac{1}{\sqrt{n+1}} \operatorname{diag}(1, \omega^{-1/2}, \dots, \omega^{-n/2}) F$$, where $$F$$ is the standard Discrete Fourier Transform matrix with entries $$F_{jk} = \omega^{jk}$$ for $$j,k \in \{0, \dots, n\}$$. The matrix $$P$$ is the cyclic permutation matrix with entries $$P_{j, j-1} = 1$$ for $$j \in \{1, \dots, n\}$$ and $$P_{0,n}=1$$. In this new frame $$e$$, the metric becomes diagonal, with its matrix representation being $$G_e = \operatorname{diag}(e^{w_0}, e^{w_1}, \dots, e^{w_n})$$ for some real functions $$w_j(t, \overline{t})$$. Your task is to derive the complete system of equations governing these functions $$w_j$$. This system consists of a set of coupled partial differential equations arising from the tt*-equation and an algebraic anti-symmetry condition on the $$w_j$$ functions arising from the reality condition $$\kappa^2=\operatorname{Id}_E$$.

\subsubsection*{Ground Truth Solution}
\[ \boxed{\begin{cases}
(w_j)_{t \overline{t}} = e^{w_j - w_{j-1}} - e^{w_{j+1}-w_j}, \quad j=0,\dots,n \\ 
w_j + w_{n-j} = 0, \quad j=0,\dots,n
\end{cases}
\text{where } w_{-1} = w_n \text{ and } w_{n+1} = w_0.} \]

\subsubsection*{Model Outputs}
\subsubsection*{Model: o4-mini (Score: N/A)}
\paragraph*{Model Solution:}
\[ \begin{aligned}  
&\partial_{t}\partial_{\bar t}\,w_j  
\;=\;  
e^{\,w_j - w_{j-1}}\;-\;e^{\,w_{j+1}-w_j},  
\quad j=0,\dots,n,\\  
&w_{j+n+1}=w_j,  
\quad w_j = w_{n-j},  
\end{aligned} \]

\paragraph*{Judge's Evaluation:}

The ground truth expression is:
\[
\begin{cases}
(w_j)_{t \overline{t}} = e^{w_j - w_{j-1}} - e^{w_{j+1} - w_j}, \quad j=0,\dots,n \\
w_j + w_{n-j} = 0, \quad j=0,\dots,n
\end{cases}
\quad \text{where } w_{-1} = w_n \text{ and } w_{n+1} = w_0.
\]

The model-generated expression is:
\[
\begin{aligned}
&\partial_t \partial_{\bar t} w_j = e^{w_j - w_{j-1}} - e^{w_{j+1} - w_j}, \quad j=0,\dots,n, \\
&w_{j+n+1} = w_j, \\
&w_j = w_{n-j}.
\end{aligned}
\]

---

**Comparison and Justification:**

1. **Differential equation:**

- Both expressions have the same differential equation:
\[
\partial_t \partial_{\bar t} w_j = e^{w_j - w_{j-1}} - e^{w_{j+1} - w_j}.
\]
- The notation $(w_j)_{t \overline{t}}$ and $\partial_t \partial_{\bar t} w_j$ are equivalent.
- The index range $j=0,\dots,n$ matches.

2. **Boundary conditions / constraints:**

- Ground truth: \( w_j + w_{n-j} = 0 \).
- Model: \( w_j = w_{n-j} \).

This is a significant difference: the ground truth states an antisymmetric condition \( w_j = - w_{n-j} \), while the model states a symmetric condition \( w_j = w_{n-j} \).

3. **Periodicity conditions:**

- Ground truth: \( w_{-1} = w_n \) and \( w_{n+1} = w_0 \).
- Model: \( w_{j+n+1} = w_j \).

The model's periodicity condition is more general and consistent with the ground truth, since \( w_{-1} = w_n \) and \( w_{n+1} = w_0 \) imply periodicity with period \( n+1 \). The model's condition \( w_{j+n+1} = w_j \) is equivalent to this.

---

**Summary:**

- The differential equation and periodicity conditions are correct.
- The key error is in the symmetry condition: the model has \( w_j = w_{n-j} \) instead of \( w_j + w_{n-j} = 0 \), i.e., the sign is wrong.

This is a **sign error** in the symmetry condition, which is a crucial part of the problem.

---

**Score:**

- Since there is one significant sign error in the symmetry condition, but the rest is correct, this fits the rubric for **Score 0.5 (Partially Correct)**.

---

**Final answer:**

\[
\boxed{
\text{Score: } 0.5
}
\]

\newpage
\subsection*{Problem 43 (Paper: 2506.23890v1)}
\subsubsection*{Problem Statement}
Background:  
A pseudospherical surface is a two-dimensional Riemannian manifold with constant negative Gaussian curvature $\mathcal{K}=-1$. Its geometry can be described locally by a set of three 1-forms, $\omega_1, \omega_2, \omega_3$, which satisfy the Maurer-Cartan structure equations:
\begin{align*}
d\omega_1 &= \omega_3\wedge\omega_2 \\
d\omega_2 &= \omega_1\wedge\omega_3 \\
d\omega_3 &= \omega_1\wedge\omega_2
\end{align*}
For a 1-form $\alpha = P(x,t)dx + Q(x,t)dt$, the exterior derivative is $d\alpha = (\partial_x Q - \partial_t P)dx \wedge dt$. The wedge product is bilinear and antisymmetric, with the properties $dx \wedge dt = -dt \wedge dx$ and $dx \wedge dx = dt \wedge dt = 0$. Consider a set of 1-forms that depend on a real-valued function $u(x,t)$, its derivatives, and a real, non-zero parameter $\lambda$. Let $m = u - u_{xx}$. The forms are given by:
\begin{align*}
\omega_1 &= \left(\frac{\lambda}{2}+\frac{1}{2\lambda}-m\right)dx+\left(um+\frac{\lambda}{2}u-\frac{u}{2\lambda}-\frac{1}{2}-\frac{\lambda^2}{2}\right)dt \\
\omega_2 &= -u_x dt \\
\omega_3 &= \left(m+\frac{1}{2\lambda}-\frac{\lambda}{2}\right)dx+\left(\frac{\lambda^2}{2}-\frac{1}{2}-\frac{u}{2\lambda}-\frac{\lambda}{2}u-um\right)dt
\end{align*}

Task:
For this system, the structure equations impose a constraint on the function $u(x,t)$. Derive the partial differential equation for $u(x,t)$ that results from the first structure equation, $d\omega_1=\omega_3\wedge\omega_2$. Your final answer should be expressed solely in terms of $u$ and its partial derivatives.

\subsubsection*{Ground Truth Solution}
\[ \boxed{u_t-u_{txx}+3uu_x=2u_xu_{xx}+uu_{xxx}} \]

\subsubsection*{Model Outputs}
\subsubsection*{Model: o4-mini (Score: 1.0)}
\paragraph*{Model Solution:}
\[ u_t \;-\; u_{xxt} \;+\;3\,u\,u_x \;-\;2\,u_x\,u_{xx}\;-\;u\,u_{xxx}
\;=\;0. \]

\paragraph*{Judge's Evaluation:}

First, let's rewrite both expressions clearly:

**Ground Truth:**
\[
u_t - u_{txx} + 3 u u_x = 2 u_x u_{xx} + u u_{xxx}
\]

**Model-Generated:**
\[
u_t - u_{xxt} + 3 u u_x - 2 u_x u_{xx} - u u_{xxx} = 0
\]

---

**Step 1: Check the derivatives \( u_{txx} \) vs. \( u_{xxt} \)**

Since partial derivatives commute (assuming sufficient smoothness), 
\[
u_{txx} = u_{xxt}.
\]
So this is not an error.

---

**Step 2: Rearrange the ground truth to the form of the model-generated expression**

Bring all terms to one side:
\[
u_t - u_{txx} + 3 u u_x - 2 u_x u_{xx} - u u_{xxx} = 0.
\]

This matches exactly the model-generated expression, except the model uses \( u_{xxt} \) instead of \( u_{txx} \), which is equivalent.

---

**Step 3: Check signs and coefficients**

- The signs and coefficients of all terms match exactly.
- The model expression is just the ground truth rearranged to equal zero.

---

**Conclusion:**

The model-generated expression is mathematically equivalent to the ground truth expression. The difference in the order of partial derivatives is not an error.

---

**Score:**

\[
\boxed{1}
\]

Score: 1

\newpage
\subsection*{Problem 44 (Paper: 2506.23890v1)}
\subsubsection*{Problem Statement}
Background:
Consider the Camassa-Holm (CH) equation for a real-valued function $u(x,t)$:
$$u_t - u_{txx} + 3uu_x = 2u_xu_{xx} + uu_{xxx}$$
Let $u(x,t)$ be a non-trivial solution to the CH equation, existing for a time interval $t \in [0, T)$. This solution arises from an initial condition $u(x,0) = u_0(x)$ belonging to a suitable function space (specifically, the Sobolev space $H^s(\mathbb{R})$ with $s \ge 3$). For any fixed time $t$ in the interval of existence, the solution is known to possess the following properties:
1. The functions $x \mapsto u(x,t)$ and its spatial derivative $x \mapsto u_x(x,t)$ are continuous over $\mathbb{R}$.
2. The solution and its derivative vanish at spatial infinity: $\lim_{|x|\to\infty} u(x,t) = 0$ and $\lim_{|x|\to\infty} u_x(x,t) = 0$.
3. The $H^1(\mathbb{R})$ norm, defined as $\|u(\cdot,t)\|_{H^1}^2 = \int_{-\infty}^{\infty} [u(x,t)^2 + u_x(x,t)^2] dx$, is a conserved quantity that is strictly positive, i.e., $\|u(\cdot,t)\|_{H^1}^2 = \|u_0\|_{H^1}^2 > 0$.

Due to these properties, the infimum and supremum of the solution's slope are attained for each $t$. Let us define these extrema as:
$$I(t) = \inf_{x \in \mathbb{R}} u_x(x,t)$$
$$S(t) = \sup_{x \in \mathbb{R}} u_x(x,t)$$

Task:
Based on the provided properties of the solution $u(x,t)$, derive a strict inequality that characterizes the product $h(t) = I(t)S(t)$ for all $t \in [0, T)$.

\subsubsection*{Ground Truth Solution}
\[ \boxed{h(t) < 0} \]

\subsubsection*{Model Outputs}
\subsubsection*{Model: o4-mini (Score: N/A)}
\paragraph*{Model Solution:}
\[ I(t)\,S(t)\;<\;0\quad\text{for all }0\le t<T. \]

\paragraph*{Judge's Evaluation:}

The ground truth expression is:
\[
h(t) < 0
\]

The model-generated expression is:
\[
I(t) S(t) < 0 \quad \text{for all } 0 \le t < T.
\]

**Comparison:**

- The ground truth expression involves a single function \( h(t) \) being less than zero.
- The model-generated expression involves the product of two functions \( I(t) \) and \( S(t) \) being less than zero, and it also specifies the domain \( 0 \le t < T \).
- The model expression is not a simple renaming of \( h(t) \) but a product of two different functions, which is not equivalent to the ground truth.
- The domain specification in the model expression is additional information, but the core inequality is different.
- Since the model expression is not mathematically equivalent to the ground truth expression, it cannot be scored as fully correct.
- There is more than one difference: the function is different (product vs single function), and the domain is specified in the model but not in the ground truth.
- These differences are significant, not minor.

**Conclusion:**

The model expression is incorrect relative to the ground truth.

\[
\boxed{
\text{Score: } 0
}
\]

\newpage
\subsection*{Problem 45 (Paper: 2506.23894v1)}
\subsubsection*{Problem Statement}
Background:
Consider a flow network defined by an undirected graph $G=(V(G), E(G))$, a capacity function $c: E(G) \to \mathbb{R}_+$, a source vertex $s \in V(G)$, and a sink vertex $t \in V(G)$. The flow is defined on the associated bi-directed graph $\overset{\rightleftharpoons}{G}$, where each undirected edge $e_{xy} \in E(G)$ corresponds to two directed edges $(x, y)$ and $(y, x)$ in $E(\overset{\rightleftharpoons}{G})$ with capacity $c_{(x,y)} = c_{(y,x)} = c_{e_{xy}}$. A flow is a function $f: E(\overset{\rightleftharpoons}{G}) \to \mathbb{R}$ satisfying:
1.  Skew symmetry: $f(x, y) = -f(y, x)$ for all $(x, y) \in E(\overset{\rightleftharpoons}{G})$.
2.  Capacity constraint: $f(x, y) \leq c_{(x, y)}$ for all $(x, y) \in E(\overset{\rightleftharpoons}{G})$.
3.  Flow conservation: $\sum_{y:(x, y)\in E(\overset{\rightleftharpoons}{G})} f(x, y) = 0$ for all $x \in V(G) \setminus \{s, t\}$.
The value of the flow is $|f|_c = \sum_{y:(s, y)\in E(\overset{\rightleftharpoons}{G})} f(s, y)$. A flow $f$ is a \emph{maximum flow} (max-flow) if its value is maximal. A flow $f$ induces a partial order $\leq_f$ on $V(G)$, defined as the transitive closure of the relation $x R_f y \iff f(x, y) > -c_{(x, y)}$. An $s$--$t$ cut is a bipartition $(S, T)$ of $V(G)$ such that $s \in S$ and $t \in T$. The capacity of the cut is $|(S, T)|_c = \sum_{(x, y) \in E(\overset{\rightleftharpoons}{G}), x \in S, y \in T} c_{(x, y)}$. A cut is a \emph{minimum cut} (min-cut) if its capacity is minimal. The max-flow min-cut theorem states that the value of a max-flow equals the capacity of a min-cut. For any vertex $x \in V(G)$, define the set $S_x$ as the intersection of all sets $S$ from min-cuts $(S, T)$ that contain $x$:
$S_x \coloneqq \bigcap_{(S, T): \text{min-cut and } x\in S} S$, where an empty intersection is defined as $V(G)$. A subset of vertices $C \subseteq V(G)$ is called a \emph{closure} for the partial order $\leq_f$ if for any $y \in C$, the condition $x \leq_f y$ implies $x \in C$. You are given the following property connecting closures and min-cuts for any max-flow $f$: a set $C \subseteq V(G)$ with $s \in C$ and $t \notin C$ is a closure for $\leq_f$ if and only if $(C, \Bar{C})$ is a minimum $s$--$t$ cut.

Task:
For an arbitrary max-flow $f$ and an arbitrary vertex $y \in V(G)$, derive the expression for the set $S_y$ in terms of the vertices of the graph $V(G)$ and the partial order $\leq_f$.

\subsubsection*{Ground Truth Solution}
\[ \boxed{S_y = \{x \in V(G) \mid x \leq_f y\}} \]

\subsubsection*{Model Outputs}
\subsubsection*{Model: o4-mini (Score: 1.0)}
\paragraph*{Model Solution:}
\[ S_y \;=\;\{\,x\in V(G)\mid x\le_f y\} \]

\paragraph*{Judge's Evaluation:}

The ground truth expression is:
\[
S_y = \{x \in V(G) \mid x \leq_f y\}
\]

The model-generated expression is:
\[
S_y \;=\;\{\,x\in V(G)\mid x\le_f y\}
\]

Comparing the two:

- Both expressions define the set \( S_y \) as the set of vertices \( x \) in \( V(G) \) such that \( x \leq_f y \).
- The only difference is the spacing and the use of \(\leq_f\) versus \(\le_f\).
- The notation \(\leq_f\) and \(\le_f\) are equivalent in meaning; the subscript \(f\) applies to the relation \(\leq\).
- The spacing differences (extra spaces or lack thereof) do not affect mathematical correctness.
- Variable names and symbols match exactly.

Therefore, the model-generated expression is mathematically equivalent and matches the ground truth expression exactly, aside from minor formatting differences.

**Score: 1**

\newpage
\subsection*{Problem 46 (Paper: 2506.23894v1)}
\subsubsection*{Problem Statement}
Background:
In a certain class of physical models, the statistical properties of observables can be described by a probability measure $\mu_G$ associated with a directed acyclic graph $G$. Consider a graph $G$ that is a \emph{series composition} of two subgraphs, $H_1$ and $H_2$, denoted $G = H_1 \mathbin{\text{series}} H_2$. The graph $G$ is formed by identifying the sink of $H_1$ with the source of $H_2$; let this junction vertex be $v$. The overall source of $G$ is the source of $H_1$, denoted $s$, and the overall sink of $G$ is the sink of $H_2$, denoted $t$. Associated with each subgraph $H_i$ is a measure $\mu_{H_i}$ and a positive real parameter $r_{H_i}$. The composite graph $G$ has an associated parameter $r_G = \min(r_{H_1}, r_{H_2})$.

The moments of these measures are related to concepts from free probability. For a measure $\mu$ and a permutation $\pi \in \mathcal{S}_n$, the extended moment $\varphi_\pi(\mu)$ is defined as $\varphi_\pi(\mu) = \prod_{\sigma: \text{cycle of } \pi} \int x^{\mathrm{card}(\sigma)} d\mu(x)$. The ordinary $n$-th moment is $\varphi_n(\mu) = \varphi_\gamma(\mu)$, where $\gamma = (1 \ \cdots \ n)$ as a cycle. These moments are related to free cumulants $\kappa_\pi(\mu)$ by the moment-cumulant formula: $\varphi_\pi(\mu) = \sum_{\alpha \le_{\mathrm{nc}} \pi} \kappa_\alpha(\mu)$, where $\alpha \le_{\mathrm{nc}} \pi$ means that the cycle partition of $\alpha$ is a non-crossing refinement of the cycle partition of $\pi$.

The $n$-th moment of the composite measure $\mu_G$ is given by the following expression, which sums over all permutations $\beta_v \in \mathcal{S}_n$ at the junction vertex $v$ that are intermediate in the non-crossing partial order between the identity permutation at the source and $\gamma = (1 \cdots n)$ at the sink:
\begin{equation*}
\varphi_n(\mu_G) = \sum_{\mathrm{id}\, \le_{\mathrm{nc}} \, \beta_v \, \le_{\mathrm{nc}} \, \gamma} \left(\frac{r_{H_1}}{r_G}\right)^{\#\beta_v-n} \, \varphi_{\beta_v}(\mu_{H_1}) \, \cdot \, \left(\frac{r_{H_2}}{r_G}\right)^{\#(\beta_v^{-1}\gamma)-n} \, \varphi_{\beta_v^{-1}\gamma}(\mu_{H_2})
\end{equation*}
where $\#\pi$ is the number of cycles in permutation $\pi$.

You are given the following identities from free probability theory:
1. The cumulants of a free multiplicative convolution $\mu_1 \boxtimes \mu_2$ are given by 
$\kappa_n(\mu_1 \boxtimes \mu_2) = \sum_{\alpha \le_{\mathrm{nc}} \gamma} \kappa_\alpha(\mu_1) \, \kappa_{\alpha^{-1}\gamma}(\mu_2)$.
2. For a measure $\mu$, a scalar $r \ge 1$, and a permutation $\pi \in \mathcal{S}_n$, the following relation holds:
\begin{equation*}
r^{\#\pi - n} \, \varphi_\pi(\mu) = \kappa_\pi(\hat{\pi}_r \boxtimes (r^{-1} \odot \mu)^{\boxplus r})
\end{equation*}
where $\boxplus$ is the free additive convolution, $\odot$ is a scaling operation on measures, and $\hat{\pi}_r$ is the rescaled Marchenko-Pastur distribution.

For this problem, the following explicit definitions apply:
- \textbf{Rescaled Marchenko-Pastur distribution} $\hat{\pi}_r$ denotes the Marchenko-Pastur law (also known as the free Poisson distribution) with parameter $r > 0$ and unit scale. Its $n$-th moment is given by $\sum_{k=1}^n r^k \mathrm{Cat}_{n,k}$, where $\mathrm{Cat}_{n,k}$ counts non-crossing partitions of $[n]$ into $k$ blocks; the explicit density for $r \geq 1$ is $\frac{\sqrt{(\lambda_+ - x)(x - \lambda_-)}}{2 \pi x} \mathbf{1}_{[\lambda_-,\lambda_+]}(x) dx$ with $\lambda_\pm = (1 \pm \sqrt{r})^2$.
- \textbf{Scaling of a measure} $a \odot \mu$ is defined as the push-forward of $\mu$ under the map $x \mapsto a x$, i.e., if $X \sim \mu$ then $a X \sim a \odot \mu$. For moments, $\int x^k d(a \odot \mu)(x) = a^k \int x^k d\mu(x)$.
- \textbf{Free additive convolution} $\mu_1 \boxplus \mu_2$ is the law of $X_1 + X_2$ for freely independent $X_1 \sim \mu_1$, $X_2 \sim \mu_2$.
- \textbf{Free multiplicative convolution} $\mu_1 \boxtimes \mu_2$ is the law of $X_1 X_2$ for freely independent $X_1 \sim \mu_1$, $X_2 \sim \mu_2$.

Task:
Let $r_{\max} = \max(r_{H_1}, r_{H_2})$ and $r_{\min} = \min(r_{H_1}, r_{H_2})$. Using the provided relations and the explicit definitions above, derive an expression for the $n$-th moment $\varphi_n(\mu_G)$ of the composite measure. Express your final answer as the $n$-th moment of a single, new measure constructed from $\mu_{H_1}$, $\mu_{H_2}$, $r_{H_1}$, $r_{H_2}$, $r_{\max}$, and $r_{\min}$ using the operations of free probability ($\boxplus, \boxtimes, \odot$) and the measure $\hat{\pi}_r$.


\subsubsection*{Ground Truth Solution}
\[ \boxed{\varphi_n\left( \left(\frac{r_{\min}}{r_{H_1}}\odot\mu_{H_1}\right)^{\boxplus\frac{r_{H_1}}{r_{\min}}} \boxtimes \hat{\pi}_{\frac{r_{\max}}{r_{\min}}} \boxtimes \left(\frac{r_{\min}}{r_{H_2}}\odot\mu_{H_2}\right)^{\boxplus\frac{r_{H_2}}{r_{\min}}} \right)} \]

\subsubsection*{Model Outputs}
\subsubsection*{Model: o4-mini (Score: 0.0)}
\paragraph*{Model Solution:}
\[ \varphi_{n}(\mu_{G})
\;=\;
\varphi_{n}\!\bigl(\mu_{\rm new}\bigr)
\;=\;
\varphi_{n}\!\Bigl(\,
\hat\pi_{\,r_{\min}}
\;\BOXTIMES\;
\bigl(\tfrac{r_{H_1}}{r_{\min}}\ODOT\mu_{H_1}\bigr)^{\!\BOXPLUS\,(r_{H_1}/r_{\min})}
\;\BOXTIMES\;
\bigl(\tfrac{r_{H_2}}{r_{\min}}\ODOT\mu_{H_2}\bigr)^{\!\BOXPLUS\,(r_{H_2}/r_{\min})}
\Bigr)\,. \]

\paragraph*{Judge's Evaluation:}

Let's rewrite both expressions clearly for comparison:

**Ground Truth:**
\[
\varphi_n\left( \left(\frac{r_{\min}}{r_{H_1}} \odot \mu_{H_1}\right)^{\boxplus \frac{r_{H_1}}{r_{\min}}} \boxtimes \hat{\pi}_{\frac{r_{\max}}{r_{\min}}} \boxtimes \left(\frac{r_{\min}}{r_{H_2}} \odot \mu_{H_2}\right)^{\boxplus \frac{r_{H_2}}{r_{\min}}} \right)
\]

**Model-Generated:**
\[
\varphi_n\left(
\hat{\pi}_{r_{\min}} \boxtimes \left(\frac{r_{H_1}}{r_{\min}} \odot \mu_{H_1}\right)^{\boxplus \frac{r_{H_1}}{r_{\min}}} \boxtimes \left(\frac{r_{H_2}}{r_{\min}} \odot \mu_{H_2}\right)^{\boxplus \frac{r_{H_2}}{r_{\min}}}
\right)
\]

---

**Comparison:**

1. **Order and presence of terms:**

- Ground truth has the order:
\[
\left(\frac{r_{\min}}{r_{H_1}} \odot \mu_{H_1}\right)^{\boxplus \frac{r_{H_1}}{r_{\min}}} \boxtimes \hat{\pi}_{\frac{r_{\max}}{r_{\min}}} \boxtimes \left(\frac{r_{\min}}{r_{H_2}} \odot \mu_{H_2}\right)^{\boxplus \frac{r_{H_2}}{r_{\min}}}
\]

- Model has:
\[
\hat{\pi}_{r_{\min}} \boxtimes \left(\frac{r_{H_1}}{r_{\min}} \odot \mu_{H_1}\right)^{\boxplus \frac{r_{H_1}}{r_{\min}}} \boxtimes \left(\frac{r_{H_2}}{r_{\min}} \odot \mu_{H_2}\right)^{\boxplus \frac{r_{H_2}}{r_{\min}}}
\]

2. **Differences in coefficients inside the powers:**

- Ground truth uses \(\frac{r_{\min}}{r_{H_1}}\) and \(\frac{r_{\min}}{r_{H_2}}\) inside the \(\odot\) operation.

- Model uses \(\frac{r_{H_1}}{r_{\min}}\) and \(\frac{r_{H_2}}{r_{\min}}\) inside the \(\odot\) operation.

This is a reciprocal difference, which is significant.

3. **Difference in the \(\hat{\pi}\) term:**

- Ground truth has \(\hat{\pi}_{\frac{r_{\max}}{r_{\min}}}\).

- Model has \(\hat{\pi}_{r_{\min}}\).

This is a clear mismatch in the index of \(\hat{\pi}\).

4. **Order of terms:**

- Ground truth: \(\left(\cdots\right)^{\boxplus} \boxtimes \hat{\pi} \boxtimes \left(\cdots\right)^{\boxplus}\).

- Model: \(\hat{\pi} \boxtimes \left(\cdots\right)^{\boxplus} \boxtimes \left(\cdots\right)^{\boxplus}\).

Order is different, but since \(\boxtimes\) might be associative and commutative, order may not be critical unless specified.

---

**Summary:**

- The model expression has two significant coefficient inversions inside the powers.

- The model expression has a different \(\hat{\pi}\) index.

- These are more than one minor error.

Therefore, the model expression is **incorrect**.

---

**Score: 0**

\newpage
\subsection*{Problem 47 (Paper: 2506.23939v2)}
\subsubsection*{Problem Statement}
Background:
A physical system is described by the overdamped linear time-delayed Langevin equation for a variable $x(t)$:
$$ \dot{x}(t) = ax(t) + bx(t-\tau) + \xi(t) $$
Here, $a$ and $b$ are real constants, $\tau > 0$ is a constant time delay, and $\xi(t)$ is a white Gaussian noise source characterized by temperature $T$, with $\langle \xi(t) \rangle = 0$ and $\langle \xi(t)\xi(t') \rangle = 2T\delta(t-t')$. The variance of the system, $\text{Var}[x(t)] = \langle x^2(t) \rangle - \langle x(t) \rangle^2$, can be expressed as:
$$ \text{Var}[x(t)] = 2T \int_0^t x_0^2(s)ds $$
where $x_0(t)$ is the fundamental solution of the deterministic part of the equation, i.e., $\dot{x}_0(t) = ax_0(t) + bx_0(t-\tau)$ for $t>0$, with the specific initial conditions $x_0(t)=0$ for $t<0$ and $x_0(0)=1$. The behavior of $x_0(t)$ is governed by the roots $\{S_k\}$ of the characteristic equation $S - a - be^{-S\tau} = 0$. In the long-time limit, the behavior is dominated by the roots with the largest real part. The fundamental solution can be formally expressed as a sum over these roots:
$$ x_0(t) = \sum_{k=-\infty}^{\infty} \frac{e^{S_k t}}{1+b\tau e^{-S_k \tau}} $$
We consider a system at a specific critical point known as "oscillatory criticality". This occurs when the parameters satisfy the conditions $a+b<0$, $a-b>0$, and $\tau = \frac{\arccos(-a/b)}{\sqrt{b^2-a^2}}$. Under these conditions, the characteristic equation possesses exactly one pair of purely imaginary roots, $S_{\pm} = \pm i\omega_c$ where $\omega_c = \sqrt{b^2-a^2}$, while all other roots have strictly negative real parts.

Task:
For a system at oscillatory criticality, determine the asymptotic expression for the variance $\text{Var}[x(t)]$ in the long-time limit ($t \to \infty$). Your final expression should be given in terms of $t, T, a, b, \tau, \omega_c$, and a phase angle $\theta$ defined as $\theta = \arctan\left(-\frac{b\tau\sin(\omega_c\tau)}{1+b\tau\cos(\omega_c\tau)}\right)$. The contribution from the stable modes (those with $\text{Re}[S_k]<0$) can be collected into an unspecified finite constant $C_1$.

\subsubsection*{Ground Truth Solution}
\[ \boxed{\langle x^2(t)\rangle - \langle x(t)\rangle^2 = \frac{4 T \bigg[ t-\frac{1}{2\omega_c}\bigg( \sin[2(\omega_ct+\theta)]-\sin(2\theta) \bigg) \bigg]}{\bigg(1+b\tau\cos(\omega_c\tau)\bigg)^2+b^2\tau^2\sin^2(\omega_c\tau)} + C_1} \]

\subsubsection*{Model Outputs}
\subsubsection*{Model: o4-mini (Score: 0.0)}
\paragraph*{Model Solution:}
\[ \Var[x(t)]\;\simeq\;
\frac{2T\,t}{\bigl|1+b\tau e^{-i\omega_{c}\tau}\bigr|^{2}}
\;+\;\frac{T}{\omega_{c}\,\bigl|1+b\tau e^{-i\omega_{c}\tau}\bigr|^{2}}\,
\sin\!\bigl(2\omega_{c}t-2\theta\bigr)\;+\;C_{1}\,. \]

\paragraph*{Judge's Evaluation:}

The ground truth expression is:
\[
\langle x^2(t)\rangle - \langle x(t)\rangle^2 = \frac{4 T \left[ t - \frac{1}{2\omega_c} \left( \sin\big(2(\omega_c t + \theta)\big) - \sin(2\theta) \right) \right]}{\left(1 + b \tau \cos(\omega_c \tau)\right)^2 + b^2 \tau^2 \sin^2(\omega_c \tau)} + C_1
\]

The model-generated expression is:
\[
\operatorname{Var}[x(t)] \simeq \frac{2 T t}{\left|1 + b \tau e^{-i \omega_c \tau}\right|^2} + \frac{T}{\omega_c \left|1 + b \tau e^{-i \omega_c \tau}\right|^2} \sin\big(2 \omega_c t - 2 \theta\big) + C_1
\]

---

**Comparison and analysis:**

1. **Denominator equivalence:**

\[
\left|1 + b \tau e^{-i \omega_c \tau}\right|^2 = (1 + b \tau \cos(\omega_c \tau))^2 + (b \tau \sin(\omega_c \tau))^2
\]

This matches exactly the denominator in the ground truth.

2. **Structure of the numerator:**

- Ground truth numerator inside the bracket:
\[
4 T \left[ t - \frac{1}{2 \omega_c} \left( \sin(2 \omega_c t + 2 \theta) - \sin(2 \theta) \right) \right]
\]

- Model numerator is split into two terms:
\[
\frac{2 T t}{\ldots} + \frac{T}{\omega_c \ldots} \sin(2 \omega_c t - 2 \theta)
\]

3. **Differences in the sine terms:**

- Ground truth has:
\[
\sin(2 \omega_c t + 2 \theta) - \sin(2 \theta)
\]

- Model has only:
\[
\sin(2 \omega_c t - 2 \theta)
\]

The model expression is missing the constant sine term \(- \sin(2 \theta)\) inside the bracket, which is important for the exact expression.

4. **Coefficients:**

- Ground truth has an overall factor of 4 in numerator.
- Model splits into terms with coefficients 2 and 1, which sum to 4 if combined properly, but the sine term is not matching in sign or phase.

5. **Phase difference:**

- Ground truth sine argument: \(2(\omega_c t + \theta) = 2 \omega_c t + 2 \theta\)
- Model sine argument: \(2 \omega_c t - 2 \theta\)

This is a phase difference of \(4 \theta\), which is significant and not a trivial variable renaming.

6. **Missing term:**

- The model expression lacks the \(- \sin(2 \theta)\) term entirely.

---

**Conclusion:**

- The denominator is correct.
- The model expression is close in form but misses the constant sine term \(- \sin(2 \theta)\).
- The sine term has a wrong phase (sign in front of \(\theta\)).
- The overall factor 4 is split differently but could be consistent if the sine term were correct.
- Because of the missing sine term and phase difference, the model expression is not mathematically equivalent to the ground truth.

This constitutes more than one minor error (missing term and wrong phase), so partial credit is not justified.

---

**Score:**

Score: 0

\newpage
\subsection*{Problem 48 (Paper: 2506.23939v2)}
\subsubsection*{Problem Statement}
Background:
Consider a particle described by an overdamped linear time-delayed Langevin equation $\dot{x}(t) = a x(t) + b x(t-\tau) + \xi(t)$, where $a$ and $b$ are force constants, $\tau > 0$ is a fixed time delay, and $\xi(t)$ is Gaussian white noise with $\langle \xi(t) \rangle = 0$ and $\langle \xi(t)\xi(t') \rangle = 2T\delta(t-t')$, where $T$ is the temperature. The system is prepared with an initial history of $x(t) = 0$ for all $t \le 0$, which implies that the mean displacement $\langle x(t) \rangle = 0$ for all $t > 0$. The system is at a state of \emph{oscillatory criticality}, where the parameters satisfy the conditions $a+b<0$, $a-b>0$, and $\tau=\dfrac{\arccos(-a/b)}{\omega_c}$ with $\omega_c = \sqrt{b^2-a^2}$. In this regime, the variance for large $t$ is given by $\langle x^2(t)\rangle \approx 2T\int_0^t x_0^2(s)ds$, where $x_0(t)$ is the fundamental solution. The asymptotic form of the fundamental solution is given by
$x_0(t) \approx \dfrac{2\sin(\omega_c t+\theta)}{\sqrt{\left(1+b\tau\cos(\omega_c\tau)\right)^2+b^2\tau^2\sin^2(\omega_c\tau)}}$,
where $\theta=\arctan\left(-\dfrac{b\tau\sin(\omega_c\tau)}{1+b\tau\cos(\omega_c\tau)} \right)$. The average rate of heat dissipation into the environment is given by the exact relation $\left\langle \dfrac{dq}{dt} \right\rangle=b^2\langle x^2(t-\tau)\rangle-a^2\langle x^2(t) \rangle + a\dfrac{d\langle x^2(t)\rangle}{dt} -aT$.

Task:
Derive the asymptotic expression for the average heat dissipation rate $\langle dq/dt \rangle$ as $t \rightarrow \infty$. Express your answer in terms of $a, b, T, t, \omega_c, \tau, \theta$, and the constant $A_0 \equiv \dfrac{4 T}{\left( 1 + b\tau \cos(\omega_c\tau)\right)^2+b^2\tau^2\sin^2(\omega_c\tau)}$. All time-independent terms should be collected into a single additive constant $B_0$.

\subsubsection*{Ground Truth Solution}
\[ \boxed{\left\langle \frac{dq}{dt}\right\rangle = (b^2-a^2)A_0 t - \frac{A_0}{2\omega_c}\bigg\{ b^2\sin [2(\omega_c(t-\tau)+\theta)] -a^2\sin [2(\omega_ct+\theta)]\bigg\} + aA_0\bigg( 1-\cos[2(\omega_ct+\theta)] \bigg)+B_0} \]

\subsubsection*{Model Outputs}
\subsubsection*{Model: o4-mini (Score: 0.0)}
\paragraph*{Model Solution:}
\[ \bigl\langle\tfrac{dq}{dt}\bigr\rangle
\;\simeq\;
A_{0}\,\omega_{c}^{2}\,t
\;+\;\frac{A_{0}\,\omega_{c}}{2}\,\sin\!\bigl(2\omega_{c}t+2\theta\bigr)
\;-\;2\,a\,A_{0}\,\cos\!\bigl(2\omega_{c}t+2\theta\bigr)
\;+\;B_{0}\,. \]

\paragraph*{Judge's Evaluation:}

The ground truth expression is:
\[
\left\langle \frac{dq}{dt}\right\rangle = (b^2 - a^2) A_0 t - \frac{A_0}{2 \omega_c} \left\{ b^2 \sin \big[ 2(\omega_c (t-\tau) + \theta) \big] - a^2 \sin \big[ 2(\omega_c t + \theta) \big] \right\} + a A_0 \left( 1 - \cos \big[ 2(\omega_c t + \theta) \big] \right) + B_0
\]

The model-generated expression is:
\[
\left\langle \frac{dq}{dt} \right\rangle \simeq A_0 \omega_c^2 t + \frac{A_0 \omega_c}{2} \sin(2 \omega_c t + 2 \theta) - 2 a A_0 \cos(2 \omega_c t + 2 \theta) + B_0
\]

**Comparison and analysis:**

1. **Linear in time term:**
   - Ground truth: $(b^2 - a^2) A_0 t$
   - Model: $A_0 \omega_c^2 t$
   
   These are clearly different. The model replaces $(b^2 - a^2)$ with $\omega_c^2$, which is not justified or equivalent unless $b^2 - a^2 = \omega_c^2$, which is not stated.

2. **Sinusoidal terms:**
   - Ground truth has two sine terms with arguments $2(\omega_c (t-\tau) + \theta)$ and $2(\omega_c t + \theta)$, multiplied by $b^2$ and $a^2$ respectively, and divided by $2 \omega_c$.
   - Model has a single sine term $\frac{A_0 \omega_c}{2} \sin(2 \omega_c t + 2 \theta)$.
   
   The model expression omits the $b^2$ term and the time delay $\tau$ in the sine argument, and the coefficients differ significantly.

3. **Cosine term:**
   - Ground truth: $a A_0 (1 - \cos[2(\omega_c t + \theta)])$
   - Model: $-2 a A_0 \cos(2 \omega_c t + 2 \theta)$
   
   The model misses the constant term $a A_0$ (the "1" inside the parentheses), and the coefficient differs by a factor of 2 and a sign difference in the cosine term. Also, the model uses $2 \theta$ instead of $\theta$ multiplied by 2 inside the cosine argument, which is consistent with the ground truth, so this is acceptable.

4. **Constant term:**
   - Both have $B_0$.

**Summary:**

- The model expression differs in the linear term coefficient.
- The model expression omits the $b^2$ sine term with the time delay $\tau$.
- The model expression has a different coefficient and missing constant term in the cosine part.
- The model expression has only one sine term instead of two.
- The model expression is missing the $a^2$ factor in the sine term.
- The model expression is missing the constant term $a A_0$ from the cosine bracket.

These are multiple significant differences, not just minor errors.

**Conclusion:**

The model expression is not mathematically equivalent to the ground truth and contains more than one error (wrong coefficients, missing terms, missing time delay, missing constant term). Therefore, it is incorrect.

**Score: 0**

---

**Score: 0**

\newpage
\subsection*{Problem 49 (Paper: 2506.23953v1)}
\subsubsection*{Problem Statement}
Background:
Consider a $\mathbb{Z}_2\times\mathbb{Z}_2$-graded Lie superalgebra $\mathfrak{g} = \bigoplus_{\boldsymbol{a} \in \mathbb{Z}_2\times\mathbb{Z}_2} \mathfrak{g}_{\boldsymbol{a}}$, where the bracket $\left[\cdot,\cdot\right]$ for homogeneous elements $x_{\boldsymbol{a}} \in \mathfrak{g}_{\boldsymbol{a}}$ and $y_{\boldsymbol{b}} \in \mathfrak{g}_{\boldsymbol{b}}$ satisfies $\left[x_{\boldsymbol{a}}, y_{\boldsymbol{b}} \right] \in \mathfrak{g}_{\boldsymbol{a}+\boldsymbol{b}}$ and $\left[ x_{\boldsymbol{a}}, y_{\boldsymbol{b}} \right] = -(-1)^{\boldsymbol{a}\cdot\boldsymbol{b}} \left[ y_{\boldsymbol{b}}, x_{\boldsymbol{a}} \right]$, with $\boldsymbol{a}=(a_1,a_2)$, $\boldsymbol{b}=(b_1,b_2)$, and $\boldsymbol{a}\cdot\boldsymbol{b} = a_1b_1+a_2b_2$ (mod $2$). For a degree vector $\boldsymbol{a}=(a_1,a_2)$, define $\boldsymbol{a} \cdot \boldsymbol{a} := a_1^2 + a_2^2$ (mod $2$), so that $(-1)^{\boldsymbol{a}\cdot\boldsymbol{a}}$ is well-defined. Let this algebra be realized by the space of $(m+n+1) \times (m+n+1)$ matrices, where $m=m_1+m_2$ and $n=n_1+n_2$. Let $e_{ij}$ be the standard matrix units for $i,j \in \{0, 1, \dots, m+n\}$. Each index $k \in \{0, 1, \dots, m+n\}$ is assigned a degree $d_k \in \mathbb{Z}_2\times\mathbb{Z}_2$ as follows:
\begin{align*}
d_k &= 
\begin{cases}
\left(0,0\right); & k=0,\ldots,m_1\\
\left(1,1\right); & k=m_1+1,\ldots,m_1+m_2=m\\
\left(1,0\right); & k=m_1+m_2+1,\ldots,m_1+m_2+n_1=m+n_1\\
\left(0,1\right); & k=m_1+m_2+n_1+1,\ldots,m_1+m_2+n_1+n_2=m+n
\end{cases}
\end{align*}
The matrix unit $e_{ij}$ is a homogeneous element with degree $\deg(e_{ij})=d_i+d_j$. The bracket for these matrix units is given by $\left[ e_{ij}, e_{kl}\right] = \delta_{jk}e_{il}-(-1)^{(d_i+d_j)\cdot(d_k+d_l)}\delta_{il}e_{kj}$. A set of creation and annihilation operators, $a_k^+$ and $a_k^-$ for $k=1,\ldots,m+n$, are defined as $a_k^+=e_{k0}$ and $a_k^-=e_{0k}$. These operators are homogeneous with degree $\deg(a_k^{\pm})=d_k$.

Task:
Using the definitions provided, derive the expression for the triple bracket $\left[ \left[ a_i^{+}, a_j^{-} \right], a_k^+\right]$ for $i,j,k \in \{1, \dots, m+n\}$. Express your final result solely in terms of the creation operators $a_l^+$ and Kronecker deltas.

Solution:

\subsubsection*{Ground Truth Solution}
\[ \boxed{\delta_{jk}a_i^++(-1)^{d_i\cdot d_i}\delta_{ij}a_k^+} \]

\subsubsection*{Model Outputs}
\subsubsection*{Model: o4-mini (Score: 0.5)}
\paragraph*{Model Solution:}
\[ \bigl[[\,a_i^+,\,a_j^-\,],\,a_k^+ \bigr]
\;=\;
\delta_{jk}\,a_i^+ \;+\;(-1)^{d_i\cdot d_j}\,\delta_{ij}\,a_k^+\,. \]

\paragraph*{Judge's Evaluation:}

The ground truth expression is:
\[
\delta_{jk} a_i^+ + (-1)^{d_i \cdot d_i} \delta_{ij} a_k^+
\]

The model-generated expression is:
\[
\bigl[[a_i^+, a_j^-], a_k^+ \bigr] = \delta_{jk} a_i^+ + (-1)^{d_i \cdot d_j} \delta_{ij} a_k^+
\]

**Comparison:**

- The first term, \(\delta_{jk} a_i^+\), matches exactly.
- The second term differs in the exponent of \((-1)\):
  - Ground truth: \((-1)^{d_i \cdot d_i}\)
  - Model: \((-1)^{d_i \cdot d_j}\)

Since \(d_i \cdot d_i\) and \(d_i \cdot d_j\) are generally not equal unless \(i = j\), this is a significant difference in the sign factor.

- The Kronecker deltas \(\delta_{ij}\) and \(\delta_{jk}\) are correctly placed.
- The model expression includes the commutator notation \(\bigl[[a_i^+, a_j^-], a_k^+ \bigr]\), which is presumably the context for the expression, but the question is about the right-hand side expression.

**Conclusion:**

The model's expression differs from the ground truth in the exponent of the sign factor, which is a key part of the expression. This is more than a minor typo or missing constant; it changes the sign factor's dependence.

Therefore, the model's expression is not completely correct, but it is very close, differing only in one exponent term.

According to the rubric:

- One wrong sign or one minor coefficient error can get 0.5.
- More than one minor error or a major error gets 0.

Here, the difference is one exponent in the sign factor, which is a single minor error.

**Score: 0.5**

---

**Score: 0.5**

\newpage
\subsection*{Problem 50 (Paper: 2506.23953v1)}
\subsubsection*{Problem Statement}
Background:
Consider a $\mathbb{Z}_2 \times \mathbb{Z}_2$-graded Lie superalgebra $\mathfrak{g} = \bigoplus_{\boldsymbol{a} \in \mathbb{Z}_2 \times \mathbb{Z}_2} \mathfrak{g}_{\boldsymbol{a}}$, where the degree of a homogeneous element $x_{\boldsymbol{a}} \in \mathfrak{g}_{\boldsymbol{a}}$ is $\deg(x_{\boldsymbol{a}}) = \boldsymbol{a} = (a_1, a_2)$. The algebra is endowed with a bracket that satisfies $[x_{\boldsymbol{a}}, y_{\boldsymbol{b}}] = x_{\boldsymbol{a}} y_{\boldsymbol{b}} - (-1)^{\boldsymbol{a}\cdot\boldsymbol{b}} y_{\boldsymbol{b}} x_{\boldsymbol{a}}$, where $\boldsymbol{a} \cdot \boldsymbol{b} = a_1 b_1 + a_2 b_2$. The algebra is generated by a set of creation operators $a_i^+$ and annihilation operators $a_i^-$, which are classified into four families based on their grading:
\begin{itemize}
    \item $b_i^\pm$ for $i=1, \dots, m_1$, with degree $(0,0)$.
    \item $\tilde{b}_i^\pm$ for $i=1, \dots, m_2$, with degree $(1,1)$.
    \item $f_i^\pm$ for $i=1, \dots, n_1$, with degree $(1,0)$.
    \item $\tilde{f}_i^\pm$ for $i=1, \dots, n_2$, with degree $(0,1)$.
\end{itemize}
For any two creation operators $a_i^+$ and $a_j^+$, the bracket relation is $[a_i^+, a_j^+] = 0$.

A class of irreducible representations, known as Fock representations, is constructed on a space $W_p$ labeled by a positive integer $p$. This space is spanned by a set of orthonormal basis vectors $|p;r_1,\ldots,r_{m_1},l_1,\ldots,l_{m_2},\theta_1,\ldots,\theta_{n_1},\lambda_1,\ldots,\lambda_{n_2}\rangle$. These vectors are constructed by the action of creation operators on a normalized vacuum state $|0\rangle$ as follows:
\begin{align*}
& |p;r_1,\ldots,r_{m_1},l_1,\ldots,l_{m_2},\theta_1,\ldots,\theta_{n_1},\lambda_1,\ldots,\lambda_{n_2}\rangle = \\
& \sqrt{\frac{(p-R)!}{p!\; r_1! \ldots r_{m_1}!\; l_1! \ldots l_{m_2}!\; \theta_1! \ldots \theta_{n_1}!\; \lambda_1! \ldots \lambda_{n_2}!}} \\
& \quad \times (b_1^+)^{r_1}\ldots (b_{m_1}^+)^{r_{m_1}}(\tilde{b}_1^+)^{l_1}\ldots (\tilde{b}_{m_2}^+)^{l_{m_2}} (f_1^+)^{\theta_1}\ldots (f_{n_1}^+)^{\theta_{n_1}}(\tilde{f}_1^+)^{\lambda_1}\ldots (\tilde{f}_{n_2}^+)^{\lambda_{n_2}} |0\rangle,
\end{align*}
where the occupation numbers are integers $r_i, l_i \ge 0$ and $\theta_i, \lambda_i \in \{0,1\}$. The total particle number is $R = \sum_{i=1}^{m_1} r_i + \sum_{i=1}^{m_2} l_i + \sum_{i=1}^{n_1} \theta_i + \sum_{i=1}^{n_2} \lambda_i$, and is subject to the constraint $R \le p$.

Task:
Derive the action of the creation operator $f_i^+$ (for a given $i \in \{1, \ldots, n_1\}$) on a general orthonormal basis state $|p;r_1,\ldots,l_1,\ldots,\theta_1,\ldots,\theta_i,\ldots,\theta_{n_1},\ldots\rangle$. Express your result as a coefficient multiplying another orthonormal basis state of $W_p$.

\subsubsection*{Ground Truth Solution}
\[ \boxed{f_i^+|p;\ldots,r_{m_1},l_1,\ldots,l_{m_2},\theta_1,\ldots,\theta_{i-1},\theta_i,\theta_{i+1},\ldots,\theta_{n_1},\ldots )=(1-\theta_i)(-1)^{\sum_{k=1}^{m_2}l_k} (-1)^{\sum_{k=1}^{i-1}\theta_k} \sqrt{p-R} |p;\ldots,r_{m_1},l_1,\ldots,l_{m_2},\theta_1,\ldots,\theta_{i-1},\theta_i+1,\theta_{i+1},\ldots,\theta_{n_1},\ldots )} \]

\subsubsection*{Model Outputs}
\subsubsection*{Model: o4-mini (Score: 0.0)}
\paragraph*{Model Solution:}
\[ f_i^+\,|p;\dots,\theta_i,\dots\rangle
\;=\;
\begin{cases}
(-1)^{\sum_{k=1}^{i-1}\theta_k}\,\sqrt{\,p - R\,}\;
\bigl|p;\dots,\theta_i+1,\dots\bigr\rangle,
&\theta_i=0,\;R<p,\\[6pt]
0,&\text{otherwise.}
\end{cases} \]

\paragraph*{Judge's Evaluation:}

The ground truth expression is:
\[
f_i^+|p;\ldots,r_{m_1},l_1,\ldots,l_{m_2},\theta_1,\ldots,\theta_{i-1},\theta_i,\theta_{i+1},\ldots,\theta_{n_1},\ldots ) = (1-\theta_i)(-1)^{\sum_{k=1}^{m_2} l_k} (-1)^{\sum_{k=1}^{i-1} \theta_k} \sqrt{p-R} \; |p;\ldots,r_{m_1},l_1,\ldots,l_{m_2},\theta_1,\ldots,\theta_{i-1},\theta_i+1,\theta_{i+1},\ldots,\theta_{n_1},\ldots )
\]

The model-generated expression is:
\[
f_i^+ |p;\dots,\theta_i,\dots\rangle = 
\begin{cases}
(-1)^{\sum_{k=1}^{i-1} \theta_k} \sqrt{p - R} \; |p;\dots,\theta_i+1,\dots\rangle, & \theta_i=0, \; R < p, \\[6pt]
0, & \text{otherwise}.
\end{cases}
\]

---

**Comparison:**

1. **Factor \((1-\theta_i)\):**  
   The ground truth explicitly includes the factor \((1-\theta_i)\), which ensures the operator acts only if \(\theta_i=0\). The model uses a case distinction with \(\theta_i=0\) and zero otherwise, which is equivalent.

2. **Sign factor \((-1)^{\sum_{k=1}^{m_2} l_k}\):**  
   The ground truth includes an additional sign factor \((-1)^{\sum_{k=1}^{m_2} l_k}\) involving the \(l_k\) variables. The model expression **omits** this factor entirely.

3. **Sign factor \((-1)^{\sum_{k=1}^{i-1} \theta_k}\):**  
   Both expressions include this factor identically.

4. **Square root factor \(\sqrt{p-R}\):**  
   Both expressions include this factor identically.

5. **State ket:**  
   Both expressions increment \(\theta_i\) by 1 in the ket, consistent.

6. **Domain of variables \(r_{m_1}, l_{m_2}\):**  
   The ground truth explicitly includes these variables in the ket and the sign factor, while the model expression omits them entirely.

---

**Conclusion:**

The model expression is missing the crucial sign factor \((-1)^{\sum_{k=1}^{m_2} l_k}\) and the dependence on the \(l_k\) variables. This is a significant omission, not a minor typo or missing constant. The rest of the expression is correct and the case distinction is equivalent to the factor \((1-\theta_i)\).

Because there is more than one minor error (missing an entire sign factor and variables), the model expression is **not** fully correct, nor is it just a minor error.

---

**Score: 0**

\newpage
\subsection*{Problem 51 (Paper: 2506.23987v1)}
\subsubsection*{Problem Statement}
Background:
Consider a system of $n$ spins $\sigma = (\sigma_1, \dots, \sigma_n)$ constrained to the sphere $S_n = \{\sigma \in \mathbb{R}^n \mid \sum_{i=1}^n \sigma_i^2 = n\}$. The expectation over the uniform measure on this sphere is denoted by $\mathbb{E}[\cdot]$. The system is described by a Non-Intersecting Monomial (NIM) Hamiltonian, which is a sum of $K$ monomial terms:
$H_n(\sigma) = \sum_{k=1}^K \widehat{H}_k(\sigma)$, where $\widehat{H}_k(\sigma) = C_k \left(\prod_{j \in I_k} \sigma_j\right) n^{-(p_k-2)/2}$.
The index sets $I_k \subset \{1, \dots, n\}$ are pairwise disjoint ($I_k \cap I_j = \emptyset$ for $k \neq j$), and their sizes are denoted by $|I_k| = p_k \ge 2$. The terms $C_k$ are real-valued coupling constants. The partition function is $Z_n = \mathbb{E}[\exp(H_n(\sigma))]$ and the free energy per spin is $F_n = \frac{1}{n}\log Z_n$.

For a system with a single monomial term, $\widehat{H}(\sigma) = C \left(\prod_{j \in I} \sigma_j\right) n^{-(p-2)/2}$ with $|I|=p$, the following properties are known:
1.  The partition function $Z_{n,1} = \mathbb{E}[\exp(\widehat{H})]$ can be expanded as a Taylor series. The term proportional to $C^z$ is related to the expectation $\mathbb{E}[|\widehat{H}|^z]$. The function $g(z) = \log \frac{\mathbb{E}[|\widehat{H}|^z]}{z!}$ is convex for $z \in \mathbb{R}_{\ge 0}$.
2.  There exists a threshold $C_p^*$ for the coupling constant. If $|C| > C_p^*$, the limiting free energy is non-trivial and given by $F = \lim_{n\to\infty} \frac{1}{n}\log Z_{n,1} = f_p(C)$. If $|C| \le C_p^*$, the limiting free energy is zero.
3.  The function $f_p(C)$ is defined as $f_p(C) = 2\lambda_p(C) - \frac{1}{2}\log(1+2p\lambda_p(C))$. The quantity $\lambda_p(C)$ is the largest real solution $\lambda$ to the equation:
    $2\log|C| + (p-2)\log(2\lambda) - p\log(2p\lambda+1) = 0$.
4.  This phase transition arises because for $|C| > C_p^*$, the Taylor series of $Z_{n,1}$ is dominated by terms of order $C^{2\ell}$ where $\ell \approx n\lambda_p(C)$, while for $|C| \le C_p^*$, it is dominated by the lowest-order terms ($\ell=0, 1$).

Assume for the NIM model that all $K$ couplings are above their respective thresholds, i.e., $|C_k| > C_{p_k}^*$ for all $k=1, \dots, K$.

Task:
Using the provided properties of single-monomial systems, derive the limiting free energy $F = \lim_{n\to\infty} F_n$ for the general Non-Intersecting Monomial (NIM) model. Express your result in terms of the functions $f_{p_k}$ and the coupling constants $C_k$.

\subsubsection*{Ground Truth Solution}
\[ \boxed{F = \max_{k \in \{1, \dots, K\}} f_{p_k}(C_k)} \]

\subsubsection*{Model Outputs}
\subsubsection*{Model: o4-mini (Score: 0.0)}
\paragraph*{Model Solution:}
\[ F \;=\;\sum_{k=1}^K f_{p_k}(C_k)\;. \]

\paragraph*{Judge's Evaluation:}

The ground truth expression is 
\[
F = \max_{k \in \{1, \dots, K\}} f_{p_k}(C_k),
\]
which means \(F\) is the maximum value of the set \(\{f_{p_1}(C_1), f_{p_2}(C_2), \dots, f_{p_K}(C_K)\}\).

The model-generated expression is 
\[
F = \sum_{k=1}^K f_{p_k}(C_k),
\]
which means \(F\) is the sum of all the values \(f_{p_k}(C_k)\) for \(k=1\) to \(K\).

These two expressions represent fundamentally different operations: one is a maximum, the other is a sum. This is more than a minor error; it changes the meaning and the value of \(F\) significantly.

Therefore, the model-generated expression is incorrect.

Score: 0

\newpage
\subsection*{Problem 52 (Paper: 2506.23987v1)}
\subsubsection*{Problem Statement}
Background:
Consider a system of $n$ spins $\sigma = (\sigma_1, \dots, \sigma_n)$ constrained to the sphere $\sum_{i=1}^n \sigma_i^2 = n$. The expectation $\mathbb{E}[\cdot]$ is taken with respect to the uniform probability measure on this sphere. The system is described by a $p$-spin Hamiltonian, with $p \ge 3$, given by
\[
H_n(\sigma) = \sum_{i=1}^{N} c_i \sigma_{I_i} n^{-(p-2)/2}
\]
where $I_i \subset \{1, \dots, n\}$ are distinct $p$-element index sets, $\sigma_{I_i} = \prod_{j \in I_i} \sigma_j$ is a spin monomial, and $c_i$ are positive real coefficients. The interactions between these monomials can be described by a monomial graph $G=(V,E)$, where the vertices $V = \{\sigma_{I_i}\}_{i=1}^N$ correspond to the monomials, and an edge $(i,j) \in E$ exists if the index sets of the corresponding monomials have a non-empty intersection, i.e., $I_i \cap I_j \neq \emptyset$.

You are given the following facts:
1.  The monomial graph $G$ is $C$-colorable for some integer $C \ge 1$. This implies that the set of all monomials $V$ can be partitioned into $C$ disjoint classes (color classes), $V = \mathcal{C}_1 \cup \dots \cup \mathcal{C}_C$, such that within each class $\mathcal{C}_j$, all monomials are mutually non-intersecting (i.e., for any two distinct $\sigma_{I_a}, \sigma_{I_b} \in \mathcal{C}_j$, their index sets are disjoint, $I_a \cap I_b = \emptyset$).
2.  For a Hamiltonian composed of a sum of mutually non-intersecting monomials, $H_{\text{NIM}} = \sum_{k} d_k \sigma_{J_k} n^{-(p-2)/2}$, the partition function is bounded by the product of individual partition functions: $\mathbb{E}[\exp(H_{\text{NIM}})] \le \prod_k \mathbb{E}[\exp(d_k \sigma_{J_k} n^{-(p-2)/2})]$.
3.  For a single monomial with a sufficiently small coefficient $d$, the logarithm of its partition function for $p \ge 3$ can be approximated by the leading term of its series expansion: $\log \mathbb{E}[\exp(d \sigma_J n^{-(p-2)/2})] = \frac{1}{2} n^{2-p} d^2 + O(n^{4-2p}d^4)$.

Task:
For a given positive parameter $s$, use the provided facts and H\"older's inequality to derive the leading-order upper bound for the log-partition function, $\log \mathbb{E}[\exp(s H_n(\sigma))]$. Express your answer in terms of $s, C, p, n,$ and the sum of the squares of the coefficients, $\sum_{i=1}^N c_i^2$.

\subsubsection*{Ground Truth Solution}
\[ \boxed{\frac{1}{2} s^2 C n^{2-p} \sum_{i=1}^N c_i^2} \]

\subsubsection*{Model Outputs}
\subsubsection*{Model: o4-mini (Score: 0.5)}
\paragraph*{Model Solution:}
\[ \log \E\big[e^{sH_n(\sigma)}\big]
\;\le\;
\frac{C\,s^2\,n^{2-p}}2\sum_{i=1}^N c_i^2
\,+\,o(1)\,. \]

\paragraph*{Judge's Evaluation:}

The ground truth expression is:
\[
\frac{1}{2} s^2 C n^{2-p} \sum_{i=1}^N c_i^2
\]

The model-generated expression is:
\[
\log \mathbb{E}\big[e^{sH_n(\sigma)}\big] \le \frac{C s^2 n^{2-p}}{2} \sum_{i=1}^N c_i^2 + o(1)
\]

**Comparison:**

- The main term on the right-hand side of the model expression matches the ground truth expression exactly in form: the factor $\frac{1}{2} s^2 C n^{2-p} \sum_{i=1}^N c_i^2$ is present.
- The model expression includes an inequality ($\le$) and an additional $o(1)$ term.
- The ground truth expression is a standalone expression, presumably the exact value or limit.
- The model expression is an upper bound with a small error term $o(1)$.
- Since the rubric states the final expression must be mathematically equivalent, and the model expression is an inequality with an extra $o(1)$ term, it is not exactly the same as the ground truth.
- However, the main term matches perfectly, and the extra terms are minor (an inequality and a small error term).
- According to the rubric, partial credit (0.5) is given for one minor error such as a missing minor constant or one wrong sign.
- Here, the difference is the inequality and the $o(1)$ term, which is a minor difference but changes the nature of the expression (from equality to inequality with a small error).
- This is close but not exactly equivalent.

**Conclusion:**

The model expression is very close but not exactly the same as the ground truth. The presence of the inequality and the $o(1)$ term means it is not fully correct but very close.

Score: 0.5

---

**Score: 0.5**

\newpage
\subsection*{Problem 53 (Paper: 2506.23991v1)}
\subsubsection*{Problem Statement}
Background:
Consider a non-dissipative two-fluid plasma model described by the variables $ (\boldsymbol{E}, \boldsymbol{B}, n_i, n_e, \boldsymbol{u}_e, \boldsymbol{u}_i) $, where $ \boldsymbol{E} $ and $ \boldsymbol{B} $ are the electric and magnetic fields, and $ n_\sigma, \boldsymbol{u}_\sigma $ are the number density and fluid velocity for species $ \sigma \in \{i, e\} $ (ions, electrons). The system possesses a Hamiltonian structure defined by a Poisson bracket. This structure can be described by an anchor map $ \pi^\# $, which maps the functional derivatives of any functional $ F $, denoted $ F_{\chi_i} = \fdv{F}{\chi_i} $, to the corresponding Hamiltonian vector field $ X_F = (\dot{\boldsymbol{B}}, \dot{\boldsymbol{E}}, \dot{\boldsymbol{u}_i}, \dot{\boldsymbol{u}_e}, \dot{n_i}, \dot{n_e}) $. The components of this vector field are given by:
\begin{align*}
\dot{\boldsymbol{B}} &= -\frac{1}{\epsilon \epsilon_o}\nabla\times{F_{\boldsymbol{E}}} \\
\dot{\boldsymbol{E}} &= \frac{1}{\epsilon \epsilon_o}\nabla\times{F_{\boldsymbol{B}}} - \frac{q_e}{\epsilon^2 \epsilon_o m_e}F_{\boldsymbol{u}_e} +  \frac{Z_iq_e}{\epsilon^2 \epsilon_o m_i}F_{\boldsymbol{u}_i} \\
\dot{\boldsymbol{u}_i} &= -\frac{Z_iq_e}{\epsilon m_i^2n_i}(F_{\boldsymbol{u}_i}\times \boldsymbol{B}) -\frac{Z_iq_e}{\epsilon^2 \epsilon_o m_i}F_{\boldsymbol{E}}-\frac{1}{m_i}\nabla F_{n_i} + \frac{1}{m_i n_i} F_{\boldsymbol{u}_{i}} \times (\nabla \times \boldsymbol{u}_{i}) \\ 
\dot{\boldsymbol{u}_e} &= \frac{q_e}{\epsilon m_e^2n_e}(F_{\boldsymbol{u}_e}\times \boldsymbol{B}) +\frac{q_e}{\epsilon^2 \epsilon_o m_e}F_{\boldsymbol{E}}-\frac{1}{m_e}\nabla F_{n_e} + \frac{1}{m_e n_e} F_{\boldsymbol{u}_{e}} \times (\nabla \times \boldsymbol{u}_{e}) \\
\dot{n_i} &= - m_i^{-1}\nabla\cdot{F_{\boldsymbol{u}_i}} \\
\dot{n_e} &= - m_e^{-1}\nabla\cdot{F_{\boldsymbol{u}_e}}
\end{align*}
Here, $ m_\sigma $ is the species mass, $ q_e $ is the electron charge, $ Z_i $ is the ion atomic number, and $ \epsilon, \epsilon_o $ are physical constants. We wish to reduce this system by imposing Gauss's Law as a constraint. This defines a submanifold $ N $ where the electron number density $ n_e $ is no longer an independent variable, but is determined by the other fields:
\begin{equation*}
n_e = Z_i n_i + \epsilon^2 \epsilon_o q_e^{-1}\nabla \cdot \boldsymbol{E}
\end{equation*}
The state on this submanifold $ N $ is described by the variables $ (\boldsymbol{E}, \boldsymbol{B}, n_i, \boldsymbol{u}_e, \boldsymbol{u}_i) $. It can be shown that $ N $ is a Poisson-Dirac submanifold. For this specific case, the submanifold is a Poisson submanifold, which implies that the `$\pi$-orthogonal complement` of its tangent space, $(T_pN)^{\pi\perp}$, is trivial. This property leads to a significant simplification in the procedure for finding the reduced Poisson bracket on $ N $: the reduced anchor map $ \pi_N^\# $ can be obtained by taking the full anchor map $ \pi^\# $ and applying it to functionals $ F $ that have no explicit dependence on the eliminated variable $ n_e $. This is equivalent to setting the functional derivative $ F_{n_e} $ to zero in the expressions for the components of the vector field $ X_F $.

Task:
Using the simplification described above, determine the components of the reduced Hamiltonian vector field $ X_F $ on the submanifold $ N $. The vector field on $ N $ has components $ (\dot{\boldsymbol{B}}, \dot{\boldsymbol{E}}, \dot{\boldsymbol{u}_i}, \dot{\boldsymbol{u}_e}, \dot{n_i}) $. In your final expression, any remaining instances of $ n_e $ should be understood as being determined by the constraint equation.

\subsubsection*{Ground Truth Solution}
\[ \boxed{X_F = \begin{pmatrix}
            -\frac{1}{\epsilon \epsilon_o}\nabla\times{F_{\boldsymbol{E}}} \\
            \frac{1}{\epsilon \epsilon_o}\nabla\times{F_{\boldsymbol{B}}} - \frac{q_e}{\epsilon^2 \epsilon_o m_e}F_{\boldsymbol{u}_e} +  \frac{Z_iq_e}{\epsilon^2 \epsilon_o m_i}F_{\boldsymbol{u}_i} \\
            -\frac{Z_iq_e}{\epsilon m_i^2n_i}(F_{\boldsymbol{u}_i}\times \boldsymbol{B}) -\frac{Z_iq_e}{\epsilon^2 \epsilon_o m_i}F_{\boldsymbol{E}}-m_i^{-1}\nabla F_{n_i} + \frac{1}{m_i n_i} F_{\boldsymbol{u}_{i}} \times (\nabla \times \boldsymbol{u}_{i}) \\ 
            \frac{q_e}{\epsilon m_e^2n_e}(F_{\boldsymbol{u}_e}\times \boldsymbol{B}) +\frac{q_e}{\epsilon^2 \epsilon_o m_e}F_{\boldsymbol{E}} + \frac{1}{m_e n_e} F_{\boldsymbol{u}_{e}} \times (\nabla \times \boldsymbol{u}_{e}) \\
            - m_i^{-1}\nabla\cdot{F_{\boldsymbol{u}_i}} \\
        \end{pmatrix}} \]

\subsubsection*{Model Outputs}
\subsubsection*{Model: o4-mini (Score: 1.0)}
\paragraph*{Model Solution:}
\[ \begin{aligned}
\dot{\boldsymbol B}&=-\frac{1}{\epsilon\epsilon_0}\,\nabla\times F_{\boldsymbol E}\,,\\
\dot{\boldsymbol E}&=\frac{1}{\epsilon\epsilon_0}\,\nabla\times F_{\boldsymbol B}
\;-\;\frac{q_e}{\epsilon^2\epsilon_0\,m_e}\,F_{\boldsymbol u_e}
\;+\;\frac{Z_iq_e}{\epsilon^2\epsilon_0\,m_i}\,F_{\boldsymbol u_i}\,,\\
\dot{\boldsymbol u_i}&=
-\frac{Z_iq_e}{\epsilon\,m_i^2\,n_i}\,\bigl(F_{\boldsymbol u_i}\times\boldsymbol B\bigr)
\;-\;\frac{Z_iq_e}{\epsilon^2\epsilon_0\,m_i}\,F_{\boldsymbol E}
\;-\;\frac{1}{m_i}\,\nabla F_{n_i}
\;+\;\frac{1}{m_i\,n_i}\,F_{\boldsymbol u_i}\times(\nabla\times\boldsymbol u_i)\,,\\
\dot{\boldsymbol u_e}&=
\frac{q_e}{\epsilon\,m_e^2\,n_e}\,\bigl(F_{\boldsymbol u_e}\times\boldsymbol B\bigr)
\;+\;\frac{q_e}{\epsilon^2\epsilon_0\,m_e}\,F_{\boldsymbol E}
\;+\;\frac{1}{m_e\,n_e}\,F_{\boldsymbol u_e}\times(\nabla\times\boldsymbol u_e)\,,\\
\dot n_i&=-\frac{1}{m_i}\,\nabla\cdot F_{\boldsymbol u_i}\,,
\end{aligned} \]

\paragraph*{Judge's Evaluation:}

The Ground Truth expression is given as a vector $X_F$ with five components, each corresponding to a physical quantity's time derivative or evolution equation. The Model-Generated expression writes these components explicitly as time derivatives of the corresponding variables, with the same terms and coefficients.

Let's compare each component carefully:

1. **First component:**

Ground Truth:
\[
-\frac{1}{\epsilon \epsilon_0} \nabla \times F_{\boldsymbol{E}}
\]

Model:
\[
\dot{\boldsymbol{B}} = -\frac{1}{\epsilon \epsilon_0} \nabla \times F_{\boldsymbol{E}}
\]

This matches exactly.

2. **Second component:**

Ground Truth:
\[
\frac{1}{\epsilon \epsilon_0} \nabla \times F_{\boldsymbol{B}} - \frac{q_e}{\epsilon^2 \epsilon_0 m_e} F_{\boldsymbol{u}_e} + \frac{Z_i q_e}{\epsilon^2 \epsilon_0 m_i} F_{\boldsymbol{u}_i}
\]

Model:
\[
\dot{\boldsymbol{E}} = \frac{1}{\epsilon \epsilon_0} \nabla \times F_{\boldsymbol{B}} - \frac{q_e}{\epsilon^2 \epsilon_0 m_e} F_{\boldsymbol{u}_e} + \frac{Z_i q_e}{\epsilon^2 \epsilon_0 m_i} F_{\boldsymbol{u}_i}
\]

Exact match.

3. **Third component:**

Ground Truth:
\[
-\frac{Z_i q_e}{\epsilon m_i^2 n_i} (F_{\boldsymbol{u}_i} \times \boldsymbol{B}) - \frac{Z_i q_e}{\epsilon^2 \epsilon_0 m_i} F_{\boldsymbol{E}} - m_i^{-1} \nabla F_{n_i} + \frac{1}{m_i n_i} F_{\boldsymbol{u}_i} \times (\nabla \times \boldsymbol{u}_i)
\]

Model:
\[
\dot{\boldsymbol{u}}_i = -\frac{Z_i q_e}{\epsilon m_i^2 n_i} (F_{\boldsymbol{u}_i} \times \boldsymbol{B}) - \frac{Z_i q_e}{\epsilon^2 \epsilon_0 m_i} F_{\boldsymbol{E}} - \frac{1}{m_i} \nabla F_{n_i} + \frac{1}{m_i n_i} F_{\boldsymbol{u}_i} \times (\nabla \times \boldsymbol{u}_i)
\]

Exact match.

4. **Fourth component:**

Ground Truth:
\[
\frac{q_e}{\epsilon m_e^2 n_e} (F_{\boldsymbol{u}_e} \times \boldsymbol{B}) + \frac{q_e}{\epsilon^2 \epsilon_0 m_e} F_{\boldsymbol{E}} + \frac{1}{m_e n_e} F_{\boldsymbol{u}_e} \times (\nabla \times \boldsymbol{u}_e)
\]

Model:
\[
\dot{\boldsymbol{u}}_e = \frac{q_e}{\epsilon m_e^2 n_e} (F_{\boldsymbol{u}_e} \times \boldsymbol{B}) + \frac{q_e}{\epsilon^2 \epsilon_0 m_e} F_{\boldsymbol{E}} + \frac{1}{m_e n_e} F_{\boldsymbol{u}_e} \times (\nabla \times \boldsymbol{u}_e)
\]

Exact match.

5. **Fifth component:**

Ground Truth:
\[
- m_i^{-1} \nabla \cdot F_{\boldsymbol{u}_i}
\]

Model:
\[
\dot{n}_i = -\frac{1}{m_i} \nabla \cdot F_{\boldsymbol{u}_i}
\]

Exact match.

---

**Additional notes:**

- The model uses $\epsilon_0$ instead of $\epsilon_o$; this is a trivial difference in notation.
- The model uses explicit time derivatives $\dot{\boldsymbol{B}}, \dot{\boldsymbol{E}}, \dot{\boldsymbol{u}}_i, \dot{\boldsymbol{u}}_e, \dot{n}_i$ instead of a vector $X_F$; this is acceptable as it is just a different presentation.
- All coefficients, signs, and terms match exactly.
- No missing terms or factors.

---

**Conclusion:**

The model-generated expression is mathematically equivalent to the ground truth expression, with only superficial differences in notation and formatting.

**Score: 1**

---

**Final answer:**

The model-generated expression matches the ground truth exactly in all components, with no errors or omissions. The differences are only in notation and presentation, which do not affect correctness.

Score: 1

\newpage
\subsection*{Problem 54 (Paper: 2506.23991v1)}
\subsubsection*{Problem Statement}
Background:
Consider the ideal two-fluid plasma model, whose state is described by the fields $ (\boldsymbol{E}, \boldsymbol{B}, n_i, n_e, \boldsymbol{u}_e, \boldsymbol{u}_i) $ representing the electric field, magnetic field, ion and electron number densities, and ion and electron fluid velocities, respectively. The dynamics are Hamiltonian, governed by a Poisson bracket $ \{F,G\} $ for any two functionals $F$ and $G$ of the state variables. The associated Poisson anchor map $ \pi^\# $, which maps a covector (functional derivative) $ \delta F $ to a vector (time derivative) $X_F = \dot{z}$, can be expanded in a small parameter $\epsilon$ as $ \epsilon^2 \pi^\#_\epsilon = \pi_0 + \epsilon \pi_1 + \epsilon^2 \pi_2 + \mathcal{O}(\epsilon^3) $. The electron number density $n_e$ is determined by the other fields via Gauss's Law, $ \epsilon_0 \nabla \cdot \boldsymbol{E} = q_e n_e + q_i n_i $, where $ q_i = -Z_i q_e $ is the ion charge. The phase space can be described by the independent variables $ z = (n_i, \boldsymbol{u}_i, \boldsymbol{B}, \boldsymbol{u}_e, \boldsymbol{E}) $. The components of the vector $ X_F = (\delta \boldsymbol{B}, \delta \boldsymbol{E}, \delta \boldsymbol{u}_i, \delta \boldsymbol{u}_e, \delta n_i)^T $ for each order in $\epsilon$ are given by:
\begin{align*}
\pi_0(\delta F) &= \begin{pmatrix}
0 \\
- \frac{q_e}{\epsilon_o m_e}F_{\boldsymbol{u}_e} +  \frac{Z_i q_e}{\epsilon_o m_i}F_{\boldsymbol{u}_i} \\
-\frac{Z_i q_e}{ \epsilon_o m_i}F_{\boldsymbol{E}}  \\ 
\frac{q_e}{ \epsilon_o m_e}F_{\boldsymbol{E}}  \\
0
\end{pmatrix}, \\
\pi_1(\delta F) &= \begin{pmatrix}
-\frac{1}{ \epsilon_o}\nabla\times{F_{\boldsymbol{E}}} \\
\frac{1}{ \epsilon_o}\nabla\times{F_{\boldsymbol{B}}}  \\
\frac{-Z_i q_e}{m_i^2 n_i}(F_{\boldsymbol{u}_i}\times \boldsymbol{B}) \\ 
\frac{q_e}{m_e^2 n_e}(F_{\boldsymbol{u}_e}\times \boldsymbol{B}) \\
0
\end{pmatrix}, \\
\pi_2(\delta F) &= \begin{pmatrix}
0 \\
0\\
-m_i^{-1}\nabla F_{n_i} + \frac{1}{m_i n_i} F_{\boldsymbol{u}_{i}} \times (\nabla \times \boldsymbol{u}_{i}) \\ 
 \frac{1}{m_e n_e} F_{\boldsymbol{u}_{e}} \times (\nabla \times \boldsymbol{u}_{e}) \\
- m_i^{-1}\nabla\cdot{F_{\boldsymbol{u}_i}}
\end{pmatrix}.
\end{align*}
We are interested in the dynamics on the "slow manifold" $N$, which is a submanifold of the full phase space defined by the constraints $\boldsymbol{u}_e = \boldsymbol{u}_i$ and $\boldsymbol{E} = -\boldsymbol{u}_i \times \boldsymbol{B}$. The variables on $N$ are $ (n_i, \boldsymbol{u}_i, \boldsymbol{B}) $. The induced Poisson bracket on $N$ is given by $ \{F,G\}_N = \{\tilde{F}, \tilde{G}\}|_N $, where $\tilde{F}$ is an extension of the functional $F$ from $N$ to the full phase space. An extension $\tilde{F}$ must satisfy $\tilde{F}|_N = F$ and its associated Hamiltonian vector field $X_{\tilde{F}} = \pi^\#(\delta \tilde{F})$ must be tangent to $N$. The one-form $\delta \tilde{F}$ for an extension can be written as
$$
\delta \tilde{F}=\int \left( F_{n_i}\delta n_i + (F_{\boldsymbol{u}_{i}} - \Theta_{\boldsymbol{E}}\times \boldsymbol{B} - \Theta_{\boldsymbol{u}_e} )\cdot \delta \boldsymbol{u}_{i} + (\Theta_{\boldsymbol{E}}\times \boldsymbol{u}_{i}+ F_{\boldsymbol{B}})\cdot \delta \boldsymbol{B} + \Theta_{\boldsymbol{u}_e}\cdot \delta \boldsymbol{u}_e + \Theta_{\boldsymbol{E}}\cdot \delta \boldsymbol{E} \right) \,d^3\boldsymbol{x},
$$
where $F_{n_i}, F_{\boldsymbol{u}_i}, F_{\boldsymbol{B}}$ are the functional derivatives of $F$ on $N$, and the functionals $\Theta_{\boldsymbol{u}_e}$ and $\Theta_{\boldsymbol{E}}$ are determined by enforcing the tangency condition on $X_{\tilde{F}}$ perturbatively in $\epsilon$. In your calculations, you should expand quantities involving $n_e$ as a power series in $\epsilon$, using $n_e = Z_i n_i - (\epsilon^2 \epsilon_0/q_e) \nabla \cdot \boldsymbol{E}$ and the on-manifold relation for $\boldsymbol{E}$. Define the mass ratio $\nu = m_e/m_i$ and the total mass $\mathsf{m} = (1+\nu Z_i)m_i$.

Task:
By solving for the extension functionals $\Theta_{\boldsymbol{u}_e}$ and $\Theta_{\boldsymbol{E}}$ perturbatively and computing the induced bracket, determine the $\mathcal{O}(1)$ Poisson bracket $\{F,G\}_N$ on the slow manifold $N$. This corresponds to the $\mathcal{O}(\epsilon^2)$ part of the full bracket $\{\tilde{F}, \tilde{G}\}$. Your final answer should be a single integral expression for $\{F,G\}_N$ in terms of the functional derivatives $F_{n_i}, F_{\boldsymbol{u}_i}, F_{\boldsymbol{B}}$ and their counterparts for $G$.

\subsubsection*{Ground Truth Solution}
\[ \boxed{\{F,G\}_N = \int \left( -\mathsf{m}^{-1}(G_{\boldsymbol{u}_i}\cdot \nabla F_{n_i} + G_{n_i}\nabla \cdot F_{\boldsymbol{u}_i}) + \frac{(\nabla\times \boldsymbol{u}_i)}{\mathsf{m}n_i} \cdot (G_{\boldsymbol{u}_i}\times F_{\boldsymbol{u}_i}) +\frac{1}{\mathsf{m}n_i}\boldsymbol{B}\cdot(G_{\boldsymbol{u}_i}\times (\nabla\times F_{\boldsymbol{B}}) - F_{\boldsymbol{u}_i}\times (\nabla\times G_{\boldsymbol{B}})) \right) \,d^3\boldsymbol{x}} \]

\subsubsection*{Model Outputs}
\subsubsection*{Model: o4-mini (Score: 0.0)}
\paragraph*{Model Solution:}
\[ %
\{F,G\}_N
=\int d^3x\Big\{
n\,(F_{u}\!\cdot\!\nabla G_{n}-G_{u}\!\cdot\!\nabla F_{n})
+\frac{1}{\rho\,n}\,(\nabla\!\times u)\!\cdot\!(F_{u}\times G_{u})
+B\!\cdot\!\bigl[F_{u}\times(\nabla\!\times G_{B})
  -G_{u}\times(\nabla\!\times F_{B})\bigr]
-\frac{1}{\mu_{0}q_{e}Z_{i}\,n}\,B\!\cdot\!\bigl[(\nabla\!\times F_{B})\times(\nabla\!\times G_{B})\bigr]
\Big\}\,. \]

\paragraph*{Judge's Evaluation:}

The ground truth expression is:
\[
\{F,G\}_N = \int \left( -\mathsf{m}^{-1}(G_{\boldsymbol{u}_i}\cdot \nabla F_{n_i} + G_{n_i}\nabla \cdot F_{\boldsymbol{u}_i}) + \frac{(\nabla\times \boldsymbol{u}_i)}{\mathsf{m}n_i} \cdot (G_{\boldsymbol{u}_i}\times F_{\boldsymbol{u}_i}) +\frac{1}{\mathsf{m}n_i}\boldsymbol{B}\cdot\bigl(G_{\boldsymbol{u}_i}\times (\nabla\times F_{\boldsymbol{B}}) - F_{\boldsymbol{u}_i}\times (\nabla\times G_{\boldsymbol{B}})\bigr) \right) \,d^3\boldsymbol{x}
\]

The model-generated expression is:
\[
\{F,G\}_N = \int d^3x \Big\{
n\,(F_{u}\cdot \nabla G_{n} - G_{u}\cdot \nabla F_{n})
+ \frac{1}{\rho n} (\nabla \times u) \cdot (F_{u} \times G_{u})
+ B \cdot \bigl[ F_{u} \times (\nabla \times G_{B}) - G_{u} \times (\nabla \times F_{B}) \bigr]
- \frac{1}{\mu_0 q_e Z_i n} B \cdot \bigl[ (\nabla \times F_{B}) \times (\nabla \times G_{B}) \bigr]
\Big\}
\]

---

**Comparison and analysis:**

1. **Structure and terms:**

- The ground truth has three terms inside the integral:
  - A term proportional to $-\mathsf{m}^{-1}(G_{\boldsymbol{u}_i}\cdot \nabla F_{n_i} + G_{n_i} \nabla \cdot F_{\boldsymbol{u}_i})$.
  - A vorticity term: $\frac{(\nabla \times \boldsymbol{u}_i)}{\mathsf{m} n_i} \cdot (G_{\boldsymbol{u}_i} \times F_{\boldsymbol{u}_i})$.
  - A magnetic term: $\frac{1}{\mathsf{m} n_i} \boldsymbol{B} \cdot \left( G_{\boldsymbol{u}_i} \times (\nabla \times F_{\boldsymbol{B}}) - F_{\boldsymbol{u}_i} \times (\nabla \times G_{\boldsymbol{B}}) \right)$.

- The model expression has four terms:
  - $n (F_u \cdot \nabla G_n - G_u \cdot \nabla F_n)$
  - $\frac{1}{\rho n} (\nabla \times u) \cdot (F_u \times G_u)$
  - $B \cdot \left( F_u \times (\nabla \times G_B) - G_u \times (\nabla \times F_B) \right)$
  - $-\frac{1}{\mu_0 q_e Z_i n} B \cdot \left( (\nabla \times F_B) \times (\nabla \times G_B) \right)$

2. **Sign and ordering in the first term:**

- Ground truth first term is:
  \[
  -\mathsf{m}^{-1} \left( G_{\boldsymbol{u}_i} \cdot \nabla F_{n_i} + G_{n_i} \nabla \cdot F_{\boldsymbol{u}_i} \right)
  \]

- Model's first term is:
  \[
  n (F_u \cdot \nabla G_n - G_u \cdot \nabla F_n)
  \]

- The model's first term is structurally different: it has $n$ multiplying $(F_u \cdot \nabla G_n - G_u \cdot \nabla F_n)$, whereas the ground truth has $-\mathsf{m}^{-1}$ multiplying $(G_u \cdot \nabla F_n + G_n \nabla \cdot F_u)$.

- Also, the ground truth has a divergence term $G_n \nabla \cdot F_u$ which is missing in the model.

- The signs and order of $F$ and $G$ derivatives are different.

3. **Second term (vorticity term):**

- Ground truth:
  \[
  \frac{(\nabla \times \boldsymbol{u}_i)}{\mathsf{m} n_i} \cdot (G_{\boldsymbol{u}_i} \times F_{\boldsymbol{u}_i})
  \]

- Model:
  \[
  \frac{1}{\rho n} (\nabla \times u) \cdot (F_u \times G_u)
  \]

- The model swaps the order in the cross product: $(F_u \times G_u)$ vs $(G_u \times F_u)$.

- Since $\mathbf{a} \times \mathbf{b} = - \mathbf{b} \times \mathbf{a}$, this is a sign difference.

- Also, the coefficient differs: $\frac{1}{\rho n}$ vs $\frac{1}{\mathsf{m} n_i}$. Possibly $\rho$ corresponds to $\mathsf{m} n_i$ (mass density), so this may be consistent.

4. **Third term (magnetic term):**

- Ground truth:
  \[
  \frac{1}{\mathsf{m} n_i} \boldsymbol{B} \cdot \left( G_{\boldsymbol{u}_i} \times (\nabla \times F_{\boldsymbol{B}}) - F_{\boldsymbol{u}_i} \times (\nabla \times G_{\boldsymbol{B}}) \right)
  \]

- Model:
  \[
  B \cdot \left( F_u \times (\nabla \times G_B) - G_u \times (\nabla \times F_B) \right)
  \]

- The order inside the parentheses is reversed and the signs are flipped compared to the ground truth.

- The ground truth has $G_u \times (\nabla \times F_B)$ minus $F_u \times (\nabla \times G_B)$, while the model has $F_u \times (\nabla \times G_B)$ minus $G_u \times (\nabla \times F_B)$.

- This is a sign difference.

- Also, the model is missing the factor $\frac{1}{\mathsf{m} n_i}$ in front of this term.

5. **Additional term in the model:**

- The model has an extra term:
  \[
  - \frac{1}{\mu_0 q_e Z_i n} B \cdot \left( (\nabla \times F_B) \times (\nabla \times G_B) \right)
  \]

- This term is not present in the ground truth.

---

**Summary:**

- The model expression differs in the first term: missing divergence term and different sign and structure.

- The vorticity term has the cross product reversed, which changes the sign.

- The magnetic term has reversed order and missing factor.

- The model includes an extra term not in the ground truth.

- The coefficients differ in notation but may be consistent if $\rho = \mathsf{m} n_i$.

- Overall, there are multiple errors: missing terms, sign errors, and an extra term.

---

**Conclusion:**

The model expression is not mathematically equivalent to the ground truth. It has more than one minor error (missing divergence term, sign errors in cross products, missing factor, and an extra term). Therefore, it cannot receive partial credit.

---

**Score:**

\[
\boxed{0}
\]

Score: 0

\newpage
\subsection*{Problem 55 (Paper: 2506.24021v1)}
\subsubsection*{Problem Statement}
Background:
Consider a physical system whose state is described by a particle's position $\Xb \in \mathbb{R}^d$. The system is in contact with a thermal bath at temperature $T$. A process of duration $\tau$ transforms the probability distribution of the particle's position from an initial state $\rho_0(\Xb)$ to a final state $\rho_\tau(\Xb)$. The minimum work required for this transformation is bounded by the thermodynamic speed limit:
$$W \geq T \Dkl(\rho_\tau || \rho_0) + \frac{1}{\tau}\was^2(\rho_\tau, \rho_0),$$
where $\Dkl(\nu || \mu) = \int \nu(\xb)\log\left(\frac{\nu(\xb)}{\mu(\xb)}\right) \der \xb$ is the Kullback-Leibler (KL) divergence, and $\was^2(\nu,\mu) = \inf_{\pi \in \Pi(\nu,\mu)} \int \|\xb - \yb\|^2 \pi(\xb, \yb) \der \xb \der \yb$ is the squared 2-Wasserstein distance over all joint distributions $\pi$ with marginals $\nu$ and $\mu$.

We define two logical operations based on this principle:
1.  **One-bit erasure (1BE)**: The system is one-dimensional ($d=1$). A bit is encoded by the sign of the particle's position $x$. The initial distribution $\rho_0^x(x)$ is symmetric, i.e., $\rho_0^x(x) = \rho_0^x(-x)$. The erasure operation constrains the final distribution $\rho_\tau^x(x)$ to be supported only on the positive real axis, i.e., its support is a subset of $\mathbb{R}^+$. The minimal work for this operation is given by the infimum over all such valid final distributions:
    $$W^{\rm 1BE}(\rho_0^x) = \underset{\rho_\tau^x : \supp(\rho_\tau^x) \subseteq \mathbb{R}^+}{\inf}\left[T \Dkl(\rho_\tau^x || \rho_0^x) + \frac{1}{\tau}\was^2(\rho_\tau^x, \rho_0^x)\right].$$

2.  **Two-bit erasure (2BE)**: The system is two-dimensional ($d=2$). Two bits are encoded by the signs of the coordinates $(x,y)$. The initial distribution is a product of two independent, symmetric 1D distributions, $\rho_0(x,y) = \rho_0^x(x)\rho_0^y(y)$, where $\rho_0^x(x) = \rho_0^x(-x)$ and $\rho_0^y(y) = \rho_0^y(-y)$. The erasure operation constrains the final distribution $\rho_\tau(x,y)$ to be supported only in the quadrant where $x>0$ and $y<0$. The minimal work is
    $$W^{\rm 2BE}(\rho_0) = \underset{\rho_\tau : \supp(\rho_\tau) \subseteq \mathbb{R}^+ \times \mathbb{R}^-}{\inf}\left[T \Dkl(\rho_\tau || \rho_0) + \frac{1}{\tau}\was^2(\rho_\tau, \rho_0)\right].$$

You may use the following properties without proof:
- For any distribution $p(x,y)$ and a product distribution $q(x,y)=q_x(x)q_y(y)$, the KL divergence can be decomposed as: $\Dkl(p || q) = \Dkl(p || p_x p_y) + \Dkl(p_x || q_x) + \Dkl(p_y || q_y)$, where $p_x, p_y$ are the marginals of $p$. The term $\Dkl(p || p_x p_y)$, representing mutual information, is non-negative.
- For any two distributions $\pi$ and $\mu$ on $\mathbb{R}^2$, $\was^2(\pi, \mu) \geq \was^2(\pi_x, \mu_x) + \was^2(\pi_y, \mu_y)$, where subscripts denote marginal distributions.
- The squared 2-Wasserstein distance between two product measures $\mu = \mu_x \otimes \mu_y$ and $\nu = \nu_x \otimes \nu_y$ is additive: $\was^2(\mu, \nu) = \was^2(\mu_x, \nu_x) + \was^2(\mu_y, \nu_y)$.

Task:
For a system with an initial distribution $\rho_0(x,y) = \rho_0^x(x)\rho_0^y(y)$, derive the exact relationship between the minimal work for two-bit erasure, $W^{\rm 2BE}(\rho_0)$, and the minimal works for one-bit erasure of the corresponding marginals, $W^{\rm 1BE}(\rho_0^x)$ and $W^{\rm 1BE}(\rho_0^y)$.

\subsubsection*{Ground Truth Solution}
\[ \boxed{W^{\rm 2BE}(\rho_0^x\rho_0^y) = W^{\rm 1BE}(\rho_0^x) + W^{\rm 1BE}(\rho_0^y)} \]

\subsubsection*{Model Outputs}
\subsubsection*{Model: o4-mini (Score: 0.0)}
\paragraph*{Model Solution:}
\[ %
W^{\rm 2BE}(\rho_0)
\;=\;
W^{\rm 1BE}(\rho_0^x)\;+\;W^{\rm 1BE}(\rho_0^y)\,.% \]

\paragraph*{Judge's Evaluation:}

The ground truth expression is:
\[
W^{\rm 2BE}(\rho_0^x \rho_0^y) = W^{\rm 1BE}(\rho_0^x) + W^{\rm 1BE}(\rho_0^y)
\]

The model-generated expression is:
\[
W^{\rm 2BE}(\rho_0) = W^{\rm 1BE}(\rho_0^x) + W^{\rm 1BE}(\rho_0^y)
\]

The difference lies in the argument of the left-hand side function \( W^{\rm 2BE} \). The ground truth has the argument as the product \(\rho_0^x \rho_0^y\), while the model uses only \(\rho_0\), which is ambiguous and does not explicitly represent the product.

Since the argument is a crucial part of the expression and the model's expression omits the product, this is a significant deviation from the ground truth. The rest of the expression matches exactly.

Therefore, this is more than a minor error; it changes the meaning of the expression.

**Score: 0**

\newpage
\subsection*{Problem 56 (Paper: 2506.24021v1)}
\subsubsection*{Problem Statement}
Background:
Consider a physical system described by the position of a particle $\mathbf{X} \in \mathbb{R}^2$. The system is to be driven from an initial probability distribution $\rho_0(\mathbf{X}_0)$ to a target distribution $\pi_\tau(\mathbf{X}_\tau)$ that realizes a "partial bit erasure" operation. This operation is defined by the constraint that the final distribution must have a specific total probability mass $\alpha$ in a designated region $\mathcal{C}_{00} \subset \mathbb{R}^2$, i.e., $\int_{\mathcal{C}_{00}} \pi_\tau(\mathbf{X}) \, d\mathbf{X} = \alpha$. The optimal protocol for this transformation can be found by solving a regularized unbalanced optimal transport problem. The goal is to find the coupling $\pi(\mathbf{X}_\tau, \mathbf{X}_0)$, a joint probability distribution, that minimizes the objective functional:
\begin{align*}
\mathcal{J}[\pi] = F_0(\pi_0) + F_\tau(\pi_\tau) + \epsilon \, \mathrm{D_{KL}}(\pi \,\|\, e^{-c/\epsilon})
\end{align*}
Here, $\pi_0$ and $\pi_\tau$ are the marginals of $\pi$, $\epsilon > 0$ is a regularization parameter, $c(\mathbf{X}_\tau, \mathbf{X}_0) = \tau^{-1}\lVert\mathbf{X}_0 - \mathbf{X}_\tau\rVert^2$ is the transport cost over a duration $\tau$, and $\mathrm{D_{KL}}$ is the Kullback-Leibler divergence. The functionals $F_0$ and $F_\tau$ enforce the physical constraints:
1.  $F_0(p) = \iota_{\rho_0}(p)$, where $\iota_{\rho_0}(p)$ is an indicator function that is zero if $p = \rho_0$ and $+\infty$ otherwise, fixing the initial marginal.
2.  $F_\tau(p) = T \, \mathrm{D_{KL}}(p \,\|\, \rho_0) + \iota_{\alpha}(p)$, where $T$ is temperature and $\iota_{\alpha}(p)$ is an indicator function enforcing the partial erasure constraint $\int_{\mathcal{C}_{00}} p(\mathbf{X}) \, d\mathbf{X} = \alpha$.

The optimal coupling $\pi^\epsilon$ has the form $\pi^\epsilon(\mathbf{X}_\tau, \mathbf{X}_0) = e^{u(\mathbf{X}_\tau)/\epsilon} K(\mathbf{X}_\tau, \mathbf{X}_0) e^{v(\mathbf{X}_0)/\epsilon}$, where $K(\mathbf{X}_\tau, \mathbf{X}_0) = e^{-c(\mathbf{X}_\tau, \mathbf{X}_0)/\epsilon}$ and $u, v$ are dual potentials found via an iterative scheme. The update for the target potential $u^{\ell+1}$ given the source potential $v^{\ell}$ is found by solving the maximization problem:
\begin{align*}
u^{\ell+1} = \underset{u}{\rm argmax}\left[-\widetilde{F}_\tau(-u) - \epsilon \langle e^{u/\epsilon}, (K * e^{v^\ell/\epsilon}) \rangle\right]
\end{align*}
where $\widetilde{F}_\tau(u) = \sup_p \left[ \langle p, u \rangle - F_\tau(p) \right]$ is the Legendre-Fenchel transform, $\langle f, g \rangle = \int f(\mathbf{X}) g(\mathbf{X}) \, d\mathbf{X}$, and $(K * s)(\mathbf{X}_\tau) = \int K(\mathbf{X}_\tau, \mathbf{X}_0) s(\mathbf{X}_0) \, d\mathbf{X}_0$.

Task:
Derive the explicit update rule for the dual potential $u(\mathbf{X}_\tau)$. The derivation involves calculating the Legendre-Fenchel transform $\widetilde{F}_\tau$ by introducing a Lagrange multiplier to handle the constraint $\iota_\alpha(p)$, and then solving the $\operatorname{argmax}$ problem for $u$. Your final expression for $u(\mathbf{X}_\tau)$ should be given in terms of $\rho_0(\mathbf{X}_\tau)$, the kernel operator $K$, the previous potential $v^\ell(\mathbf{X}_0)$, the parameters $T$, $\epsilon$, $\alpha$, and the indicator function $\mathbbm{1}_{\mathcal{C}_{00}}(\mathbf{X}_\tau)$. The expression will depend on the term $A[-u] = \int_{\mathcal{C}_{00}} \rho_0(\mathbf{X}_\tau) e^{-u(\mathbf{X}_\tau)/T} \, d\mathbf{X}_\tau$, which arises from the Lagrange multiplier constraint.

Solution:

\subsubsection*{Ground Truth Solution}
\[ \boxed{u(\mathbf{X}_\tau) = \frac{T \epsilon}{T + \epsilon}\left[\log(\rho_0(\mathbf{X}_\tau)) - \log\left((K*e^{v^\ell/\epsilon})(\mathbf{X}_\tau)\right) + \mathbbm{1}_{\mathcal{C}_{00}}(\mathbf{X}_\tau) \left(\log(\alpha) - \log(A[-u])\right)\right]} \]

\subsubsection*{Model Outputs}
\subsubsection*{Model: o4-mini (Score: 1.0)}
\paragraph*{Model Solution:}
\[ u^{\ell+1}(\mathbf X_\tau)
=
\frac{\epsilon\,T}{\epsilon+T}\,
\Biggl[\,
\ln\!\bigl(\tfrac{\rho_{0}(\mathbf X_\tau)}{(K*e^{v^\ell/\epsilon})(\mathbf X_\tau)}\bigr)
\;+\;
\mathbbm{1}_{\mathcal C_{00}}(\mathbf X_\tau)\,
\ln\!\bigl(\tfrac{\alpha}{A[-\,u^{\ell+1}]}\bigr)
\Biggr]. \]

\paragraph*{Judge's Evaluation:}

The ground truth expression is:
\[
u(\mathbf{X}_\tau) = \frac{T \epsilon}{T + \epsilon} \left[ \log(\rho_0(\mathbf{X}_\tau)) - \log\left((K * e^{v^\ell/\epsilon})(\mathbf{X}_\tau)\right) + \mathbbm{1}_{\mathcal{C}_{00}}(\mathbf{X}_\tau) \left( \log(\alpha) - \log(A[-u]) \right) \right]
\]

The model-generated expression is:
\[
u^{\ell+1}(\mathbf{X}_\tau) = \frac{\epsilon T}{\epsilon + T} \left[ \ln\left( \frac{\rho_0(\mathbf{X}_\tau)}{(K * e^{v^\ell/\epsilon})(\mathbf{X}_\tau)} \right) + \mathbbm{1}_{\mathcal{C}_{00}}(\mathbf{X}_\tau) \ln\left( \frac{\alpha}{A[-u^{\ell+1}]} \right) \right]
\]

**Comparison:**

1. **Prefactor:** Both have the factor \(\frac{T \epsilon}{T + \epsilon}\) (order of multiplication does not matter), so this matches exactly.

2. **Logarithm terms inside the bracket:**

   - Ground truth writes:
   \[
   \log(\rho_0(\mathbf{X}_\tau)) - \log\left((K * e^{v^\ell/\epsilon})(\mathbf{X}_\tau)\right)
   \]
   which is equivalent to
   \[
   \log\left(\frac{\rho_0(\mathbf{X}_\tau)}{(K * e^{v^\ell/\epsilon})(\mathbf{X}_\tau)}\right)
   \]
   as in the model expression.

3. **Indicator term:**

   - Ground truth:
   \[
   \mathbbm{1}_{\mathcal{C}_{00}}(\mathbf{X}_\tau) \left( \log(\alpha) - \log(A[-u]) \right)
   \]
   - Model:
   \[
   \mathbbm{1}_{\mathcal{C}_{00}}(\mathbf{X}_\tau) \ln\left( \frac{\alpha}{A[-u^{\ell+1}]} \right)
   \]
   which is equivalent.

4. **Variable names:**

   - The ground truth uses \(u(\mathbf{X}_\tau)\) and \(u\) inside \(A[-u]\).
   - The model uses \(u^{\ell+1}(\mathbf{X}_\tau)\) and \(u^{\ell+1}\) inside \(A[-u^{\ell+1}]\).
   
   This is acceptable as the problem states differences in variable names are allowed.

5. **Log vs ln:**

   - The ground truth uses \(\log\), the model uses \(\ln\).
   - This is a standard notation difference and does not affect correctness.

6. **Signs and terms:**

   - The model expression correctly combines the logarithms inside a single \(\ln\) with division, matching the subtraction in the ground truth.
   - The indicator term is added, not subtracted, consistent with the ground truth.

**Conclusion:**

The model-generated expression is mathematically equivalent to the ground truth expression, with only differences in notation and variable indices, which are allowed.

**Score: 1**

---

Score: 1

\newpage
\subsection*{Problem 57 (Paper: 2506.24035v1)}
\subsubsection*{Problem Statement}
Background:
Consider a physical system described by a single order parameter $\phi$. The system is in contact with a thermal bath and subject to an external field $H$. The free energy density of the system is given by the Landau-Ginzburg form:
$f_4(\phi, H) = \frac{1}{2} a_2 \phi^2 + \frac{1}{4} a_4 \phi^4 - H \phi$
where $a_2$ and $a_4$ are constants. The dynamics of the order parameter are stochastic and can be described by a Fokker-Planck equation for the probability density $P(\phi, t)$:
$\frac{\partial P(\phi, t)}{\partial t} = \lambda \frac{\partial}{\partial \phi}\left[\frac{\partial f_4(\phi,H)}{\partial \phi} P(\phi,t)\right] + \lambda\sigma^2\frac{\partial^2P(\phi,t)}{\partial\phi^2}$
Here, $\lambda$ is a kinetic coefficient setting a characteristic timescale, and $\sigma$ is the noise strength. The ensemble average of any function of the order parameter, $\mathcal{O}(\phi)$, is defined as $\langle\mathcal{O}\rangle = \int_{-\infty}^{\infty} \mathcal{O}(\phi) P(\phi, t) d\phi$. The system is driven by linearly sweeping the external field $H$ in time $t$ at a constant rate, such that $dH/dt = \lambda v_H$, where $v_H$ is the dimensionless driving rate. Assume that the probability distribution $P(\phi, t)$ and its derivatives vanish at $\phi \to \pm\infty$.

Task:
Derive the differential equation that governs the evolution of the ensemble-averaged order parameter $\langle\phi\rangle$ with respect to the external field $H$. Your final expression should relate $d\langle\phi\rangle/dH$ to $\langle\phi\rangle$, $\langle\phi^3\rangle$, $H$, and the parameters of the model.

\subsubsection*{Ground Truth Solution}
\[ \boxed{v_H \frac{d\langle\phi\rangle}{dH} = -a_2 \langle \phi \rangle - a_4 \langle \phi^3\rangle + H} \]

\subsubsection*{Model Outputs}
\subsubsection*{Model: o4-mini (Score: 0.0)}
\paragraph*{Model Solution:}
\[ \frac{d\langle\phi\rangle}{dH}
\;=\;
\frac{\,a_{2}\langle\phi\rangle \;+\;a_{4}\langle\phi^{3}\rangle \;-\;H\,}{v_{H}} \]

\paragraph*{Judge's Evaluation:}

The ground truth expression is:
\[
v_H \frac{d\langle \phi \rangle}{dH} = -a_2 \langle \phi \rangle - a_4 \langle \phi^3 \rangle + H
\]

Rearranging for \(\frac{d\langle \phi \rangle}{dH}\), we get:
\[
\frac{d\langle \phi \rangle}{dH} = \frac{-a_2 \langle \phi \rangle - a_4 \langle \phi^3 \rangle + H}{v_H}
\]

The model-generated expression is:
\[
\frac{d\langle \phi \rangle}{dH} = \frac{a_2 \langle \phi \rangle + a_4 \langle \phi^3 \rangle - H}{v_H}
\]

Comparing the two:

- The model's numerator has the opposite signs for each term compared to the ground truth.
- Specifically, the ground truth has \(-a_2 \langle \phi \rangle - a_4 \langle \phi^3 \rangle + H\), while the model has \(+a_2 \langle \phi \rangle + a_4 \langle \phi^3 \rangle - H\).
- This is more than a minor sign error; all terms have flipped signs.
- Therefore, the model expression is not mathematically equivalent to the ground truth.

**Conclusion:** The model expression is incorrect due to multiple sign errors.

Score: 0

\newpage
\subsection*{Problem 58 (Paper: 2506.24035v1)}
\subsubsection*{Problem Statement}
Background:
Consider a system described by a stochastic order parameter $\phi$ whose dynamics are governed by a potential energy landscape that exhibits a first-order phase transition. Near a spinodal point $\left(H_{\mathrm{sp}}, \phi_{\mathrm{sp}}\right)$, the dynamics of the deviation of the order parameter, $\varphi \equiv \phi - \phi_{\mathrm{sp}}$, driven by a deviation in an external field, $h \equiv H - H_{\mathrm{sp}}$, can be approximated. When the field is swept linearly with a driving rate $v_H$, such that a field increment is $\delta h = \lambda v_H \delta t$ for some timescale $\lambda$, the discretized Langevin equation for the increment $\delta\varphi$ is given by:
\begin{equation*}
    \delta \varphi = - a_3 \varphi^2 v_H^{-1} \delta h + h v_H^{-1} \delta h + \sqrt{2 \sigma^2 v_H^{-1} \delta h}\;W
\end{equation*}
Here, $a_3$ is a constant related to the cubic term of the potential, $\sigma$ is the noise strength characterizing finite-size effects, and $W$ is a random variable from a standard normal distribution.

To analyze the finite-size effects, one can employ a scaling analysis inspired by renormalization group theory. A scaling transformation is defined where the noise strength is rescaled by a factor $\Sigma > 0$ as $\sigma \to \sigma' = \sigma \Sigma^{-1}$. Consequently, any other physical quantity $\mathcal{O}$ transforms as $\mathcal{O} \to \mathcal{O}' = \mathcal{O} \Sigma^{-[\mathcal{O}]}$, where $[\mathcal{O}]$ is the scaling dimension of $\mathcal{O}$ with respect to the noise strength. The scaling dimensions are determined by two principles:
1. The form of the dynamical equation must remain invariant under this scaling transformation.
2. The scaling dimension of the coefficient of the highest-order nonlinear term in $\varphi$, which is $a_3$, is zero, i.e., $[a_3]=0$.

This analysis allows one to determine how the properties of the system's coercivity plateau, which appears in the presence of noise, depend on the noise strength $\sigma$. The plateau is characterized by its height $H_P$ and the characteristic driving rate $v_P$ at which it is most prominent. The scaling dimensions for $h$ and $v_H$ directly yield the power-law dependence of the plateau's height deviation from the deterministic spinodal field, $H^* - H_P$, and the characteristic rate $v_P$ on $\sigma$.

Task:
By applying the scaling transformation to the provided dynamical equation and using the stated principles, determine the scaling dimensions $[h]$, $[\varphi]$, and $[v_H]$. Use these dimensions to derive the complete scaling relation that describes how the coercivity plateau height deviation, $H^* - H_P$, and the characteristic driving rate, $v_P$, depend on the noise strength $\sigma$.

\subsubsection*{Ground Truth Solution}
\[ \boxed{H^* - H_{P} \sim \sigma^{4/3} \quad \text{at} \quad v_{P} \sim \sigma^2} \]

\subsubsection*{Model Outputs}
\subsubsection*{Model: o4-mini (Score: 1.0)}
\paragraph*{Model Solution:}
\[ H^*-H_P \;=\;C_h\,\sigma^{4/3}
\quad,\quad
v_P \;=\;C_v\,\sigma^2 \]

\paragraph*{Judge's Evaluation:}

The ground truth expression is given as an approximate relation:
\[
H^* - H_{P} \sim \sigma^{4/3} \quad \text{at} \quad v_{P} \sim \sigma^2
\]
which indicates proportionality without specifying constants.

The model-generated expression is:
\[
H^* - H_P = C_h \, \sigma^{4/3}, \quad v_P = C_v \, \sigma^2
\]
where \(C_h\) and \(C_v\) are constants.

**Comparison:**

- The ground truth uses the symbol \(\sim\) indicating proportionality or scaling, not equality.
- The model uses equality with unspecified constants \(C_h\) and \(C_v\), which is consistent with the proportionality in the ground truth.
- The powers of \(\sigma\) match exactly.
- The variables and notation are consistent aside from the constants.
- Introducing constants \(C_h\) and \(C_v\) is a standard way to write proportionality as equality with constants.

Therefore, the model expression is mathematically equivalent to the ground truth expression, just more explicit by including constants.

**Score justification:**

The model expression correctly captures the scaling relations and is equivalent to the ground truth. The difference in notation (using equality with constants instead of \(\sim\)) is acceptable and does not constitute an error.

**Score: 1**

\newpage
\subsection*{Problem 59 (Paper: 2506.24064v1)}
\subsubsection*{Problem Statement}
Background:
Consider a coarse-grained model for a double-stranded DNA molecule, known as the "b-model". In this framework, the state of the molecule is analyzed using a grand canonical ensemble, where the grand partition function $${\cal Z}(z) = \sum_{N=1}^{\infty} Z(N) z^N$$ is defined in terms of the canonical partition function $Z(N)$ for a molecule of length $N$ and a fugacity $z$ for the number of base pairs. The thermodynamic behavior of the infinitely long molecule is governed by the dominant singularity of $${\cal Z}(z)$$, which is the singularity $z_{\text{dom}}$ closest to the origin in the complex $z$-plane. A phase transition occurs when the location of this dominant singularity changes as a function of system parameters like temperature.

In the b-model, the bound (zipped) phase allows for the formation of denatured bubbles. The free energy of this phase is determined by a singularity $z_a$. When an external force $f$ is applied to pull the two strands apart, the molecule can enter an unzipped phase, where the two strands are separated and stretched. The free energy of this force-dominated state is determined by a different singularity, $z_c$. The phase boundary $f_c(T)$ between the zipped and unzipped phases is found by equating these two singularities, i.e., $z_a = z_c$.

The singularities for the two phases are given as follows:
1.  For the bound phase with bubbles (b-model), the singularity is $z_a = \sqrt{\frac{X-1}{X}} - \frac{X-1}{X}$. Here, $X = e^{\beta \epsilon}$, where $\epsilon > 0$ is the binding energy per base pair and $\beta = 1/(k_B T)$ is the inverse temperature.
2.  For the unzipped phase, where the two strands are stretched by a force $f$, the singularity is $z_c = [2(1+\cosh(\beta f))]^{-1}$.

Task:
Using the condition for the phase boundary, $z_a = z_c$, derive the expression for the critical unzipping force $f_c(T)$ as a function of temperature $T$ and the binding energy $\epsilon$.

\subsubsection*{Ground Truth Solution}
\[ \boxed{f_c(T)= k_B T \cosh^{-1}\left(\frac{1}{2} \frac{1}{\sqrt{1-e^{-\beta\epsilon}}-1+e^{-\beta\epsilon}} - 1\right )} \]

\subsubsection*{Model Outputs}
\subsubsection*{Model: o4-mini (Score: 1.0)}
\paragraph*{Model Solution:}
\[ f_{c}(T)
\;=\;\frac{1}{\beta}\,
\arccosh\!\Bigg[
\frac{1}{2\Bigl(\sqrt{1-e^{-\beta\epsilon}}
      \;-\;\bigl(1-e^{-\beta\epsilon}\bigr)\Bigr)}
\;-\;1
\Bigg] \]

\paragraph*{Judge's Evaluation:}

First, let's rewrite both expressions clearly:

**Ground Truth:**
\[
f_c(T) = k_B T \cosh^{-1}\left( \frac{1}{2} \frac{1}{\sqrt{1 - e^{-\beta \epsilon}} - 1 + e^{-\beta \epsilon}} - 1 \right)
\]

**Model-Generated:**
\[
f_c(T) = \frac{1}{\beta} \arccosh \left[ \frac{1}{2 \left( \sqrt{1 - e^{-\beta \epsilon}} - (1 - e^{-\beta \epsilon}) \right)} - 1 \right]
\]

---

**Step 1: Compare the prefactors**

- Ground truth uses \( k_B T \).
- Model uses \( \frac{1}{\beta} \).

Recall that \(\beta = \frac{1}{k_B T}\), so:
\[
\frac{1}{\beta} = k_B T
\]
Thus, the prefactors are equivalent.

---

**Step 2: Compare the inverse hyperbolic cosine functions**

- Ground truth uses \(\cosh^{-1}(\cdots)\).
- Model uses \(\arccosh(\cdots)\).

These are equivalent notations.

---

**Step 3: Compare the arguments inside the inverse hyperbolic cosine**

Ground truth argument:
\[
\frac{1}{2} \frac{1}{\sqrt{1 - e^{-\beta \epsilon}} - 1 + e^{-\beta \epsilon}} - 1
\]

Model argument:
\[
\frac{1}{2 \left( \sqrt{1 - e^{-\beta \epsilon}} - (1 - e^{-\beta \epsilon}) \right)} - 1
\]

Rewrite the denominator inside the model:
\[
\sqrt{1 - e^{-\beta \epsilon}} - (1 - e^{-\beta \epsilon}) = \sqrt{1 - e^{-\beta \epsilon}} - 1 + e^{-\beta \epsilon}
\]

This matches exactly the denominator in the ground truth.

Therefore, the arguments inside the inverse hyperbolic cosine are identical.

---

**Conclusion:**

- The prefactors are equivalent.
- The inverse hyperbolic cosine functions are the same.
- The arguments inside the functions are identical.

Hence, the model-generated expression is mathematically equivalent to the ground truth expression.

---

**Score: 1**

\newpage
\subsection*{Problem 60 (Paper: 2506.24064v1)}
\subsubsection*{Problem Statement}
Background:
Consider the force-induced unzipping of a double-stranded DNA molecule, which can be described as a first-order phase transition between a "zipped" (bound) phase and an "unzipped" (unbound) phase. The state of the system can be described by its Gibbs free energy $G$, which is a function of temperature $T$ and an external pulling force $f$ applied to the ends of the two strands. The differential of the free energy is given by $dG = -S dT - x df$, where $S$ is the entropy and $x$ is the end-to-end separation of the strands. The transition occurs along a phase boundary in the $(T,f)$ plane, denoted by the critical force $f_c(T)$, where the free energies of the two phases are equal: $G_z(T, f_c) = G_u(T, f_c)$, with subscripts $z$ and $u$ denoting the zipped and unzipped phases, respectively.

The following assumptions provide a thermodynamic model for this transition:
1.  The zipped phase is considered to be perfectly rigid and inextensible with respect to the pulling force. Therefore, its end-to-end separation is zero, $x_z = 0$, for any force up to the critical force, $0 \le f \le f_c(T)$.
2.  The unzipped phase consists of two separated polymer strands. Close to the zero-force melting temperature $T_c$, the critical force $f_c(T)$ is small. In this regime, the unzipped strands exhibit a linear elastic response to the force, such that their end-to-end separation is given by $x_u(T, f) = \chi_T f$. The coefficient $\chi_T$ is a temperature-dependent susceptibility.

Task:
Based on these thermodynamic principles and assumptions, derive the expression for the entropy difference between the two phases, $\Delta S \equiv S_z(T, f_c) - S_u(T, f_c)$, evaluated at the phase boundary. Your final expression should be in terms of the susceptibility $\chi_T$, the critical force $f_c(T)$, and its derivative with respect to temperature, $\frac{\partial f_c(T)}{\partial T}$.

\subsubsection*{Ground Truth Solution}
\[ \boxed{\Delta S = \chi_T f_c(T) \frac{\partial f_c(T)}{\partial T}} \]

\subsubsection*{Model Outputs}
\subsubsection*{Model: o4-mini (Score: 1.0)}
\paragraph*{Model Solution:}
\[ \Delta S \;\equiv\; S_z(T,f_c)\;-\;S_u(T,f_c)
\;=\;\chi_{T}\,f_c(T)\,\frac{d\,f_c(T)}{dT}\,. \]

\paragraph*{Judge's Evaluation:}

The ground truth expression is:
\[
\Delta S = \chi_T f_c(T) \frac{\partial f_c(T)}{\partial T}
\]

The model-generated expression is:
\[
\Delta S \equiv S_z(T,f_c) - S_u(T,f_c) = \chi_T f_c(T) \frac{d f_c(T)}{d T}
\]

**Comparison:**

1. The model expression introduces an equivalence definition for $\Delta S$ as a difference of two functions, $S_z$ and $S_u$, which is not present in the ground truth. However, this is just additional context and does not contradict the main expression.

2. The main part of the expression, $\chi_T f_c(T) \frac{d f_c(T)}{d T}$, matches the ground truth $\chi_T f_c(T) \frac{\partial f_c(T)}{\partial T}$ exactly in mathematical meaning. The use of total derivative $d/dT$ instead of partial derivative $\partial/\partial T$ is acceptable here since $f_c$ is a function of $T$ only.

3. Variable names and notation are consistent.

4. No missing terms, no incorrect coefficients or signs.

**Conclusion:**

The model-generated expression is mathematically equivalent to the ground truth expression, with some additional context that does not affect correctness.

**Score: 1**

Score: 1

\newpage
\subsection*{Problem 61 (Paper: 2506.24079v1)}
\subsubsection*{Problem Statement}
Background:
A quantum process is described by a quantum channel, which is a completely positive, trace-preserving (CPTP) linear map $\mathcal{N}_{A'\to A}$ acting on density operators. Consider a quantum system $A$ with a corresponding bounded Hamiltonian $\widehat{H}_A$. The mean energy of a channel $\mathcal{N}_{A'\to A}$ is defined as the maximum possible mean energy of its output states:
$$ \langle \widehat{H}\rangle_{\mathcal{N}} := \sup_{\rho \in \St(A')} \operatorname{tr}[\widehat{H}_A\mathcal{N}(\rho_{A'})] $$
where $\St(A')$ is the set of all density operators on the input Hilbert space $\mathcal{H}_{A'}$.

The entropy of a quantum channel, $S[\mathcal{N}]$, is a measure of the uncertainty it generates. It is related to the von Neumann entropy $S(\sigma) = -\operatorname{tr}[\sigma \log \sigma]$ of its possible output states by the following properties:
1.  For any channel $\mathcal{N}_{A'\to A}$, its entropy is bounded by the minimum entropy of any of its output states:
    $$ S[\mathcal{N}] \leq \inf_{\rho \in \St(A')} S(\mathcal{N}(\rho_{A'})) $$
2.  A special class of channels are the "replacer channels" $\mathcal{R}^{\omega}_{A'\to A}$, which produce a fixed output state $\omega_A \in \St(A)$ regardless of the input state, i.e., $\mathcal{N}(\rho_{A'}) = \omega_A$ for all $\rho \in \St(A')$. For these channels, the inequality above is saturated, and the channel entropy is simply the entropy of the fixed output state, $S[\mathcal{R}^{\omega}] = S(\omega_A)$. The mean energy of a replacer channel is $\langle \widehat{H}\rangle_{\mathcal{R}^{\omega}} = \operatorname{tr}[\widehat{H}_A \omega_A]$.

Furthermore, it is a known principle that for any given mean energy $E$, the state $\sigma_A \in \St(A)$ that maximizes the entropy $S(\sigma_A)$ subject to the constraint $\operatorname{tr}[\widehat{H}_A \sigma_A] = E$ is the unique thermal state $\gamma^\beta$, defined as:
$$ \gamma^\beta = \frac{\exp(-\beta \widehat{H}_A)}{\operatorname{tr}[\exp(-\beta \widehat{H}_A)]} $$
where the parameter $\beta$ (inverse temperature) is chosen such that $\langle \widehat{H}\rangle_{\gamma^{\beta}} = \operatorname{tr}[\widehat{H}_A \gamma^\beta] = E$.

Task:
Consider the set of all quantum channels $\mathcal{N}_{A'\to A}$ that have a fixed mean energy $\langle \widehat{H}\rangle_{\mathcal{N}} = E$. Using the principles and definitions provided, determine the maximum possible value of the channel entropy $S[\mathcal{N}]$ for this set of channels. Express your answer in terms of the von Neumann entropy of the thermal state $\gamma^\beta$ corresponding to the mean energy $E$.

\subsubsection*{Ground Truth Solution}
\[ \boxed{S(\gamma^\beta)} \]

\subsubsection*{Model Outputs}
\subsubsection*{Model: o4-mini (Score: 1.0)}
\paragraph*{Model Solution:}
\[ \max_{\mathcal{N}:\,\langle H\rangle_{\mathcal{N}}=E}S[\mathcal{N}]
\;=\;S\bigl(\gamma^\beta\bigr)
\;=\;\beta E\;+\;\ln\!\bigl(\tr e^{-\beta H}\bigr)\,. \]

\paragraph*{Judge's Evaluation:}

The ground truth expression is:
\[
S(\gamma^\beta)
\]

The model-generated expression is:
\[
\max_{\mathcal{N}:\,\langle H\rangle_{\mathcal{N}}=E} S[\mathcal{N}] = S\bigl(\gamma^\beta\bigr) = \beta E + \ln\!\bigl(\operatorname{tr} e^{-\beta H}\bigr).
\]

**Comparison and Justification:**

- The model-generated expression explicitly states the maximization problem and then identifies the maximum entropy state as \( \gamma^\beta \), which matches the argument of the entropy function in the ground truth.
- The model then provides the explicit formula for the entropy \( S(\gamma^\beta) \) in terms of \( \beta E + \ln(\operatorname{tr} e^{-\beta H}) \).
- The ground truth expression is a simpler form, just \( S(\gamma^\beta) \), while the model-generated expression is a more detailed and expanded version.
- Since the model-generated expression is mathematically equivalent to the ground truth expression (it includes the ground truth as a part and expands it correctly), it is completely correct.
- There are no errors, missing terms, or incorrect signs.

**Conclusion:**

The model-generated expression is fully consistent with the ground truth and provides a correct and equivalent expression.

Score: 1

\newpage
\subsection*{Problem 62 (Paper: 2506.24097v1)}
\subsubsection*{Problem Statement}
Background:
Consider a one-dimensional quantum circuit on an infinite lattice of qubits. The dynamics over one time step are described by a unitary propagator $U$, and the corresponding Heisenberg super-operator is $\mathcal{U}(A) \coloneq U^\dagger A U$. The system is translationally invariant by $s$ sites. Let $\mathcal{S}$ be the super-operator for a 1-site translation to the right. The Heisenberg propagator commutes with translations, i.e., $\mathcal{U}\mathcal{S}^j = \mathcal{S}^j\mathcal{U}$. We define a basis of "extensive observables" with a well-defined quasi-momentum $k \in [0, 2\pi)$. These are constructed from "local densities" $\{b\}$, which are operators with finite support. The basis elements for extensive observables are labeled by an intra-cell position $m \in \{0, \dots, s-1\}$ and a local density $b$, and are defined as $B_k^{(m, b)} \coloneq \sum_{j=-\infty}^{\infty} \mathrm{e}^{-\mathrm{i} k j} \mathcal{S}^{sj + m}(b)$. We define two Hilbert-Schmidt inner products. The "local" inner product for local operators $a, b'$ is $\langle a, b' \rangle \coloneq \lim_{N\to\infty} \frac{1}{2^N} \operatorname{tr}(a^\dagger b')$. This inner product is translationally invariant: $\langle \mathcal{S}^j(a), \mathcal{S}^j(b') \rangle = \langle a, b' \rangle$. The "extensive" inner product for extensive observables $A, B$ is defined as $\langle\langle A, B \rangle\rangle \coloneq \lim_{N\to\infty} \frac{s}{N} \langle A, B \rangle$. The basis $\{B_k^{(m,b)}\}$ is orthonormal with respect to this extensive inner product, i.e., $\langle\langle B_k^{(m, b)}, B_k^{(m', b')} \rangle\rangle = \delta_{m,m'} \delta_{b,b'}$.

Task:
The dynamics of extensive observables are described by the propagator of extensive observables, $\mathcal{U}_k$, which is the Heisenberg propagator $\mathcal{U}$ represented in the basis of extensive observables $\{B_k^{(m,b)}\}$. Find the general expression for the matrix elements $\left[\mathcal U_k\right]_{(m, b), (m', b')} \coloneq \langle\langle B^{(m, b)}_k, \mathcal U\left(B^{(m', b')}_k \right) \rangle\rangle$ in terms of the local inner product and local densities.


\subsubsection*{Ground Truth Solution}
\[ \boxed{\left[\mathcal U_k\right]_{(m, b), (m', b')} = \sum_{j} \mathrm{e}^{\mathrm{i} k j} \langle \mathcal S^{sj + m}(b), \mathcal U \left(\mathcal S^{m'}(b')\right) \rangle} \]

\subsubsection*{Model Outputs}
\subsubsection*{Model: o4-mini (Score: 0.0)}
\paragraph*{Model Solution:}
\[ \bigl[\Uc_{k}\bigr]_{(m,b),(m',b')}
=\sum_{j\in\Z}
e^{\iu k j}\,
\bigl\langle b,\;S^{s\,j+(m'-m)}\!\bigl(\Uc(b')\bigr)\bigr\rangle\,. \]

\paragraph*{Judge's Evaluation:}

First, let's rewrite both expressions clearly for comparison:

**Ground Truth:**
\[
\left[\mathcal{U}_k\right]_{(m,b),(m',b')} = \sum_j e^{i k j} \langle \mathcal{S}^{s j + m}(b), \mathcal{U}(\mathcal{S}^{m'}(b')) \rangle
\]

**Model-Generated:**
\[
\left[\mathcal{U}_k\right]_{(m,b),(m',b')} = \sum_{j \in \mathbb{Z}} e^{i k j} \langle b, \mathcal{S}^{s j + (m' - m)}(\mathcal{U}(b')) \rangle
\]

---

**Comparison and Analysis:**

1. **Summation index and exponential factor:**
   - Both sum over $j$ with the factor $e^{i k j}$, consistent.

2. **Inner product arguments:**

   - Ground truth inner product is:
     \[
     \langle \mathcal{S}^{s j + m}(b), \mathcal{U}(\mathcal{S}^{m'}(b')) \rangle
     \]
   
   - Model inner product is:
     \[
     \langle b, \mathcal{S}^{s j + (m' - m)}(\mathcal{U}(b')) \rangle
     \]

3. **Differences:**

   - The ground truth applies the shift operator $\mathcal{S}$ to $b$ on the **left** argument, and applies $\mathcal{S}^{m'}$ to $b'$ **inside** the argument of $\mathcal{U}$ on the right.

   - The model applies no shift to $b$ on the left argument (just $b$), but applies a shift $\mathcal{S}^{s j + (m' - m)}$ to $\mathcal{U}(b')$ on the right.

4. **Are these equivalent?**

   - The ground truth has the form:
     \[
     \langle \mathcal{S}^{s j + m}(b), \mathcal{U}(\mathcal{S}^{m'}(b')) \rangle
     \]
   
   - The model has:
     \[
     \langle b, \mathcal{S}^{s j + (m' - m)}(\mathcal{U}(b')) \rangle
     \]

   - If the inner product is shift-invariant or if $\mathcal{S}$ is unitary and the inner product satisfies:
     \[
     \langle \mathcal{S}^a x, y \rangle = \langle x, \mathcal{S}^{-a} y \rangle,
     \]
     then:
     \[
     \langle \mathcal{S}^{s j + m}(b), \mathcal{U}(\mathcal{S}^{m'}(b')) \rangle = \langle b, \mathcal{S}^{-(s j + m)} \mathcal{U}(\mathcal{S}^{m'}(b')) \rangle.
     \]

   - But the model has:
     \[
     \langle b, \mathcal{S}^{s j + (m' - m)}(\mathcal{U}(b')) \rangle,
     \]
     which is not the same as:
     \[
     \langle b, \mathcal{S}^{-(s j + m)} \mathcal{U}(\mathcal{S}^{m'}(b')) \rangle.
     \]

   - Also, the model applies $\mathcal{U}$ to $b'$ first, then shifts by $s j + (m' - m)$, whereas the ground truth applies $\mathcal{U}$ to $\mathcal{S}^{m'}(b')$.

5. **Conclusion:**

   - The model expression is not a simple renaming of variables; it changes the order of operators and the shifts applied.
   - The model's expression is not mathematically equivalent to the ground truth.
   - The model expression has at least two errors:
     - The shift on the left argument is missing.
     - The shift inside the argument of $\mathcal{U}$ is replaced by a shift after $\mathcal{U}$, and the shift exponent differs.

Therefore, the model expression is **incorrect**.

---

**Score: 0**

\newpage
\subsection*{Problem 63 (Paper: 2506.24097v1)}
\subsubsection*{Problem Statement}
Background:
Consider a one-dimensional many-body system in a statistical equilibrium state, characterized by an expectation value $\ev{\dots}$. The system possesses a single conserved quantity, whose local density is given by the operator $\sigma^z(x, t)$ at position $x$ and time $t$. The dynamics of this density is governed by a continuity equation,
\begin{equation}
    \partial_t \sigma^z(x, t) = -\partial_x j(x, t),
\end{equation}
where $j(x, t)$ is the corresponding local current density. On large spatio-temporal scales, the transport of the conserved quantity is diffusive. This is reflected in the equilibrium auto-correlation function of the local density, which is given by the Green's function of the diffusion equation:
\begin{equation}
    \ev{\sigma^z(x, t) \sigma^z(0, 0)} = \frac{1}{\sqrt{4 \pi D t}} \e{-\frac{x^2}{4Dt}},
\end{equation}
where $D$ is the diffusion constant. Assume that the equilibrium state is time-translationally and space-translationally invariant.

Task:
Using the provided relations for the diffusive dynamics of the conserved density, derive an expression for the equilibrium auto-correlation function of the local current, $\ev{j(x, t) j(0, 0)}$.

Solution:


\subsubsection*{Ground Truth Solution}
\[ \boxed{\frac{x^2 - 2 D t}{8\sqrt{\pi D} t^{5/2}} \e{-\frac{x^2}{4Dt}}} \]

\subsubsection*{Model Outputs}
\subsubsection*{Model: o4-mini (Score: 1.0)}
\paragraph*{Model Solution:}
\[ \ev{j(x,t)\,j(0,0)}
=
\frac{x^2-2Dt}{4\,t^2\,\sqrt{4\pi D t}}\exp\!\Bigl(-\frac{x^2}{4Dt}\Bigr)\,. \]

\paragraph*{Judge's Evaluation:}

Let's write both expressions clearly and compare them:

**Ground Truth:**
\[
\frac{x^2 - 2 D t}{8 \sqrt{\pi D} \, t^{5/2}} \, e^{-\frac{x^2}{4 D t}}
\]

**Model-Generated:**
\[
\frac{x^2 - 2 D t}{4 \, t^2 \, \sqrt{4 \pi D t}} \, \exp\left(-\frac{x^2}{4 D t}\right)
\]

---

### Step 1: Simplify the denominator of the model expression

The denominator is:
\[
4 \, t^2 \, \sqrt{4 \pi D t} = 4 t^2 \times \sqrt{4 \pi D t}
\]

Note that:
\[
\sqrt{4 \pi D t} = \sqrt{4} \sqrt{\pi} \sqrt{D} \sqrt{t} = 2 \sqrt{\pi D t}
\]

So the denominator becomes:
\[
4 t^2 \times 2 \sqrt{\pi D t} = 8 t^2 \sqrt{\pi D t}
\]

---

### Step 2: Compare with the ground truth denominator

Ground truth denominator:
\[
8 \sqrt{\pi D} \, t^{5/2} = 8 \sqrt{\pi D} \, t^{2.5} = 8 \sqrt{\pi D} \, t^2 t^{1/2} = 8 t^2 \sqrt{\pi D} \sqrt{t}
\]

Model denominator:
\[
8 t^2 \sqrt{\pi D t} = 8 t^2 \sqrt{\pi D} \sqrt{t}
\]

They are exactly the same.

---

### Step 3: Numerator and exponential

Both numerators are:
\[
x^2 - 2 D t
\]

Both exponentials are:
\[
e^{-\frac{x^2}{4 D t}}
\]

---

### Conclusion:

The model-generated expression is mathematically equivalent to the ground truth expression, just written in a slightly different but fully consistent form.

---

**Score: 1**

\newpage
\subsection*{Problem 64 (Paper: 2506.24115v1)}
\subsubsection*{Problem Statement}
Background:
Consider a 2D lattice model realizing the $D(S_3)$ topological phase, which is enriched by a global $\mathbb{Z}_2$ symmetry. The symmetry group of the underlying gauge theory is $S_3 = \{1, r, r^2, s, sr, rs\}$, with relations $r^3 = s^2 = (sr)^2 = 1$. The system possesses two distinct symmetry sectors, labeled $+$ and $-$, which are interchanged by the action of the non-trivial $\mathbb{Z}_2$ symmetry generator, denoted $\mathcal{G}_{em}$. The excitations of this model are anyons. We are interested in the non-Abelian anyon of type $D$. While the anyon type $D$ as a whole is invariant under the symmetry, its internal degrees of freedom transform non-trivially. The internal Hilbert space of a $D$-type anyon is 3-dimensional, spanned by an orthonormal basis $\{\ket{D_p}\}$, where the label $p$ is a "flux type" from the set $\{s, rs, sr\}$. A complete basis state for a $D$-anyon in a given sector is denoted $\ket{D_p, \sigma}$, where $p \in \{s, rs, sr\}$ and $\sigma \in \{+,-\}$.

The action of the symmetry operator $\mathcal{G}_{em}$ on the basis states of a $D$-anyon in the $+$ sector is given by a $3 \times 3$ unitary matrix $\rho^D(\mathcal{G}_{em})$:
$$ \mathcal{G}_{em}\ket{D_p, +} = \sum_{q \in \{s, rs, sr\}} [\rho^D(\mathcal{G}_{em})]_{qp} \ket{D_q, -} $$
Physically, this transformation is realized when an anyon crosses a domain wall separating the $+$ and $-$ sectors. There are two types of domain walls, $\alpha$ and $\beta$. The transformation $\mathcal{G}_{em}$ is defined by the process of an anyon crossing an $\alpha$-type domain wall. The amplitudes for these processes are given by the half-braiding tensors, or $z$-tensors, of the form $z^{(D,D);k}_{a,b,c}$. Here, $a$ and $b$ are the initial and final internal states of the anyon (e.g., $s_{++}$ denotes state $\ket{D_s,+}$), and $c$ and $k$ are the initial and final domain wall types (e.g., $\alpha_{+-}$ denotes an $\alpha$-wall separating the $+$ sector on the left from the $-$ sector on the right). The matrix element $[\rho^D(\mathcal{G}_{em})]_{qp}$ is given by the $z$-tensor for the process where an anyon in state $p_{++}$ crosses an $\alpha_{+-}$ domain wall and emerges in state $q_{--}$. Crucially, for the $D$-type anyon, this specific transformation process also changes the domain wall from type $\alpha$ to type $\beta$.

The non-zero values of the relevant $z$-tensors for the $(D,D)$ anyon interacting with domain walls are given below. Note that $p_{++}$ indicates a flux $p$ in the $+$ sector, $q_{--}$ indicates a flux $q$ in the $-$ sector, and so on.
\begin{align*}
& z^{(D,D);\beta_{+-}}_{s_{++},s_{--},\alpha_{+-}} = \frac{1}{\sqrt{3}} & & z^{(D,D);\alpha_{+-}}_{s_{++},s_{--},\beta_{+-}} = \frac{1}{\sqrt{3}} \\
& z^{(D,D);\beta_{+-}}_{s_{++},rs_{--},\alpha_{+-}} = \frac{1}{\sqrt{3}} & & z^{(D,D);\alpha_{+-}}_{s_{++},rs_{--},\beta_{+-}} = \frac{1}{\sqrt{3}} \\
& z^{(D,D);\beta_{+-}}_{s_{++},sr_{--},\alpha_{+-}} = \frac{1}{\sqrt{3}} & & z^{(D,D);\alpha_{+-}}_{s_{++},sr_{--},\beta_{+-}} = \frac{1}{\sqrt{3}} \\
& z^{(D,D);\beta_{+-}}_{rs_{++},s_{--},\alpha_{+-}} = \frac{1}{\sqrt{3}} & & z^{(D,D);\alpha_{+-}}_{rs_{++},s_{--},\beta_{+-}} = \frac{1}{\sqrt{3}} \\
& z^{(D,D);\beta_{+-}}_{rs_{++},rs_{--},\alpha_{+-}} = \frac{e^{i\frac{2\pi}{3}}}{\sqrt{3}} & & z^{(D,D);\alpha_{+-}}_{rs_{++},rs_{--},\beta_{+-}} = \frac{e^{-i\frac{2\pi}{3}}}{\sqrt{3}} \\
& z^{(D,D);\beta_{+-}}_{rs_{++},sr_{--},\alpha_{+-}} = \frac{e^{-i\frac{2\pi}{3}}}{\sqrt{3}} & & z^{(D,D);\alpha_{+-}}_{rs_{++},sr_{--},\beta_{+-}} = \frac{e^{i\frac{2\pi}{3}}}{\sqrt{3}} \\
& z^{(D,D);\beta_{+-}}_{sr_{++},s_{--},\alpha_{+-}} = \frac{1}{\sqrt{3}} & & z^{(D,D);\alpha_{+-}}_{sr_{++},s_{--},\beta_{+-}} = \frac{1}{\sqrt{3}} \\
& z^{(D,D);\beta_{+-}}_{sr_{++},rs_{--},\alpha_{+-}} = \frac{e^{-i\frac{2\pi}{3}}}{\sqrt{3}} & & z^{(D,D);\alpha_{+-}}_{sr_{++},rs_{--},\beta_{+-}} = \frac{e^{i\frac{2\pi}{3}}}{\sqrt{3}} \\
& z^{(D,D);\beta_{+-}}_{sr_{++},sr_{--},\alpha_{+-}} = \frac{e^{i\frac{2\pi}{3}}}{\sqrt{3}} & & z^{(D,D);\alpha_{+-}}_{sr_{++},sr_{--},\beta_{+-}} = \frac{e^{-i\frac{2\pi}{3}}}{\sqrt{3}}
\end{align*}
Additional non-zero tensors exist for the reverse process (crossing from $-$ to $+$), such as $z^{(D,D);\beta_{-+}}_{s_{--},s_{++},\alpha_{-+}}$, but are not listed here.

Task:
Using the basis ordering $(s, rs, sr)$ for the rows (final states $q$) and columns (initial states $p$), determine the transformation matrix $\rho^D(\mathcal{G}_{em})$.

\subsubsection*{Ground Truth Solution}
\[ \boxed{\rho^D(\mathcal{G}_{em})=\frac{1}{\sqrt{3}}\begin{pmatrix}
        1 & 1 & 1 \\
        1 & e^{i\frac{2\pi}{3}} & e^{-i\frac{2\pi}{3}} \\
        1 & e^{-i\frac{2\pi}{3}} & e^{i\frac{2\pi}{3}} \\
    \end{pmatrix}} \]

\subsubsection*{Model Outputs}
\subsubsection*{Model: o4-mini (Score: 1.0)}
\paragraph*{Model Solution:}
\[ \rho^D(\mathcal G_{em})
=\frac1{\sqrt3}
\begin{pmatrix}
1 & 1 & 1 \\[3pt]
1 & \omega & \omega^2 \\[3pt]
1 & \omega^2 & \omega
\end{pmatrix}
\,. \]

\paragraph*{Judge's Evaluation:}

The ground truth expression is:
\[
\rho^D(\mathcal{G}_{em}) = \frac{1}{\sqrt{3}} \begin{pmatrix}
1 & 1 & 1 \\
1 & e^{i\frac{2\pi}{3}} & e^{-i\frac{2\pi}{3}} \\
1 & e^{-i\frac{2\pi}{3}} & e^{i\frac{2\pi}{3}} \\
\end{pmatrix}
\]

The model-generated expression is:
\[
\rho^D(\mathcal{G}_{em}) = \frac{1}{\sqrt{3}} \begin{pmatrix}
1 & 1 & 1 \\
1 & \omega & \omega^2 \\
1 & \omega^2 & \omega \\
\end{pmatrix}
\]

Here, the model uses the notation \(\omega\) and \(\omega^2\) instead of the explicit exponentials \(e^{i\frac{2\pi}{3}}\) and \(e^{-i\frac{2\pi}{3}}\).

Recall that \(\omega = e^{i\frac{2\pi}{3}}\) is the standard notation for the primitive third root of unity, and \(\omega^2 = e^{i\frac{4\pi}{3}} = e^{-i\frac{2\pi}{3}}\).

Thus, the model's expression is exactly equivalent to the ground truth expression, just using a common shorthand notation for the roots of unity.

No errors or discrepancies are present.

**Score: 1**

\newpage
\subsection*{Problem 65 (Paper: 2506.24115v1)}
\subsubsection*{Problem Statement}
Background:
Consider a symmetry-enriched topological (SET) phase featuring non-Abelian anyons. One such anyon, of type $H$, possesses a two-dimensional internal Hilbert space. A basis for this space is given by the states $\{ \ket{H_r}, \ket{H_{r^2}} \}$, where the labels $r$ and $r^2$ are elements of the group $S_3$ satisfying $r^3=1$. The system is endowed with a global $\mathbb{Z}_2$ symmetry, whose non-trivial generator is denoted by $\mathcal{G}_{em}$. The action of this symmetry on the internal basis states of the $H$ anyon is given by a matrix representation $\rho^H(\mathcal{G}_{em})$. In the basis $(\ket{H_r}, \ket{H_{r^2}})$, the transformation is:
\begin{align*}
\mathcal{G}_{em}\ket{H_r} &= e^{-i\frac{2\pi}{3}}\ket{H_{r^2}} \\
\mathcal{G}_{em}\ket{H_{r^2}} &= e^{-i\frac{2\pi}{3}}\ket{H_r}
\end{align*}
Due to the categorical nature of the symmetry in this topological phase, the composition of two successive symmetry transformations, $\mathcal{G}_{em} \circ \mathcal{G}_{em}$, does not correspond to a simple matrix multiplication of their representations. Instead, the matrix representation of the composite transformation, which we denote $\rho^H(\mathcal{G}_{em} \circ \mathcal{G}_{em})$, is defined by a nonlinear composition rule. Its matrix elements are given by:
$$ (\rho^H(\mathcal{G}_{em} \circ \mathcal{G}_{em}))_{ac} = \sum_{b \in \{r, r^2\}} \omega_{ab} (\rho^H(\mathcal{G}_{em}))_{ab} (\rho^H(\mathcal{G}_{em}))_{bc} $$
where $a, b, c \in \{r, r^2\}$ are the basis-state labels. The rule involves a rank-2 tensor $\omega$, which encodes the effects of the underlying lattice geometry (Pachner moves). The relevant non-zero components of this tensor for the internal space of the $H$ anyon are:
\begin{align*}
\omega_{r,r} &= e^{i\frac{2\pi}{3}} \\
\omega_{r^2,r^2} &= e^{i\frac{2\pi}{3}} \\
\omega_{r,r^2} &= \omega_{r^2,r} = e^{-i\frac{2\pi}{3}}
\end{align*}
The composition of two $\mathcal{G}_{em}$ transformations is equivalent to the identity transformation, $\mathcal{G}_{em} \circ \mathcal{G}_{em} = 1$, whose representation is the identity matrix $\rho^H(1) = \idm$.

Task:
Using the provided nonlinear composition rule and the representation of a single symmetry transformation, derive the matrix representation for the composite transformation $\mathcal{G}_{em} \circ \mathcal{G}_{em}$ acting on the internal space of the H-type anyon.

\subsubsection*{Ground Truth Solution}
\[ \boxed{\begin{pmatrix} 1 & 0 \\ 0 & 1 \end{pmatrix}} \]

\subsubsection*{Model Outputs}
\subsubsection*{Model: o4-mini (Score: 1.0)}
\paragraph*{Model Solution:}
\[ \,\begin{pmatrix}1&0\\0&1\end{pmatrix}\! \]

\paragraph*{Judge's Evaluation:}

The Ground Truth Expression is 
\[
\begin{pmatrix} 1 & 0 \\ 0 & 1 \end{pmatrix}
\]

The Model-Generated Expression is 
\[
\begin{pmatrix}1&0\\0&1\end{pmatrix}
\]

Ignoring the spacing and the extra LaTeX spacing commands (\,, \!), the two matrices are exactly the same in terms of their elements and structure. The expressions are mathematically equivalent and represent the identity matrix of size 2.

**Score: 1**

\newpage

\end{document}
